% Template:     Informe/Reporte LaTeX
% Documento:    Archivo principal
% Versión:      5.1.5 (15/05/2018)
% Codificación: UTF-8
%
% Autor: Francisco Muñoz P.
%        Facultad de Ciencias Físicas y Matemáticas
%        Universidad de Chile
%        fjmunoz95@gmail.com , frmunoz@dcc.uchile.cl
%
% derechos del template utilizado corresponden a Pablo Pizarro.  pablo.pizarro@ing.uchile.cl
% Manual template: [http://latex.ppizarror.com/Template-Informe/]
% Licencia MIT:    [https://opensource.org/licenses/MIT/]

% CREACIÓN DEL DOCUMENTO
\documentclass[letterpaper,11pt]{article} % Articulo tamaño carta, 11pt
\usepackage[utf8]{inputenc} % Codificación UTF-8

% INFORMACIÓN DEL DOCUMENTO
\def\titulodelinforme {Matching Filter para la deteccion de eventos}
\def\temaatratar {Reporte de Avances}

\def\autordeldocumento {Francisco Muñoz}
\def\nombredelcurso {aaaa}
\def\codigodelcurso {cccc}

\def\nombreuniversidad {Universidad de Chile}
\def\nombrefacultad {Facultad de Ciencias Físicas y Matemáticas}
\def\departamentouniversidad { }
\def\imagendepartamento {departamentos/fcfm}
\def\imagendepartamentoescala {0.2}
\def\localizacionuniversidad {Santiago, Chile}

% INTEGRANTES, PROFESORES Y FECHAS
\def\tablaintegrantes {
\begin{tabular}{ll}
	Autor:
	& \begin{tabular}[t]{@{}l@{}}
		\autordeldocumento
	\end{tabular} \\
	Profesores Guia:
	& \begin{tabular}[t]{@{}l@{}}
		Francisco Förster \\
		Pablo Huijse
	\end{tabular} \\
	& \\
	\multicolumn{2}{l}{Ultima Actualización: \today} \\
	\multicolumn{2}{l}{\localizacionuniversidad}
\end{tabular}}{
}

% CONFIGURACIONES
% Template:     Informe/Reporte LaTeX
% Documento:    Configuraciones del template
% Versión:      5.4.0 (21/05/2018)
% Codificación: UTF-8
%
% Autor: Pablo Pizarro R. @ppizarror
%        Facultad de Ciencias Físicas y Matemáticas
%        Universidad de Chile
%        pablo.pizarro@ing.uchile.cl, ppizarror.com
%
% Manual template: [http://latex.ppizarror.com/Template-Informe/]
% Licencia MIT:    [https://opensource.org/licenses/MIT/]

% CONFIGURACIONES GENERALES
\def\addemptypagetwosides {false}  % Añade pag en blanco al imprimir a 2 caras
\def\columnsepwidth {2.1}          % Separación entre columnas [em]
\def\defaultinterline {1.0}        % Interlineado por defecto [pt]
\def\defaultnewlinesize {11}       % Tamaño del salto de línea [pt]
\def\documentlang {es-CL}          % Define el idioma del documento
\def\fontdocument {lmodern}        % Tipografía base, ver soportadas en manual
\def\fonttypewriter {tmodern}      % Tipografía de \texttt, ver manual
\def\importtikz {false}            % Utilizar la librería tikz
\def\numberedequation {true}       % Ecuaciones con \insert... numeradas
\def\pointdecimal {false}          % N° decimales con punto en vez de coma
\def\romanpageuppercase {false}    % Páginas en número romano en mayúsculas
\def\showlinenumbers {false}       % Muestra los n° de línea del documento
\def\tablepadding {1.0}            % Espaciado de celda de las tablas

% ESTILO PORTADA Y HEADER-FOOTER
\def\hfstyle {style9}              % Estilo header-footer (10 estilos disp.)
\def\portraitstyle {style9}        % Estilo portada (18 estilos disp.)

% MÁRGENES DE PÁGINA
\def\firstpagemargintop {3.8}      % Margen superior página portada [cm]
\def\pagemarginbottom {2.7}        % Margen inferior página [cm]
\def\pagemarginleft {2.54}         % Margen izquierdo página [cm]
\def\pagemarginright {2.54}        % Margen derecho página [cm]
\def\pagemargintop {3.0}           % Margen superior página [cm]

% POSICIÓN POR DEFECTO DE OBJETOS
\def\imagedefaultplacement {H}     % Posición por defecto de las imágenes
\def\tabledefaultplacement {H}     % Posición por defecto de las tablas
\def\tikzdefaultplacement {H}      % Posición por defecto de las figuras tikz

% CONFIGURACIÓN DE LAS LEYENDAS - CAPTION
\def\captionalignment {justified}  % Pos. ley. (centered,justified,left,right)
\def\captionlessmarginimage {0.1}  % Marg. sup/inf de fig. si no hay ley. [cm]
\def\captionlrmargin {2.0}         % Márgenes izq/der de la leyenda [cm]
\def\captionmarginmultimg {0.45}   % Margen izq/der leyendas múltiple img [cm]
\def\captiontbmarginfigure {9.35}  % Margen sup/inf de la leyenda en fig. [pt]
\def\captiontbmargintable {7.0}    % Margen sup/inf de la leyenda en tab. [pt]
\def\captiontextbold {false}       % Etiqueta (código,figura,tabla) en negrita
\def\codecaptiontop {true}         % Leyenda arriba del código fuente
\def\figurecaptiontop {false}      % Leyenda arriba de las imágenes
\def\showsectioncaption {sec}     % N° sec. en objeto (none,sec,ssec,sssec)
\def\tablecaptiontop {true}        % Leyenda arriba de las tablas

% CONFIGURACIÓN DEL ÍNDICE
\def\charafterobjectindex {.}      % Carácter después de n° figura,tabla,código
\def\charnumpageindex {.}          % Carácter número de página en índice
\def\equalmarginnumobject {true}   % Iguala el margen de los números de objetos
\def\indexdepth {3}                % Profundidad máxima del índice
\def\indextitlemargin {7.0}        % Mrg. título índice \insertindextitle [pt]
\def\showindex {true}              % Muestra el índice
\def\showindexofcode {true}        % Muestra la lista de códigos fuente
\def\showindexofcontents {true}    % Muestra la lista de contenidos
\def\showindexoffigures {true}     % Muestra la lista de figuras
\def\showindexoftables {true}      % Muestra la lista de tablas

% CONFIGURACIÓN DE LOS COLORES DEL DOCUMENTO
\def\captioncolor {black}          % Color de la ley. (código,figura,tabla)
\def\captiontextcolor {black}      % Color de la leyenda
\def\colorpage {white}             % Color de la página
\def\highlightcolor {yellow}       % Color del subrayado con \hl
\def\indextitlecolor {black}       % Color de los títulos del índice
\def\linenumbercolor {dgray}       % Color del n° de línea (\showlinenumbers)
\def\linkcolor {green}             % Color de los links del doc.
\def\maintextcolor {black}         % Color principal del texto
\def\numcitecolor {black}          % Color del n° de las referencias o citas
\def\portraittitlecolor {black}    % Color de los títulos de la portada
\def\showborderonlinks {true}     % Color de un link por un recuadro de color
\def\sourcecodebgcolor {lgray}     % Color de fondo del código fuente
\def\subsubtitlecolor {black}      % Color de los sub-subtítulos
\def\subtitlecolor {black}         % Color de los subtítulos
\def\tablelinecolor {black}        % Color de las líneas de las tablas
\def\titlecolor {black}            % Color de los títulos
\def\urlcolor {magenta}            % Color de los enlaces web (\url,\href)

% CONFIGURACIÓN DE FIGURAS
\def\defaultimagefolder {img/}     % Carpeta raíz de las imágenes
\def\marginbottomimages {-0.2}     % Margen inferior figura [cm]
\def\marginfloatimages {-13.0}     % Mrg. sup. fig \insertimageleft/right [pt]
\def\marginrightmultimage {0.1}    % Margen derecho imágenes múltiples [cm]
\def\margintopimages {0.0}         % Margen superior figura [cm]

% ANEXO, CITAS, REFERENCIAS
\def\apaciterefsep {9}             % Separación entre refs. {apacite} [pt]
\def\appendixindepobjnum {true}    % Anexo usa n° objetos independientes
\def\bibtexrefsep {9}              % Separación entre refs. {bibtex} [pt]
\def\donumrefsection {false}       % Sección de referencias numerada
\def\natbibrefsep {5}              % Separación entre referencia {natbib} [pt]
\def\sectionappendixlastchar {.}   % Carácter entre n° de sec. anexo y título
\def\showappendixsecindex {false}  % Título de la sec. de anexos en el índice
\def\showappendixsectitle {false}  % Título de la sec. de anexo en el informe
\def\stylecitereferences {bibtex}  % Estilo cita/ref. (bibtex,apacite,natbib)
\def\twocolumnreferences {true}    % Referencias en dos columnas

% CONFIGURACIÓN DE LOS TÍTULOS
\def\anumsecaddtocounter {true}    % Insertar títulos 'anum' aumenta n° de sec
\def\fontsizesubsubtitle {\large}  % Tamaño sub-subtítulos
\def\fontsizesubtitle {\Large}     % Tamaño subtítulos
\def\fontsizetitle {\huge}         % Tamaño títulos
\def\fontsizetitlei {\huge}        % Tamaño títulos en el índice
\def\showdotontitles {true}        % Punto al final de n° (ti/sub)título
\def\stylesubsubtitle {\bfseries}  % Estilo sub-subtítulos
\def\stylesubtitle {\bfseries}     % Estilo subtítulos
\def\styletitle {\bfseries}        % Estilo títulos
\def\styletitlei {\bfseries}       % Estilo títulos en el índice

% OPCIONES DEL PDF COMPILADO
\def\addindextobookmarks {false}   % Añade el índice a los marcadores del pdf
\def\cfgbookmarksopenlevel {1}     % Nivel marcadores en pdf (1:secciones)
\def\cfgpdfbookmarkopen {true}     % Expande marcadores del nivel configurado
\def\cfgpdfcenterwindow {true}     % Centra ventana del lector al abrir el pdf
\def\cfgpdfcopyright {}            % Establece el copyright del documento
\def\cfgpdfdisplaydoctitle {true}  % Muestra título del informe en visor
\def\cfgpdffitwindow {false}       % Ajusta la ventana del lector tamaño pdf
\def\cfgpdfmenubar {true}          % Muestra el menú del lector
\def\cfgpdfpagemode {OneColumn}    % Modo de página (OneColumn,SinglePage)
\def\cfgpdfpageview {FitH}         % Fit,FitH,FitV,FitR,FitB,FitBH,FitBV
\def\cfgpdfsecnumbookmarks {true}  % Número de la sec. en marcadores del pdf
\def\cfgpdftoolbar {true}          % Muestra barra de herramientas lector pdf
\def\cfgshowbookmarkmenu {true}    % Muestra menú marcadores al abrir el pdf
\def\pdfcompileversion {7}         % Versión mínima del pdf compilado (1.x)

% NOMBRE DE OBJETOS
\def\nameabstract {Abstract}           % Nombre del resumen-abstract
\def\nameappendixsection {Appendix}     % Nombre de la sec. de anexos/apéndices
\def\nameportraitpage {Portrait Page}       % Etiqueta página de la portada
\def\namereferences {References}     % Nombre de la sección de referencias
\def\nomchapter {Chapter}            % Nombre de los capítulos
\def\nomltappendixsection {Appendix}     % Etiqueta sección en anexo/apéndices
\def\nomltcont {Table of Contents} % Nombre del índice de contenidos
\def\nomltfigure {Table of Figures}   % Nombre del índice de lista de figuras
\def\nomltsrc {List of Source Code}      % Nombre del índice de lista de código
\def\nomlttable {List of Tables}     % Nombre del índice de lista de tablas
\def\nomltwfigure {Figure}            % Etiqueta leyenda de las figuras
\def\nomltwsrc {Code}               % Etiqueta leyenda del código fuente
\def\nomltwtable {Table}              % Etiqueta leyenda de las tablas


% IMPORTACIÓN DE LIBRERÍAS
% Template:     Informe/Reporte LaTeX
% Documento:    Importación de librerías
% Versión:      5.1.5 (15/05/2018)
% Codificación: UTF-8
%
% Autor: Pablo Pizarro R. @ppizarror
%        Facultad de Ciencias Físicas y Matemáticas
%        Universidad de Chile
%        pablo.pizarro@ing.uchile.cl, ppizarror.com
%
% Manual template: [http://latex.ppizarror.com/Template-Informe/]
% Licencia MIT:    [https://opensource.org/licenses/MIT/]

% LIBRERÍAS IMPORTANTES
\usepackage[spanish,es-nosectiondot,es-lcroman]{babel} % Define el idioma
\usepackage[T1]{fontenc} % Caracteres acentuados
\usepackage{ifthen} % Manejo de condicionales
\ifthenelse{\equal{\fontdocument}{arial}}{ % Tipografía del documento
	\usepackage{helvet}
	\renewcommand{\familydefault}{\sfdefault}
}{
	\usepackage{\fontdocument}
}
\ifthenelse{\equal{\showlinenumbers}{true}}{ % Muestra los números de línea
	\usepackage[switch,columnwise,running]{lineno}}{
}

% PARCHES DE LIBRERÍAS
\let\counterwithout\relax
\let\counterwithin\relax

% LIBRERÍAS INDEPENDIENTES
\usepackage{amsmath}       % Librerías matemáticas
\usepackage{amssymb}       % Librerías matemáticas
\usepackage{array}         % Nuevas características a las tablas
\usepackage{bigstrut}      % Líneas horizontales en tablas
\usepackage{bm}            % Caracteres en negrita en ecuaciones
\usepackage{booktabs}      % Permite manejar elementos visuales en tablas
\usepackage{caption}       % Leyendas
\usepackage{changepage}    % Condicionales para administrar páginas
\usepackage{chngcntr}      % Añade números a las leyendas
\usepackage{color}         % Colores
\usepackage{colortbl}      % Administración de color en las tablas
\usepackage{csquotes}      % Citas y comillas
\usepackage{datetime}      % Fechas
\usepackage{floatpag}      % Maneja números de páginas
\usepackage{floatrow}      % Permite adminisrar posiciones en los caption
\usepackage{framed}        % Permite creación de recuadros
\usepackage{gensymb}       % Simbología común
\usepackage{geometry}      % Dimensiones y geometría del documento
\usepackage{graphicx}      % Propiedades extra para los gráficos
\usepackage{lipsum}        % Permite crear párrafos de prueba
\usepackage{listings}      % Permite añadir código fuente
\usepackage{listingsutf8}  % Acepta codificación UTF-8 en código fuente
\usepackage{longtable}     % Permite utilizar tablas en varias hojas
\usepackage{lscape}        % Modo página horizontal de página
\usepackage{mathtools}     % Permite utilizar notaciones matemáticas
\usepackage{multicol}      % Múltiples columnas
\usepackage{needspace}     % Maneja los espacios en página
\usepackage{notoccite}     % Desactiva las citas en el índice
\usepackage{pdfpages}      % Permite administrar páginas en pdf
\usepackage{physics}       % Paquete de matemáticas
\usepackage{rotating}      % Permite rotación de objetos
\usepackage{sectsty}       % Cambia el estilo de los títulos
\usepackage{selinput}      % Compatibilidad con acentos
\usepackage{setspace}      % Cambia el espacio entre líneas
\usepackage{siunitx}       % Unidades del sistema internacional
\usepackage{soul}          % Permite subrayar texto
\usepackage{subfig}        % Permite agrupar imágenes
\usepackage{textcomp}      % Simbología común
\usepackage{url}           % Permite añadir enlaces
\usepackage{wasysym}       % Contiene caracteres misceláneos
\usepackage{wrapfig}       % Permite comprimir imágenes
\usepackage{xspace}        % Adminsitra espacios en párrafos y líneas

% LIBRERÍAS CON PARÁMETROS
\usepackage[makeroom]{cancel} % Cancelar términos en fórmulas
\usepackage[inline]{enumitem} % Permite enumerar ítems
\usepackage[bottom,norule,hang]{footmisc} % Estilo pie de página
\usepackage[subfigure,titles]{tocloft} % Maneja entradas en el índice
\usepackage[pdfencoding=auto,psdextra]{hyperref} % Enlaces, referencias
\usepackage[figure,table,lstlisting]{totalcount} % Contador de objetos
\usepackage[normalem]{ulem} % Permite tachar y subrayar
\usepackage[usenames,dvipsnames]{xcolor} % Paquete de colores avanzado

% LIBRERÍAS CONDICIONALES
\ifthenelse{\equal{\showdotontitles}{true}}{ % Agrega punto a títulos/subtítulos
	\usepackage{secdot}
	\sectiondot{subsection}
	\sectiondot{subsubsection}}{
}
\ifthenelse{\equal{\stylecitereferences}{natbib}}{ % Formato citas natbib
	\usepackage{natbib}
}{
	\ifthenelse{\equal{\stylecitereferences}{apacite}}{ % Formato citas apacite
		\usepackage{apacite}
	}{
		\ifthenelse{\equal{\stylecitereferences}{bibtex}}{ % Formato citas bibtex
		}{}
	}
}
\ifthenelse{\equal{\showappendixsecindex}{true}}{ % Anexos/Apéndices
	\usepackage[toc]{appendix}
}{
	\usepackage{appendix}
}
\ifthenelse{\equal{\importtikz}{true}}{ % Importa la librería tikz
	\usepackage{tikz}}{
}

% LIBRERÍAS DEPENDIENTES
\usepackage{bookmark}      % Administración de marcadores en pdf
\usepackage{fancyhdr}      % Encabezados y pie de páginas
\usepackage{float}         % Administrador de posiciones de objetos
\usepackage{hyperxmp}      % Etiquetas opcionales para el pdf compilado
\usepackage{multirow}      % Agrega nuevas opciones a las tablas
\usepackage{titlesec}      % Administración de títulos

\renewcommand{\thesection}{\Roman{section}}  % redefinicion de los numeros en section

% IMPORTACIÓN DE FUNCIONES Y ENTORNOS
% Template:     Informe/Reporte LaTeX
% Documento:    Carga las funciones del template
% Versión:      5.4.0 (21/05/2018)
% Codificación: UTF-8
%
% Autor: Pablo Pizarro R. @ppizarror
%        Facultad de Ciencias Físicas y Matemáticas
%        Universidad de Chile
%        pablo.pizarro@ing.uchile.cl, ppizarror.com
%
% Manual template: [http://latex.ppizarror.com/Template-Informe/]
% Licencia MIT:    [https://opensource.org/licenses/MIT/]

% Template:     Informe/Reporte LaTeX
% Documento:    Funciones del núcleo del template
% Versión:      5.4.0 (21/05/2018)
% Codificación: UTF-8
%
% Autor: Pablo Pizarro R. @ppizarror
%        Facultad de Ciencias Físicas y Matemáticas
%        Universidad de Chile
%        pablo.pizarro@ing.uchile.cl, ppizarror.com
%
% Manual template: [http://latex.ppizarror.com/Template-Informe/]
% Licencia MIT:    [https://opensource.org/licenses/MIT/]

\newcommand{\throwerror}[2]{\errmessage{LaTeX Error: \noexpand#1 #2 (linea \the\inputlineno)}\stop}\newcommand{\throwwarning}[1]{\errmessage{LaTeX Warning: #1 (linea \the\inputlineno)}}\newcommand{\throwbadconfigondoc}[3]{\errmessage{#1 \noexpand #2=#2. Valores esperados: #3}\stop}\newcommand{\checkvardefined}[1]{\ifthenelse{\isundefined{#1}}{\errmessage{LaTeX Warning: Variable \noexpand#1 no definida}\stop}{}}\newcommand{\checkextravarexist}[2]{\ifthenelse{\isundefined{#1}}{\errmessage{LaTeX Warning: Variable \noexpand#1 no definida}\ifx\hfuzz#2\hfuzz\errmessage{LaTeX Warning: Defina la variable en el bloque de INFORMACION DEL DOCUMENTO al comienzo del archivo principal del Template}\else\errmessage{LaTeX Warning: #2}\fi}{}}\newcommand{\emptyvarerr}[3]{\ifx\hfuzz#2\hfuzz\errmessage{LaTeX Warning: \noexpand#1 #3 (linea \the\inputlineno)}\fi}\newcommand{\setcaptionmargincm}[1]{\captionsetup{margin=#1cm}}\newcommand{\setpagemargincm}[4]{\newgeometry{left=#1cm, top=#2cm, right=#3cm, bottom=#4cm}}\newcommand{\changemargin}[2]{\emptyvarerr{\changemargin}{#1}{Margen izquierdo no definido}\emptyvarerr{\changemargin}{#2}{Margen derecho no definido}\list{}{\rightmargin#2\leftmargin#1}\item[]}\let\endchangemargin=\endlist\newcommand{\bgtemplatetestimg}{\definecolor{c040302}{RGB}{4, 3, 2}\definecolor{cb26929}{RGB}{178,105, 41}\definecolor{cfa943c}{RGB}{250, 148, 60}\definecolor{cfdc594}{RGB}{253, 197, 148}\begin{tikzpicture}[y=1.0pt, x=1.0pt, yscale=-1, xscale=1, inner sep=0pt, outer sep=0pt, opacity=0.1]\path[fill=cfdc594](46.3187,318.2550)..controls(46.0737,317.5650)and(45.7487,314.0750)..(45.5987,310.5)--(45.3247,304)--(37.8247,303.5)--(30.3247,303)--(29.8247,295.5)-- (29.3247,288)--(21.8247,287.5)--(14.3247,287)--(13.8247,271.5)--(13.3247,256)--(5.8247,255.5)-- (-1.6753,255)--(-1.6753,206.5)--(-1.6753,158)--(5.8247,157.5)--(13.3247,157)--(13.8247,141.5)-- (14.3247,126)--(21.8247,125.5)--(29.3247,125)--(29.8247,109.5)--(30.3247,94)--(37.8247,93.5)-- (45.3247,93)--(45.8247,69.5)--(46.3247,46)--(53.8247,45.5)--(61.3247,45)--(61.8247,29.5)-- (62.3247,14)--(69.8247,13.5)--(77.3247,13)--(77.8247,5.5)--(78.3247,-2)--(94.8247,-2)-- (111.3247,-2)--(111.8247,5.5)--(112.3247,13)--(119.8247,13.5)--(127.3247,14)--(127.8247,21.5)-- (128.3247,29)--(166.8247,29)--(205.3247,29)--(205.8247,21.5)--(206.3247,14)--(213.8247,13.5)-- (221.3247,13)--(221.8247,5.5)--(222.3247,-2)--(230.8247,-2)--(239.3247,-2)--(239.8247,5.5)-- (240.3247,13)--(247.8247,13.5)--(255.3247,14)--(255.8247,21.5)--(256.3247,29)--(263.8247,29.5)-- (271.3247,30)--(271.8247,45.5)--(272.3247,61)--(279.8247,61.5)--(287.3247,62)--(287.8247,85.5)-- (288.3247,109)--(295.8247,109.5)--(303.3247,110)--(303.8247,133.5)--(304.3247,157)--(311.8247,157.5)-- (319.3247,158)--(319.3247,198.5)--(319.3247,239)--(311.8247,239.5)--(304.3247,240)--(303.8247,255.5)-- (303.3247,271)--(295.8247,271.5)--(288.3247,272)--(287.8247,279.5)--(287.3247,287)--(279.8247,287.5)-- (272.3247,288)--(271.8247,295.5)--(271.3247,303)--(263.8247,303.5)--(256.3247,304)--(255.8247,311.5)-- (255.3247,319)--(151.0447,319.2550)..controls(68.1357,319.4570)and (46.6737,319.2530)..(46.3187,318.2550)--cycle(173.8247,142.5)-- (173.8247,127.5)--(166.8247,127.5)--(159.8247,127.5)-- (159.8247,142.5)--(159.8247,157.5)--(166.8247,157.5)-- (173.8247,157.5)--cycle(269.8247,126.5)--(269.8247,111.5)-- (262.8247,111.5)--(255.8247,111.5)--(255.8247,126.5)-- (255.8247,141.5)--(262.8247,141.5)--(269.8247,141.5)--cycle; \path[fill=cfa943c](47.3187,317.2550)..controls(47.0737,316.5650)and (46.7487,313.0750)..(46.5987,309.5)--(46.3247,303)--(38.8247,302.5)--(31.3247,302)--(30.8247,294.5)-- (30.3247,287)--(22.8247,286.5)--(15.3247,286)--(14.8247,270.5)--(14.3247,255)--(6.8247,254.5)-- (-0.6753,254)--(-0.6753,206.5)--(-0.6753,159)--(6.8247,158.5)--(14.3247,158)--(14.8247,142.5)-- (15.3247,127)--(22.8247,126.5)--(30.3247,126)--(30.8247,110.5)--(31.3247,95)--(38.8247,94.5)-- (46.3247,94)--(46.8247,70.5)--(47.3247,47)--(54.8247,46.5)--(62.3247,46)--(62.8247,30.5)-- (63.3247,15)--(70.8247,14.5)--(78.3247,14)--(78.8247,6.5)--(79.3247,-1)--(94.8247,-1)-- (110.3247,-1)--(110.8247,6.5)--(111.3247,14)--(118.8247,14.5)--(126.3247,15)--(126.8247,22.5)-- (127.3247,30)--(166.8247,30)--(206.3247,30)--(206.8247,22.5)--(207.3247,15)--(214.8247,14.5)-- (222.3247,14)--(222.8247,6.5)--(223.3247,-1)--(230.8247,-1)--(238.3247,-1)--(238.8247,6.5)-- (239.3247,14)--(246.8247,14.5)--(254.3247,15)--(254.8247,22.5)--(255.3247,30)--(262.8247,30.5)-- (270.3247,31)--(270.8247,46.5)--(271.3247,62)--(278.8247,62.5)--(286.3247,63)--(286.8247,86.5)-- (287.3247,110)--(294.8247,110.5)--(302.3247,111)--(302.8247,134.5)--(303.3247,158)--(310.8247,158.5)-- (318.3247,159)--(318.3247,198.5)--(318.3247,238)--(310.8247,238.5)--(303.3247,239)--(302.8247,254.5)-- (302.3247,270)--(294.8247,270.5)--(287.3247,271)--(286.8247,278.5)--(286.3247,286)--(278.8247,286.5)-- (271.3247,287)--(270.8247,294.5)--(270.3247,302)--(262.8247,302.5)--(255.3247,303)--(254.8247,310.5)-- (254.3247,318)--(151.0447,318.2550)..controls(68.9357,318.4570)and (47.6737,318.2520)..(47.3187,317.2550)--cycle(253.8247,294.70).. controls(253.8247,290.7440)and(254.3267,287.3980)..(255.0247,286.70).. controls(255.7227,286.0020)and(259.0687,285.5)..(263.0247,285.5)-- (269.8247,285.5)--(269.8247,278.70)..controls(269.8247,274.7440)and (270.3267,271.3980)..(271.0247,270.70)..controls(271.7227,270.0020)and (275.0687,269.5)..(279.0247,269.5)--(285.8247,269.5)-- (285.8247,254.70)..controls(285.8247,244.5220)and(286.1997,239.5250).. (287.0247,238.70)..controls(287.7227,238.0020)and(291.0687,237.5).. (295.0247,237.5)--(301.8247,237.5)--(301.8247,206)-- (301.8247,174.5)--(294.8247,174.5)--(287.8247,174.5)-- (287.8247,189.80)..controls(287.8247,200.3670)and(287.4537,205.4710).. (286.6247,206.30)..controls(285.9267,206.9980)and(282.5807,207.5).. (278.6247,207.5)--(271.8247,207.5)--(271.8247,214.30)..controls (271.8247,218.2560)and(271.3227,221.6020)..(270.6247,222.30)..controls (269.9267,222.9980)and(266.5807,223.5)..(262.6247,223.5)-- (255.8247,223.5)--(255.8247,230.3780)..controls(255.8247,234.9380)and (255.3687,237.6340)..(254.4727,238.3780)..controls(252.4847,240.0280)and (176.6787,239.9540)..(175.0247,238.30)..controls(174.3337,237.6090)and (173.8247,234.30)..(173.8247,230.5)..controls(173.8247,226.70)and (174.3337,223.3910)..(175.0247,222.70)..controls(175.9167,221.8080)and (186.1807,221.5)..(215.0247,221.5)--(253.8247,221.5)-- (253.8247,214.5)--(253.8247,207.5)--(249.0747,207.3720)..controls (246.4627,207.3020)and(235.9997,206.9640)..(225.8247,206.6220)-- (207.3247,206)--(206.8247,198.5)--(206.3247,191)--(190.8247,190.5)--(175.3247,190)--(174.8247,182.5)-- (174.3247,175)--(134.8247,175)--(95.3247,175)--(94.8247,182.5)--(94.3247,190)--(86.8247,190.5)-- (79.3247,191)--(79.3247,222.5)--(79.3247,254)--(86.8247,254.5)--(94.3247,255)--(94.8247,270.5)-- (95.3247,286)--(102.8247,286.5)--(110.3247,287)-- (110.6187,294.2500)--(110.9127,301.5)--(182.3687,301.5)-- (253.8247,301.5)--cycle(174.6247,158.30)..controls (176.4747,156.4500)and(176.2867,128.1290)..(174.4147,126.5750)..controls (173.5527,125.8600)and(170.3467,125.5210)..(166.1647,125.7030)-- (159.3247,126)--(159.0477,141.4580)..controls(158.8957,149.9600)and (158.9937,157.4970)..(159.2667,158.2080)..controls(159.9187,159.9070)and (172.9397,159.9850)..(174.6257,158.30)--cycle(270.6247,142.30).. controls(272.4747,140.4500)and(272.2867,112.1290)..(270.4147,110.5750).. controls(269.5527,109.8600)and(266.3467,109.5210)..(262.1647,109.7030)-- (255.3247,110)--(255.0477,125.4580)..controls(254.8957,133.9600)and (254.9937,141.4970)..(255.2667,142.2080)..controls(255.9187,143.9070)and (268.9397,143.9850)..(270.6257,142.30)--cycle(78.8247,86.5)-- (79.3247,79)--(86.8247,78.5)--(94.3247,78)--(94.8247,70.5)--(95.3247,63)--(102.8247,62.5)-- (110.3247,62)--(110.8247,54.5)--(111.3247,47)-- (118.5747,46.7060)--(125.8247,46.4120)--(125.8247,38.9560)-- (125.8247,31.5)--(119.0247,31.5)..controls(115.0687,31.5)and (111.7227,30.9980)..(111.0247,30.30)..controls(110.3267,29.6020)and (109.8247,26.2560)..(109.8247,22.30)--(109.8247,15.5)-- (94.8247,15.5)--(79.8247,15.5)--(79.8247,30.30)..controls (79.8247,40.4780)and(79.4497,45.4750)..(78.6247,46.30)..controls (77.9267,46.9980)and(74.5807,47.5)..(70.6247,47.5)-- (63.8247,47.5)--(63.8247,71.0440)--(63.8247,94.5880)-- (71.0747,94.2940)--(78.3247,94)--cycle(253.8247,39)-- (253.8247,31.5)--(247.0247,31.5)..controls(243.0687,31.5)and (239.7227,30.9980)..(239.0247,30.30)..controls(238.3267,29.6020)and (237.8247,26.2560)..(237.8247,22.30)--(237.8247,15.5)-- (230.8247,15.5)--(223.8247,15.5)--(223.8247,22.30)..controls (223.8247,26.2560)and(223.3227,29.6020)..(222.6247,30.30)..controls (221.9267,30.9980)and(218.5807,31.5)..(214.6247,31.5)-- (207.8247,31.5)--(207.8247,39)--(207.8247,46.5)-- (230.8247,46.5)--(253.8247,46.5)--cycle; \path[fill=cb26929](47.3187,317.2550)..controls(47.0737,316.5650)and (46.7487,313.0750)..(46.5987,309.5)--(46.3247,303)--(38.8247,302.5)--(31.3247,302)--(30.8247,294.5)-- (30.3247,287)--(22.8247,286.5)--(15.3247,286)--(14.8247,270.5)--(14.3247,255)--(6.8247,254.5)-- (-0.6753,254)--(-0.6753,206.5)--(-0.6753,159)--(6.8247,158.5)--(14.3247,158)--(14.8247,142.5)-- (15.3247,127)--(22.8247,126.5)--(30.3247,126)--(30.8247,110.5)--(31.3247,95)--(38.8247,94.5)-- (46.3247,94)--(46.8247,70.5)--(47.3247,47)--(54.8247,46.5)--(62.3247,46)--(62.8247,30.5)-- (63.3247,15)--(70.8247,14.5)--(78.3247,14)--(78.8247,6.5)--(79.3247,-1)--(94.8247,-1)-- (110.3247,-1)--(110.8247,6.5)--(111.3247,14)--(118.8247,14.5)--(126.3247,15)--(126.8247,22.5)-- (127.3247,30)--(166.8247,30)--(206.3247,30)--(206.8247,22.5)--(207.3247,15)--(214.8247,14.5)-- (222.3247,14)--(222.8247,6.5)--(223.3247,-1)--(230.8247,-1)--(238.3247,-1)--(238.8247,6.5)-- (239.3247,14)--(246.8247,14.5)--(254.3247,15)--(254.8247,22.5)--(255.3247,30)--(262.8247,30.5)-- (270.3247,31)--(270.8247,46.5)--(271.3247,62)--(278.8247,62.5)--(286.3247,63)--(286.8247,86.5)-- (287.3247,110)--(294.8247,110.5)--(302.3247,111)--(302.8247,134.5)--(303.3247,158)--(310.8247,158.5)-- (318.3247,159)--(318.3247,198.5)--(318.3247,238)--(310.8247,238.5)--(303.3247,239)--(302.8247,254.5)-- (302.3247,270)--(294.8247,270.5)--(287.3247,271)--(286.8247,278.5)--(286.3247,286)--(278.8247,286.5)-- (271.3247,287)--(270.8247,294.5)--(270.3247,302)--(262.8247,302.5)--(255.3247,303)--(254.8247,310.5)-- (254.3247,318)--(151.0447,318.2550)..controls(68.9357,318.4570)and (47.6737,318.2520)..(47.3187,317.2550)--cycle(253.8247,294.70).. controls(253.8247,290.7440)and(254.3267,287.3980)..(255.0247,286.70).. controls(255.7227,286.0020)and(259.0687,285.5)..(263.0247,285.5)-- (269.8247,285.5)--(269.8247,278.70)..controls(269.8247,274.7440)and (270.3267,271.3980)..(271.0247,270.70)..controls(271.7227,270.0020)and (275.0687,269.5)..(279.0247,269.5)--(285.8247,269.5)-- (285.8247,254.70)..controls(285.8247,244.5220)and(286.1997,239.5250).. (287.0247,238.70)..controls(287.7227,238.0020)and(291.0717,237.5).. (295.0367,237.5)--(301.8487,237.5)--(301.5867,198.7500)-- (301.3247,160)--(294.8247,160)--(288.3247,160)-- (288.0547,182.4210)..controls(287.8807,196.9060)and(287.3947,205.3120).. (286.6827,206.1710)..controls(285.9607,207.0410)and(283.2017,207.5).. (278.7027,207.5)--(271.8247,207.5)--(271.8247,214.30)..controls (271.8247,218.2560)and(271.3227,221.6020)..(270.6247,222.30)..controls (269.9267,222.9980)and(266.5807,223.5)..(262.6247,223.5)-- (255.8247,223.5)--(255.8247,230.3780)..controls(255.8247,234.9380)and (255.3687,237.6340)..(254.4727,238.3780)..controls(252.4847,240.0280)and (176.6787,239.9540)..(175.0247,238.30)..controls(174.3337,237.6090)and (173.8247,234.30)..(173.8247,230.5)..controls(173.8247,226.70)and (174.3337,223.3910)..(175.0247,222.70)..controls(175.9167,221.8080)and (186.1807,221.5)..(215.0247,221.5)--(253.8247,221.5)-- (253.8247,214.5)--(253.8247,207.5)--(247.0747,207.4850)..controls (243.3627,207.4770)and(239.7747,207.1200)..(239.1027,206.6930)..controls (238.2507,206.1520)and(237.7967,201.3880)..(237.6027,190.9580)-- (237.3247,176)--(229.8247,175.5)--(222.3247,175)--(222.3247,166.5)--(222.3247,158)--(253.8247,157.5)-- (285.3247,157)--(285.3247,134.5)--(285.3247,112)--(277.8247,111.5)--(270.3247,111)--(269.8247,87.5)-- (269.3247,64)--(261.8247,63.5)--(254.3247,63)-- (254.0497,47.2500)--(253.7737,31.5)--(246.9987,31.5)..controls (243.0637,31.5)and(239.7217,30.9970)..(239.0247,30.30)..controls (238.3267,29.6020)and(237.8247,26.2560)..(237.8247,22.30)-- (237.8247,15.5)--(230.8247,15.5)--(223.8247,15.5)-- (223.8247,22.30)..controls(223.8247,26.2560)and(223.3227,29.6020).. (222.6247,30.30)..controls(221.9277,30.9970)and(218.5897,31.5).. (214.6667,31.5)--(207.9087,31.5)--(207.6167,39.2500)-- (207.3247,47)--(166.8247,47)--(126.3247,47)-- (126.0327,39.2500)--(125.7407,31.5)--(118.9827,31.5)..controls (115.0597,31.5)and(111.7217,30.9970)..(111.0247,30.30)..controls (110.3267,29.6020)and(109.8247,26.2560)..(109.8247,22.30)-- (109.8247,15.5)--(94.8247,15.5)--(79.8247,15.5)-- (79.8247,30.30)..controls(79.8247,40.4780)and(79.4497,45.4750).. (78.6247,46.30)..controls(77.9277,46.9970)and(74.5847,47.5).. (70.6427,47.5)--(63.8607,47.5)--(63.5927,71.2500)-- (63.3247,95)--(55.8247,95.5)--(48.3247,96)--(47.8247,111.5)--(47.3247,127)--(39.8247,127.5)-- (32.3247,128)--(31.8247,143.5)--(31.3247,159)--(23.8247,159.5)--(16.3247,160)--(16.3247,175)-- (16.3247,190)--(23.3247,190.5)--(30.3247,191)--(30.8247,198.5)--(31.3247,206)--(38.8247,206.5)-- (46.3247,207)--(46.8247,214.5)--(47.3247,222)--(62.8247,222.5)--(78.3247,223)--(78.8247,238.5)-- (79.3247,254)--(86.8247,254.5)--(94.3247,255)--(94.8247,270.5)--(95.3247,286)--(102.8247,286.5)-- (110.3247,287)--(110.6187,294.2500)--(110.9127,301.5)-- (182.3687,301.5)--(253.8247,301.5)--cycle(126.2657,158.2080).. controls(125.9927,157.4970)and(125.8947,149.9600)..(126.0467,141.4580)-- (126.3247,126)--(142.5747,125.7250)--(158.8247,125.4500)-- (158.8247,142.4750)--(158.8247,159.5)--(142.7937,159.5)..controls (130.6137,159.5)and(126.6437,159.1900)..(126.2667,158.2080)-- cycle(222.2657,142.2080)..controls(221.9927,141.4970)and (221.8947,133.9600)..(222.0467,125.4580)--(222.3247,110)-- (238.5747,109.7250)--(254.8247,109.4500)--(254.8247,126.4750)-- (254.8247,143.5)--(238.7937,143.5)..controls(226.6137,143.5)and (222.6437,143.1900)..(222.2667,142.2080)--cycle; \path[fill=c040302](46.3247,303)--(79.0747,303.0080)..controls (95.9877,303.0120)and(142.2457,303.1250)..(181.8707,303.2580)-- (253.9167,303.5)--(253.6207,310.2500)--(253.3247,317)-- (151.0447,317.2550)..controls(65.3137,317.4680)and(48.6807,317.2880).. (48.2407,316.1430)..controls(46.9629,307.2110)and(46.3247,303).. (46.3247,303)--cycle(32.2567,300.1840)..controls(31.9597,299.4100) and(31.8537,296.1260)..(32.0207,292.8880)--(32.3247,287)-- (39.3247,287)--(46.3247,287)--(46.3247,294)-- (46.3247,301)--(39.5607,301.2960)..controls(34.4617,301.5190)and (32.6637,301.2460)..(32.2567,300.1840)--cycle(255.8247,294.5460)-- (255.8247,287.5)--(262.8707,287.5)--(269.9167,287.5)-- (269.6207,294.2500)--(269.3247,301)--(262.5747,301.2960)-- (255.8247,301.5920)--cycle(16.2817,284.2480)..controls(15.9977,283.5090) and(15.8917,276.6260)..(16.0457,268.9520)--(16.3247,255)-- (23.3247,255)--(30.3247,255)--(30.3247,270)-- (30.3247,285)--(23.5607,285.2960)..controls(18.6047,285.5130)and (16.6587,285.2330)..(16.2807,284.2480)--cycle(271.8247,278.5460)-- (271.8247,271.5)--(278.8707,271.5)--(285.9167,271.5)-- (285.6207,278.2500)--(285.3247,285)--(278.5747,285.2960)-- (271.8247,285.5920)--cycle(287.8247,254.5460)--(287.8247,239.5)-- (294.8517,239.5)--(301.8787,239.5)--(301.6017,254.2500)-- (301.3247,269)--(294.5747,269.2960)--(287.8247,269.5920)-- cycle(0.3007,252.2960)..controls(0.0267,251.5830)and(-0.0793,230.5250).. (0.0637,205.5)--(0.3247,160)--(6.9587,159.7070)-- (13.5927,159.4140)--(14.2807,190.7070)..controls(14.6597,207.9180)and (14.8247,228.9750)..(14.6467,237.5)--(14.3247,253)-- (7.5607,253.2960)..controls(2.7217,253.5080)and(0.6557,253.2230).. (0.2997,252.2960)--cycle(175.8247,230.5)--(175.8247,223.5)-- (214.8247,223.5)--(253.8247,223.5)--(253.8247,230.5)-- (253.8247,237.5)--(214.8247,237.5)--(175.8247,237.5)-- cycle(303.8247,198.5)--(303.8247,159.4090)--(310.5747,159.7050)-- (317.3247,160.0010)--(317.3247,198.5010)--(317.3247,237.0010)-- (310.5747,237.2970)--(303.8247,237.5930)--cycle(255.8247,214.70).. controls(255.8247,210.7440)and(255.3227,207.3980)..(254.6247,206.70).. controls(253.9267,206.0020)and(250.5807,205.5)..(246.6247,205.5)-- (239.8247,205.5)--(239.8247,190.70)..controls(239.8247,180.5220)and (239.4497,175.5250)..(238.6247,174.70)..controls(237.9267,174.0020)and (234.5807,173.5)..(230.6247,173.5)--(223.8247,173.5)-- (223.8247,166.5)--(223.8247,159.5)--(254.8247,159.5)-- (285.8247,159.5)--(285.8247,182.5)--(285.8247,205.5)-- (279.0247,205.5)..controls(275.0687,205.5)and(271.7227,206.0020).. (271.0247,206.70)..controls(270.3267,207.3980)and(269.8247,210.7440).. (269.8247,214.70)--(269.8247,221.5)--(262.8247,221.5)-- (255.8247,221.5)--cycle(16.0477,142.7500)--(16.3247,128)-- (23.0747,127.7040)--(29.8247,127.4090)--(29.8247,142.4540)-- (29.8247,157.5)--(22.7977,157.5)--(15.7717,157.5)-- cycle(127.8247,142.5)--(127.8247,127.5)--(142.8247,127.5)-- (157.8247,127.5)--(157.8247,142.5)--(157.8247,157.5)-- (142.8247,157.5)--(127.8247,157.5)--cycle(287.8247,134.4540)-- (287.8247,111.4080)--(294.5747,111.7040)--(301.3247,112)-- (301.5937,134.7500)--(301.8627,157.5)--(294.8437,157.5)-- (287.8247,157.5)--cycle(223.8247,126.5)--(223.8247,111.5)-- (238.8247,111.5)--(253.8247,111.5)--(253.8247,126.5)-- (253.8247,141.5)--(238.8247,141.5)--(223.8247,141.5)-- cycle(32.0477,110.7500)--(32.3247,96)--(39.0747,95.7040)-- (45.8247,95.4090)--(45.8247,110.4540)--(45.8247,125.5)-- (38.7977,125.5)--(31.7717,125.5)--cycle(271.8247,86.4540)-- (271.8247,63.4090)--(278.5747,63.7050)--(285.3247,64.0010)-- (285.5937,86.7510)--(285.8627,109.5010)--(278.8437,109.5010)-- (271.8247,109.5010)--cycle(48.0557,70.7500)--(48.3247,48)-- (55.0747,47.7040)--(61.8247,47.4090)--(61.8247,70.4540)-- (61.8247,93.5)--(54.8057,93.5)--(47.7867,93.5)-- cycle(255.8247,46.4540)--(255.8247,31.4090)--(262.5747,31.7050)-- (269.3247,32.0010)--(269.6017,46.7510)--(269.8787,61.5010)-- (262.8517,61.5010)--(255.8247,61.5010)--cycle(64.0477,30.7500)-- (64.3247,16)--(71.0747,15.7040)--(77.8247,15.4090)--(77.8247,30.4540)--(77.8247,45.5)--(70.7977,45.5)-- (63.7707,45.5)--cycle(127.8247,38.5)--(127.8247,31.5)--(166.8247,31.5)--(205.8247,31.5)--(205.8247,38.5)-- (205.8247,45.5)--(166.8247,45.5)--(127.8247,45.5)-- cycle(111.8247,22.4540)--(111.8247,15.4080)--(118.5747,15.7040)-- (125.3247,16)--(125.6207,22.7500)--(125.9167,29.5)-- (118.8707,29.5)--(111.8247,29.5)--cycle(208.0287,22.7500)-- (208.3247,16)--(215.0747,15.7040)--(221.8247,15.4090)-- (221.8247,22.4540)--(221.8247,29.5)--(214.7787,29.5)-- (207.7327,29.5)--cycle(239.8247,22.4540)--(239.8247,15.4080)-- (246.5747,15.7040)--(253.3247,16)--(253.6207,22.7500)-- (253.9167,29.5)--(246.8707,29.5)--(239.8247,29.5)--cycle(80.0287,6.7500)--(80.3247,0)--(94.8247,0)-- (109.3247,0)--(109.6207,6.7500)--(109.9167,13.5)--(94.8247,13.5)--(79.7337,13.5)--cycle(224.0287,6.7500)-- (224.3247,0)--(230.8247,0)--(237.3247,0)--(237.6207,6.7500)--(237.9167,13.5)--(230.8247,13.5)-- (223.7337,13.5)--cycle;\end{tikzpicture}}\def\bgtemplatetestcode {d0g3}\def\beginplandscape {\newgeometry{landscape,left=\pagemarginleft cm,top=\pagemargintop cm,right=\pagemarginright cm,bottom=\pagemarginbottom cm}\headwidth=\textwidth}\def\endplandscape {\restoregeometry\setpagemargincm{\pagemarginleft}{\pagemargintop}{\pagemarginright}{\pagemarginbottom}\headwidth=\textwidth}\newcommand{\checkonlyonenvimage}{\ifthenelse{\equal{\envimagesinitialized}{true}}{}{\throwwarning{Funciones \noexpand\addimage o \noexpand\addimageboxed no pueden usarse fuera del entorno \noexpand\images}\stop}}\newcommand{\checkoutsideenvimage}{\ifthenelse{\equal{\envimagesinitialized}{true}}{\throwwarning{Esta funcion solo puede usarse fuera del entorno \noexpand\images}\stop}{}}\newcommand{\alertdeprecatedcmdimage}[1]{\throwwarning{Funcion \noexpand#1 deprecada, se recomienda utilizar el entorno \noexpand\images en su lugar}}\newcommand{\coreimporttikz}{\ifthenelse{\equal{\importtikz}{false}}{\usepackage{tikz}\usetikzlibrary{babel}}{}}
% Template:     Informe/Reporte LaTeX
% Documento:    Funciones matemáticas
% Versión:      5.1.5 (15/05/2018)
% Codificación: UTF-8
%
% Autor: Pablo Pizarro R. @ppizarror
%        Facultad de Ciencias Físicas y Matemáticas
%        Universidad de Chile
%        pablo.pizarro@ing.uchile.cl, ppizarror.com
%
% Manual template: [http://latex.ppizarror.com/Template-Informe/]
% Licencia MIT:    [https://opensource.org/licenses/MIT/]

% Insertar sub-índice, a_b
% 	#1	Elemento inferior (a)
%	#2	Elemento superior (b)
\newcommand{\lpow}[2]{
	\ensuremath{{#1}_{#2}}
}

% Insertar elevado, a^b
% 	#1	Elemento inferior (a)
%	#2	Elemento superior (b)
\newcommand{\pow}[2]{
	\ensuremath{{#1}^{#2}}
}

% Inserta inverso función seno, sin^-1
%	#1	Elemento
\newcommand{\aasin}[1][]{
	\ifx\hfuzz#1\hfuzz
		\ensuremath{\sin^{-1}#1}
	\else
		\ensuremath{{\sin}^{-1}}
	\fi
}

% Inserta inverso función coseno, cos^-1
%	#1	Elemento
\newcommand{\aacos}[1][]{
	\ifx\hfuzz#1\hfuzz
		\ensuremath{\cos^{-1}#1}
	\else
		\ensuremath{\cos^{-1}}
	\fi
}

% Inserta inverso función tangente, tan^-1
%	#1	Elemento
\newcommand{\aatan}[1][]{
	\ifx\hfuzz#1\hfuzz
		\ensuremath{\tan^{-1}#1}
	\else
		\ensuremath{\tan^{-1}}
	\fi
}

% Inserta inverso función cosecante, csc^-1
%	#1	Elemento
\newcommand{\aacsc}[1][]{
	\ifx\hfuzz#1\hfuzz
		\ensuremath{\csc^{-1}#1}
	\else
		\ensuremath{\csc^{-1}}
	\fi
}

% Inserta inverso función secante, sec^-1
%	#1	Elemento
\newcommand{\aasec}[1][]{
	\ifx\hfuzz#1\hfuzz
		\ensuremath{\sec^{-1}#1}
	\else
		\ensuremath{\sec^{-1}}
	\fi
}

% Inserta inverso función cotangente, cot^-1
%	#1	Elemento
\newcommand{\aacot}[1][]{
	\ifx\hfuzz#1\hfuzz
		\ensuremath{\cot^{-1}#1}
	\else
		\ensuremath{\cot^{-1}}
	\fi
}

% Fracción de derivadas parciales af/ax
% 	#1	Función a derivar (f)
%	#2	Variable a derivar (x)
\newcommand{\fracpartial}[2]{
	\ensuremath{\pdv{#1}{#2}}
}

% Fracción de derivadas parciales dobles a^2f/ax^2
% 	#1	Función a derivar (f)
%	#2	Variable a derivar (x)
\newcommand{\fracdpartial}[2]{
	\ensuremath{\pdv[2]{#1}{#2}}
}

% Fracción de derivadas parciales en n, a^nf/ax^n
% 	#1	Función a derivar (f)
%	#2	Variable a derivar (x)
%	#3	Orden (n)
\newcommand{\fracnpartial}[3]{
	\ensuremath{\pdv[#3]{#1}{#2}}
}

% Fracción de derivadas df/dx
% 	#1	Función a derivar (f)
%	#2	Variable a derivar (x)
\newcommand{\fracderivat}[2]{
	\ensuremath{\dv{#1}{#2}}
}

% Fracción de derivadas dobles d^2/dx^2
% 	#1	Función a derivar (f)
%	#2	Variable a derivar (x)
\newcommand{\fracdderivat}[2]{
	\ensuremath{\dv[2]{#1}{#2}}
}

% Fracción de derivadas en n d^nf/dx^n
% 	#1	Función a derivar (f)
%	#2	Variable a derivar (x)
%	#3	Orden de la derivada (n)
\newcommand{\fracnderivat}[3]{
	\ensuremath{\dv[#3]{#1}{#2}}
}

% Llave superior de equivalencia
% 	#1	Elemento a igualar
%	#2	Igualdad
\newcommand{\topequal}[2]{
	\ensuremath{\overbrace{#1}^{\mathclap{#2}}}
}

% Llave inferior de equivalencia
% 	#1	Elemento a igualar
%	#2	Igualdad
\newcommand{\underequal}[2]{
	\ensuremath{\underbrace{#1}_{\mathclap{#2}}}
}

% Rectángulo superior de equivalencia
% 	#1	Elemento a igualar
%	#2	Igualdad
\newcommand{\topsequal}[2]{
	\ensuremath{\overbracket{#1}^{\mathclap{#2}}}
}

% Rectángulo inferior de equivalencia
% 	#1	Elemento a igualar
%	#2	Igualdad
\newcommand{\undersequal}[2]{
	\ensuremath{\underbracket{#1}_{\mathclap{#2}}}
}
% Template:     Informe/Reporte LaTeX
% Documento:    Funciones para insertar ecuaciones
% Versión:      5.1.5 (15/05/2018)
% Codificación: UTF-8
%
% Autor: Pablo Pizarro R. @ppizarror
%        Facultad de Ciencias Físicas y Matemáticas
%        Universidad de Chile
%        pablo.pizarro@ing.uchile.cl, ppizarror.com
%
% Manual template: [http://latex.ppizarror.com/Template-Informe/]
% Licencia MIT:    [https://opensource.org/licenses/MIT/]

% Redimensiona una ecuación en textwidth
% 	#1	Tamaño del nuevo objeto (En textwidth)
%	#2	Ecuación a redimensionar
\newcommand{\equationresize}[2]{
	\emptyvarerr{\equationresize}{#1}{Dimension no definida}
	\emptyvarerr{\equationresize}{#2}{Ecuacion a redimensionar no definida}
	\resizebox{#1\textwidth}{!}{$#2$}
}

% Insertar una ecuación
% 	#1	Label (opcional)
%	#2	Ecuación
\newcommand{\insertequation}[2][]{
	\emptyvarerr{\insertequation}{#2}{Ecuacion no definida}
	\ifthenelse{\equal{\numberedequation}{true}}{
		\vspace{-0.1cm}
		\begin{equation}
			\text{#1} #2
		\end{equation}
		\vspace{-0.26cm}
	}{
		\ifx\hfuzz#1\hfuzz
		\else
			\throwwarning{Label invalido en ecuacion sin numero}
		\fi
		\insertequationanum{#2}
	}
}

% Insertar una ecuación sin número
%	#1	Ecuación
\newcommand{\insertequationanum}[1]{
	\emptyvarerr{\insertequationanum}{#1}{Ecuacion no definida}
	\vspace{-0.1cm}
	\begin{equation*}
		\ensuremath{#1}
	\end{equation*}
	\vspace{-0.26cm}
}

% Insertar una ecuación con leyenda
% 	#1	Label (opcional)
%	#2	Ecuación
%	#3	Leyenda
\newcommand{\insertequationcaptioned}[3][]{
	\emptyvarerr{\insertequationcaptioned}{#2}{Ecuacion no definida}
	\ifx\hfuzz#3\hfuzz
		\insertequation[#1]{#2}
	\else
		\ifthenelse{\equal{\numberedequation}{true}}{
			\vspace{-0.10cm}
			\begin{equation}
				\text{#1} #2
			\end{equation}
			\vspace{-0.65cm}
			\begin{changemargin}{\captionlrmargin cm}{\captionlrmargin cm}
				\centering \textcolor{\captiontextcolor}{#3}
				\vspace{0.05cm}
			\end{changemargin}
			\vspace{0.05cm}
		}{
			\ifx\hfuzz#1\hfuzz
			\else
				\throwwarning{Label invalido en ecuacion sin numero}
			\fi
			\insertequationcaptionedanum{#2}{#3}
		}
	\fi
}

% Insertar una ecuación con leyenda sin número
%	#1	Ecuación
%	#2	Leyenda
\newcommand{\insertequationcaptionedanum}[2]{
	\emptyvarerr{\insertequationcaptionedanum}{#1}{Ecuacion no definida}
	\ifx\hfuzz#2\hfuzz
		\insertequationanum{#1}
	\else
		\vspace{-0.10cm}
		\begin{equation*}
			\ensuremath{#1}
		\end{equation*}
		\vspace{-0.65cm}
		\begin{changemargin}{\captionlrmargin cm}{\captionlrmargin cm}
			\centering \textcolor{\captiontextcolor}{#2}
			\vspace{0.05cm}
		\end{changemargin}
		\vspace{0.05cm}
	\fi
}

% Insertar una ecuación con el ambiente gather
%	#1	Ecuación
\newcommand{\insertgather}[1]{
	\emptyvarerr{\insertgather}{#1}{Ecuacion no definida}
	\ifthenelse{\equal{\numberedequation}{true}}{
		\vspace{-0.4cm}
		\begin{gather}
			\ensuremath{#1}
		\end{gather}
		\vspace{-0.4cm}
	}{
		\insertgatheranum{#1}
	}
}

% Insertar una ecuación con el ambiente gather sin número
%	#1	Ecuación
\newcommand{\insertgatheranum}[1]{
	\emptyvarerr{\insertgatheranum}{#1}{Ecuacion no definida}
	\vspace{-0.4cm}
	\begin{gather*}
		\ensuremath{#1}
	\end{gather*}
	\vspace{-0.4cm}
}

% Insertar una ecuación (gather) con leyenda
%	#1	Ecuación
%	#2	Leyenda
\newcommand{\insertgathercaptioned}[2]{
	\emptyvarerr{\insertgathercaptioned}{#1}{Ecuacion no definida}
	\ifx\hfuzz#2\hfuzz
		\insertgather{#1}
	\else
		\ifthenelse{\equal{\numberedequation}{true}}{
			\vspace{-0.45cm}
			\begin{gather}
				\ensuremath{#1}
			\end{gather}
			\vspace{-0.77cm}
			\begin{changemargin}{\captionlrmargin cm}{\captionlrmargin cm}
				\centering \textcolor{\captiontextcolor}{#2}
				\vspace{0.05cm}
			\end{changemargin}
			\vspace{0cm}
		}{
			\insertgathercaptionedanum{#1}{#2}
		}
	\fi
}

% Insertar una ecuación (gather) con leyenda sin número
%	#1	Ecuación
%	#2	Leyenda
\newcommand{\insertgathercaptionedanum}[2]{
	\emptyvarerr{\insertgathercaptionedanum}{#1}{Ecuacion no definida}
	\ifx\hfuzz#2\hfuzz
		\insertgatheranum{#1}
	\else
		\vspace{-0.45cm}
		\begin{gather*}
			\ensuremath{#1}
		\end{gather*}
		\vspace{-0.77cm}
		\begin{changemargin}{\captionlrmargin cm}{\captionlrmargin cm}
			\centering \textcolor{\captiontextcolor}{#2}
			\vspace{0.05cm}
		\end{changemargin}
		\vspace{0cm}
	\fi
}

% Insertar una ecuación con el ambiente gathered
% 	#1	Label (opcional)
%	#2	Ecuación
\newcommand{\insertgathered}[2][]{
	\emptyvarerr{\insertgathered}{#2}{Ecuacion no definida}
	\ifthenelse{\equal{\numberedequation}{true}}{
		\vspace{-0.1cm}
		\begin{equation}
			\begin{gathered}
				\text{#1} \ensuremath{#2}
			\end{gathered}
		\end{equation}
		\vspace{-0.05cm}
	}{
		\ifx\hfuzz#1\hfuzz
		\else
			\throwwarning{Label invalido en ecuacion (gathered) sin numero}
		\fi
		\insertgatheredanum{#2}
	}
}

% Insertar una ecuación con el ambiente gathered sin número
%	#1	Ecuación
\newcommand{\insertgatheredanum}[1]{
	\emptyvarerr{\insertgatheredanum}{#1}{Ecuacion no definida}
	\vspace{-0.4cm}
	\begin{gather*}
		\ensuremath{#1}
	\end{gather*}
	\vspace{-0.4cm}
}

% Insertar una ecuación (gathered) con leyenda
% 	#1	Label (opcional)
%	#2	Ecuación
%	#3	Leyenda
\newcommand{\insertgatheredcaptioned}[3][]{
	\emptyvarerr{\insertgatheredcaptioned}{#2}{Ecuacion no definida}
	\ifx\hfuzz#3\hfuzz
		\insertgathered[#1]{#2}
	\else
		\ifthenelse{\equal{\numberedequation}{true}}{
			\vspace{0cm}
			\begin{equation}
				\begin{gathered}
					\text{#1} \ensuremath{#2}
				\end{gathered}
			\end{equation}
			\vspace{-0.65cm}
			\begin{changemargin}{\captionlrmargin cm}{\captionlrmargin cm}
				\centering \textcolor{\captiontextcolor}{#3}
				\vspace{0.05cm}
			\end{changemargin}
			\vspace{0cm}
		}{
			\ifx\hfuzz#1\hfuzz
			\else
				\throwwarning{Label invalido en ecuacion (gathered) sin numero}
			\fi
			\insertgatheredcaptionedanum{#2}{#3}
		}
		\fi
}

% Insertar una ecuación (gathered) con leyenda sin número
%	#1	Ecuación
%	#2	Leyenda
\newcommand{\insertgatheredcaptionedanum}[2]{
	\emptyvarerr{\insertgatheredcaptionedanum}{#1}{Ecuacion no definida}
	\ifx\hfuzz#2\hfuzz
		\insertgatheredanum{#1}
	\else
		\vspace{-0.45cm}
		\begin{gather*}
			\ensuremath{#1}
		\end{gather*}
		\vspace{-0.7cm}
		\begin{changemargin}{\captionlrmargin cm}{\captionlrmargin cm}
			\centering \textcolor{\captiontextcolor}{#2}
			\vspace{0.05cm}
		\end{changemargin}
		\vspace{0cm}
	\fi
}

% Insertar una ecuación con el ambiente align
%	#1	Ecuación
\newcommand{\insertalign}[1]{
	\emptyvarerr{\insertalign}{#1}{Ecuacion no definida}
	\ifthenelse{\equal{\numberedequation}{true}}{
		\vspace{-0.45cm}
		\begin{align}
			\ensuremath{#1}
		\end{align}
		\vspace{-0.4cm}
	}{
		\insertalignanum{#1}
	}
}

% Insertar una ecuación con el ambiente align sin número
%	#1	Ecuación
\newcommand{\insertalignanum}[1]{
	\emptyvarerr{\insertalignanum}{#1}{Ecuacion no definida}
	\vspace{-0.45cm}
	\begin{align*}
		\ensuremath{#1}
	\end{align*}
	\vspace{-0.4cm}
}

% Insertar una ecuación (align) con leyenda
%	#1	Ecuación
%	#2	Leyenda
\newcommand{\insertaligncaptioned}[2]{
	\emptyvarerr{\insertaligncaptioned}{#1}{Ecuacion no definida}
	\ifx\hfuzz#2\hfuzz
		\insertalign{#1}
	\else
		\ifthenelse{\equal{\numberedequation}{true}}{
			\vspace{-0.45cm}
			\begin{align}
				\ensuremath{#1}
			\end{align}
			\vspace{-0.77cm}
			\begin{changemargin}{\captionlrmargin cm}{\captionlrmargin cm}
				\centering \textcolor{\captiontextcolor}{#2}
				\vspace{0.05cm}
			\end{changemargin}
			\vspace{0cm}
		}{
			\insertaligncaptionedanum{#1}{#2}
		}
	\fi
}

% Insertar una ecuación (align) con leyenda sin número
%	#1	Ecuación
%	#2	Leyenda
\newcommand{\insertaligncaptionedanum}[2]{
	\emptyvarerr{\insertaligncaptioned}{#1}{Ecuacion no definida}
	\ifx\hfuzz#2\hfuzz
		\insertalignanum{#1}
	\else
		\vspace{-0.45cm}
		\begin{align*}
			\ensuremath{#1}
		\end{align*}
		\vspace{-0.77cm}
		\begin{changemargin}{\captionlrmargin cm}{\captionlrmargin cm}
			\centering \textcolor{\captiontextcolor}{#2}
			\vspace{0.05cm}
		\end{changemargin}
		\vspace{0cm}
	\fi
}

% Insertar una ecuación con el ambiente aligned
% 	#1	Label (opcional)
%	#2	Ecuación
\newcommand{\insertaligned}[2][]{
	\emptyvarerr{\insertaligned}{#2}{Ecuacion no definida}
	\ifthenelse{\equal{\numberedequation}{true}}{
		\vspace{-0.1cm}
		\begin{equation}
			\begin{aligned}
				\text{#1} \ensuremath{#2}
			\end{aligned}
		\end{equation}
		\vspace{-0.05cm}
	}{
		\ifx\hfuzz#1\hfuzz
		\else
			\throwwarning{Label invalido en ecuacion (aligned) sin numero}
		\fi
		\insertalignedanum{#2}
	}
}

% Insertar una ecuación con el ambiente aligned sin número
%	#1	Ecuación
\newcommand{\insertalignedanum}[1]{
	\emptyvarerr{\insertalignedanum}{#1}{Ecuacion no definida}
	\vspace{-0.45cm}
	\begin{align*}
		\ensuremath{#1}
	\end{align*}
	\vspace{-0.4cm}
}

% Insertar una ecuación (aligned) con leyenda
% 	#1	Label (opcional)
%	#2	Ecuación
%	#3	Leyenda
\newcommand{\insertalignedcaptioned}[3][]{
	\emptyvarerr{\insertalignedcaptioned}{#2}{Ecuacion no definida}
	\ifx\hfuzz#3\hfuzz
		\insertaligned[#1]{#2}
	\else
		\ifthenelse{\equal{\numberedequation}{true}}{
			\vspace{0cm}
			\begin{equation}
				\begin{aligned}
					\text{#1} \ensuremath{#2}
				\end{aligned}
			\end{equation}
			\vspace{-0.65cm}
			\begin{changemargin}{\captionlrmargin cm}{\captionlrmargin cm}
				\centering \textcolor{\captiontextcolor}{#3}
				\vspace{0.05cm}
			\end{changemargin}
			\vspace{0cm}
		}{
			\ifx\hfuzz#1\hfuzz
			\else
				\throwwarning{Label invalido en ecuacion (aligned) sin numero}
			\fi
			\insertalignedcaptionedanum{#2}{#3}
		}
	\fi
}

% Insertar una ecuación (aligned) con leyenda sin número
%	#1	Ecuación
%	#2	Leyenda
\newcommand{\insertalignedcaptionedanum}[2]{
	\emptyvarerr{\insertalignedcaptioned}{#1}{Ecuacion no definida}
	\ifx\hfuzz#2\hfuzz
		\insertalignedanum{#1}
	\else
		\vspace{0cm}
		\begin{equation}
			\begin{aligned}
				\ensuremath{#1}
			\end{aligned}
		\end{equation}
		\vspace{-0.65cm}
		\begin{changemargin}{\captionlrmargin cm}{\captionlrmargin cm}
			\centering \textcolor{\captiontextcolor}{#2}
			\vspace{0.05cm}
		\end{changemargin}
		\vspace{0cm}
	\fi
}
% Template:     Informe/Reporte LaTeX
% Documento:    Funciones para insertar imágenes
% Versión:      5.1.5 (15/05/2018)
% Codificación: UTF-8
%
% Autor: Pablo Pizarro R. @ppizarror
%        Facultad de Ciencias Físicas y Matemáticas
%        Universidad de Chile
%        pablo.pizarro@ing.uchile.cl, ppizarror.com
%
% Manual template: [http://latex.ppizarror.com/Template-Informe/]
% Licencia MIT:    [https://opensource.org/licenses/MIT/]

% Añade una imagen a un env 'image' con borde
%	#1	Dirección de la imagen
%	#2	Parámetros de la imagen
%	#3	Leyenda de la imagen (opcional)
\newcommand{\addimage}[3]{
	\addimageboxed{#1}{#2}{0}{#3}
}

% Añade una imagen a un env 'image' con borde
%	#1	Dirección de la imagen
%	#2	Parámetros de la imagen
%	#3	Ancho de la línea (en pt)
%	#4	Leyenda de la imagen (opcional)
\newcommand{\addimageboxed}[4]{
	\checkonlyonenvimage
	\begingroup
		\setlength{\fboxsep}{0 pt}
		\setlength{\fboxrule}{#3 pt}
		\hspace{\marginrightmultimage cm}
		\subfloat[#4]{
			\fbox{\includegraphics[#2]{#1}}
		}
	\endgroup
}

% Insertar una imagen
% 	#1	Label (opcional)
%	#2	Dirección de la imagen
%	#3	Parámetros de la imagen
%	#4	Leyenda de la imagen (opcional)
\newcommand{\insertimage}[4][]{
	\insertimageboxed[#1]{#2}{#3}{0}{#4}
}

% Insertar una imagen con recuadro
% 	#1	Label (opcional)
%	#2	Dirección de la imagen
%	#3	Parámetros de la imagen
%	#4	Ancho de la línea (en pt)
%	#5	Leyenda de la imagen (opcional)
\newcommand{\insertimageboxed}[5][]{
	\emptyvarerr{\insertimageboxed}{#2}{Direccion de la imagen no definida}
	\emptyvarerr{\insertimageboxed}{#3}{Parametros de la imagen no definidos}
	\emptyvarerr{\insertimageboxed}{#4}{Ancho de la linea no definido}
	\checkoutsideenvimage
	\vspace{\margintopimages cm}
	\begin{figure}[H]
		\begingroup
			\setlength{\fboxsep}{0 pt}
			\setlength{\fboxrule}{#4 pt}
			\centering
			\fbox{\includegraphics[#3]{#2}}
		\endgroup
		\ifx\hfuzz#5\hfuzz
			\vspace{\captionlessmarginimage cm}
		\else
			\hspace{0cm}\caption{#5 #1}
		\fi
	\end{figure}
	\vspace{\marginbottomimages cm}
}

% Insertar una imagen doble
% [DEPRECADA]
% 	#1	Label (opcional)
%	#2	Dirección de la imagen 1
%	#3	Parámetros de la imagen 1
%	#4	Leyenda de la imagen 1
%	#5	Dirección de la imagen 2
%	#6	Parámetros de la imagen 2
%	#7	Leyenda de la imagen 2
%	#8	Leyenda de la imagen doble (opcional)
\newcommand{\insertdoubleimage}[8][]{
	\emptyvarerr{\insertdoubleimage}{#2}{Direccion de la imagen 1 no definida}
	\emptyvarerr{\insertdoubleimage}{#3}{Parametros de la imagen 1 no definidos}
	\emptyvarerr{\insertdoubleimage}{#5}{Direccion de la imagen 2 no definida}
	\emptyvarerr{\insertdoubleimage}{#6}{Parametros de la imagen 2 no definidos}
	\checkoutsideenvimage
	\alertdeprecatedcmdimage
	\vspace{\margintopimages cm}
	\captionsetup{margin=\captionmarginmultimg cm}
	\begin{figure}[H] \centering
		\subfloat[#4]{
			\includegraphics[#3]{#2}
		}
		\hspace{\marginrightmultimage cm}
		\subfloat[#7]{
			\includegraphics[#6]{#5}
		}
		\setcaptionmargincm{\captionlrmargin}
		\ifx\hfuzz#8\hfuzz
			\vspace{\captionlessmarginimage cm}
		\else
			\caption{#8 #1}
		\fi
	\end{figure}
	\setcaptionmargincm{\captionlrmargin}
	\vspace{\marginbottomimages cm}
}

% Insertar una imagen doble, igual propiedades
% [DEPRECADA]
% 	#1	Label (opcional)
%	#2	Dirección de la imagen 1
%	#3	Leyenda de la imagen 1
%	#4	Dirección de la imagen 2
%	#5	Leyenda de la imagen 2
%	#6	Propiedades de las imágenes
%	#7 	Leyenda de la imagen doble (opcional)
\newcommand{\insertdoubleeqimage}[7][]{
	\insertdoubleimage[#1]{#2}{#6}{#3}{#4}{#6}{#5}{#7}
}

% Insertar una imagen triple
% [DEPRECADA]
% 	#1	Label (opcional)
%	#2	Dirección de la imagen 1
%	#3	Parámetros de la imagen 1
%	#4	Dirección de la imagen 2
%	#5	Parámetros de la imagen 2
%	#6	Dirección de la imagen 3
%	#7	Parámetros de la imagen 3
%	#8	Leyenda de la imagen triple (opcional)
\newcommand{\inserttripleimage}[8][]{
	\emptyvarerr{\inserttripleimage}{#2}{Direccion de la imagen 1 no definida}
	\emptyvarerr{\inserttripleimage}{#3}{Parametros de la imagen 1 no definidos}
	\emptyvarerr{\inserttripleimage}{#4}{Direccion de la imagen 2 no definida}
	\emptyvarerr{\inserttripleimage}{#5}{Parametros de la imagen 2 no definidos}
	\emptyvarerr{\inserttripleimage}{#6}{Direccion de la imagen 3 no definida}
	\emptyvarerr{\inserttripleimage}{#7}{Parametros de la imagen 3 no definidos}
	\checkoutsideenvimage
	\alertdeprecatedcmdimage
	\vspace{\margintopimages cm}
	\captionsetup{margin=\captionmarginmultimg cm}
	\begin{figure}[H] \centering
		\subfloat[]{
			\includegraphics[#3]{#2}
		}
		\hspace{\marginrightmultimage cm}
		\subfloat[]{
			\includegraphics[#5]{#4}
		}
		\hspace{\marginrightmultimage cm}
		\subfloat[]{
			\includegraphics[#7]{#6}
		}
		\setcaptionmargincm{\captionlrmargin}
		\ifx\hfuzz#8\hfuzz
			\vspace{\captionlessmarginimage cm}
		\else
			\caption{#8 #1}
		\fi
	\end{figure}
	\setcaptionmargincm{\captionlrmargin}
	\vspace{\marginbottomimages cm}
}

% Insertar una imagen triple, igual propiedades
% [DEPRECADA]
% 	#1	Label (opcional)
%	#2	Dirección de la imagen 1
%	#3	Dirección de la imagen 2
%	#4	Dirección de la imagen 3
%	#5	Propiedades de las imágenes
%	#6	Leyenda de la imagen triple (opcional)
\newcommand{\inserttripleeqimage}[6][]{
	\inserttripleimage[#1]{#2}{#5}{#3}{#5}{#4}{#5}{#6}
}

% Insertar una imagen cuádruple, igual propiedades
% [DEPRECADA]
% 	#1	Label (opcional)
%	#2	Dirección de la imagen 1
%	#3	Dirección de la imagen 2
%	#4	Dirección de la imagen 3
%	#5	Dirección de la imagen 4
%	#6	Propiedades de las imágenes
%	#7	Leyenda de la imagen cuádruple (opcional)
\newcommand{\insertquadimage}[7][]{
	\emptyvarerr{\insertquadimage}{#2}{Direccion de la imagen 1 no definida}
	\emptyvarerr{\insertquadimage}{#3}{Direccion de la imagen 2 no definida}
	\emptyvarerr{\insertquadimage}{#4}{Direccion de la imagen 3 no definida}
	\emptyvarerr{\insertquadimage}{#5}{Direccion de la imagen 4 no definida}
	\emptyvarerr{\insertquadimage}{#6}{Propiedades de las imagenes no definidos}
	\checkoutsideenvimage
	\alertdeprecatedcmdimage
	\vspace{\margintopimages cm}
	\captionsetup{margin=\captionmarginmultimg cm cm}
	\begin{figure}[H] \centering
		\subfloat[]{
			\includegraphics[#6]{#2}
		}
		\hspace{\marginrightmultimage cm}
		\subfloat[]{
			\includegraphics[#6]{#3}
		}
		\hspace{\marginrightmultimage cm}
		\subfloat[]{
			\includegraphics[#6]{#4}
		}
		\hspace{\marginrightmultimage cm}
		\subfloat[]{
			\includegraphics[#6]{#5}
		}
		\setcaptionmargincm{\captionlrmargin}
		\ifx\hfuzz#7\hfuzz
			\vspace{\captionlessmarginimage cm}
		\else
			\caption{#7 #1}
		\fi
	\end{figure}
	\setcaptionmargincm{\captionlrmargin}
	\vspace{\marginbottomimages cm}
}

% Insertar una imagen quíntuple, igual propiedades
% [DEPRECADA]
% 	#1	Label (opcional)
%	#2	Dirección de la imagen 1
%	#3	Dirección de la imagen 2
%	#4	Dirección de la imagen 3
%	#5	Dirección de la imagen 4
%	#6	Dirección de la imagen 5
%	#7	Propiedades de las imágenes
%	#8	Leyenda de la imagen quíntuple (opcional)
\newcommand{\insertpentaimage}[8][]{
	\emptyvarerr{\insertpentaimage}{#2}{Direccion de la imagen 1 no definida}
	\emptyvarerr{\insertpentaimage}{#3}{Direccion de la imagen 2 no definida}
	\emptyvarerr{\insertpentaimage}{#4}{Direccion de la imagen 3 no definida}
	\emptyvarerr{\insertpentaimage}{#5}{Direccion de la imagen 4 no definida}
	\emptyvarerr{\insertpentaimage}{#6}{Direccion de la imagen 5 no definida}
	\emptyvarerr{\insertpentaimage}{#7}{Propiedades de las imagenes no definidas}
	\checkoutsideenvimage
	\alertdeprecatedcmdimage
	\vspace{\margintopimages cm}
	\captionsetup{margin=\captionmarginmultimg cm cm}
	\begin{figure}[H] \centering
		\subfloat[]{
			\includegraphics[#7]{#2}
		}
		\hspace{\marginrightmultimage cm}
		\subfloat[]{
			\includegraphics[#7]{#3}
		}
		\hspace{\marginrightmultimage cm}
		\subfloat[]{
			\includegraphics[#7]{#4}
		}
		\hspace{\marginrightmultimage cm}
		\subfloat[]{
			\includegraphics[#7]{#5}
		}
		\hspace{\marginrightmultimage cm}
		\subfloat[]{
			\includegraphics[#7]{#6}
		}
		\setcaptionmargincm{\captionlrmargin}
		\ifx\hfuzz#8\hfuzz
			\vspace{\captionlessmarginimage cm}
		\else
			\caption{#8 #1}
		\fi
	\end{figure}
	\setcaptionmargincm{\captionlrmargin}
	\vspace{\marginbottomimages cm}
}

% Insertar una imagen con 6 imágenes, igual propiedades
% [DEPRECADA]
% 	#1	Label (opcional)
%	#2	Dirección de la imagen 1
%	#3	Dirección de la imagen 2
%	#4	Dirección de la imagen 3
%	#5	Dirección de la imagen 4
%	#6	Dirección de la imagen 5
%	#7	Dirección de la imagen 6
%	#8	Propiedades de las imágenes
%	#9	Leyenda de la imagen global (opcional)
\newcommand{\inserthexaimage}[9][]{
	\emptyvarerr{\inserthexaimage}{#2}{Direccion de la imagen 1 no definida}
	\emptyvarerr{\inserthexaimage}{#3}{Direccion de la imagen 2 no definida}
	\emptyvarerr{\inserthexaimage}{#4}{Direccion de la imagen 3 no definida}
	\emptyvarerr{\inserthexaimage}{#5}{Direccion de la imagen 4 no definida}
	\emptyvarerr{\inserthexaimage}{#6}{Direccion de la imagen 5 no definida}
	\emptyvarerr{\inserthexaimage}{#7}{Direccion de la imagen 6 no definida}
	\emptyvarerr{\inserthexaimage}{#8}{Propiedades de las imagenes no definidas}
	\checkoutsideenvimage
	\alertdeprecatedcmdimage
	\vspace{\margintopimages cm}
	\captionsetup{margin=\captionmarginmultimg cm}
	\begin{figure}[H] \centering
		\subfloat[]{
			\includegraphics[#8]{#2}
		}
		\hspace{\marginrightmultimage cm}
		\subfloat[]{
			\includegraphics[#8]{#3}
		}
		\hspace{\marginrightmultimage cm}
		\subfloat[]{
			\includegraphics[#8]{#4}
		}
		\hspace{\marginrightmultimage cm}
		\subfloat[]{
			\includegraphics[#8]{#5}
		}
		\hspace{\marginrightmultimage cm}
		\subfloat[]{
			\includegraphics[#8]{#6}
		}
		\hspace{\marginrightmultimage cm}
		\subfloat[]{
			\includegraphics[#8]{#7}
		}
		\setcaptionmargincm{\captionlrmargin}
		\ifx\hfuzz#9\hfuzz
			\vspace{\captionlessmarginimage cm}
		\else
			\caption{#9 #1}
		\fi
	\end{figure}
	\setcaptionmargincm{\captionlrmargin}
	\vspace{\marginbottomimages cm}
}

% Insertar una imagen a la izquierda, escalada, ancho fijo
% 	#1	Label (opcional)
%	#2	Dirección de la imagen
%	#3	Ancho de la imagen (en textwidth)
%	#4	Leyenda de la imagen (opcional)
\newcommand{\insertimageleft}[4][]{
	\insertimageleftboxed[#1]{#2}{#3}{0}{#4}
}

% Insertar una imagen a la izquierda, escalada, ancho fijo
% 	#1	Label (opcional)
%	#2	Dirección de la imagen
%	#3	Ancho de la imagen (en textwidth)
%	#4	Ancho de la línea (en pt)
%	#5	Leyenda de la imagen (opcional)
\newcommand{\insertimageleftboxed}[5][]{
	\emptyvarerr{\insertimageleftboxed}{#2}{Direccion de la imagen no definida}
	\emptyvarerr{\insertimageleftboxed}{#3}{Ancho de la imagen no definido}
	\emptyvarerr{\insertimageleftboxed}{#4}{Ancho de la linea no definido}
	\checkoutsideenvimage
	~
	\vspace{-\baselineskip}
	\par
	\begin{wrapfigure}{l}{#3\textwidth}
		\setcaptionmargincm{0}
		\ifthenelse{\equal{\figurecaptiontop}{true}}{}{
			\vspace{\marginfloatimages pt}
		}
		\begingroup
			\setlength{\fboxsep}{0 pt}
			\setlength{\fboxrule}{#4 pt}
			\centering
			\fbox{\includegraphics[width=\linewidth]{#2}}
		\endgroup
		\ifx\hfuzz#5\hfuzz
			\vspace{\captionlessmarginimage cm}
		\else
			\caption{#5 #1}
		\fi
	\end{wrapfigure}
	\setcaptionmargincm{\captionlrmargin}
}

% Insertar una imagen a la izquierda, ajustada en un número de líneas, escalada, ancho fijo
% 	#1	Label (opcional)
%	#2	Dirección de la imagen
%	#3	Ancho de la imagen (en textwidth)
%	#4	Altura en líneas de la imagen
%	#5	Leyenda de la imagen (opcional)
\newcommand{\insertimageleftline}[5][]{
	\insertimageleftlineboxed[#1]{#2}{#3}{0}{#4}{#5}
}

% Insertar una imagen recuadrada a la izquierda, ajustada en un número de líneas, escalada, ancho fijo
% 	#1	Label (opcional)
%	#2	Dirección de la imagen
%	#3	Ancho de la imagen (en textwidth)
%	#4	Ancho de la línea (en pt)
%	#5	Altura en líneas de la imagen
%	#6	Leyenda de la imagen (opcional)
\newcommand{\insertimageleftlineboxed}[6][]{
	\emptyvarerr{\insertimageleftlineboxed}{#2}{Direccion de la imagen no definida}
	\emptyvarerr{\insertimageleftlineboxed}{#3}{Ancho de la imagen no definido}
	\emptyvarerr{\insertimageleftlineboxed}{#4}{Ancho de la linea no definido}
	\emptyvarerr{\insertimageleftlineboxed}{#5}{Altura en lineas de la imagen flotante izquierda no definida}
	\checkoutsideenvimage
	~
	\vspace{-\baselineskip}
	\par
	\begin{wrapfigure}[#5]{l}{#3\textwidth}
		\setcaptionmargincm{0}
		\ifthenelse{\equal{\figurecaptiontop}{true}}{}{
			\vspace{\marginfloatimages pt}}
		\begingroup
			\setlength{\fboxsep}{0 pt}
			\setlength{\fboxrule}{#4 pt}
			\centering
			\fbox{\includegraphics[width=\linewidth]{#2}}
		\endgroup
		\ifx\hfuzz#6\hfuzz
			\vspace{\captionlessmarginimage cm}
		\else
			\caption{#6 #1}
		\fi
	\end{wrapfigure}
	\setcaptionmargincm{\captionlrmargin}
}

% Insertar una imagen a la derecha, escalada, ancho fijo
% 	#1	Label (opcional)
%	#2	Dirección de la imagen
%	#3	Ancho de la imagen (en textwidth)
%	#4	Leyenda de la imagen (opcional)
\newcommand{\insertimageright}[4][]{
	\insertimagerightboxed[#1]{#2}{#3}{0}{#4}
}

% Insertar una imagen recuadrada a la derecha, escalada, ancho fijo
% 	#1	Label (opcional)
%	#2	Dirección de la imagen
%	#3	Ancho de la imagen (en textwidth)
%	#4	Ancho de la línea (en pt)
%	#5	Leyenda de la imagen (opcional)
\newcommand{\insertimagerightboxed}[5][]{
	\emptyvarerr{\insertimagerightboxed}{#2}{Direccion de la imagen no definida}
	\emptyvarerr{\insertimagerightboxed}{#3}{Ancho de la imagen no defindo}
	\emptyvarerr{\insertimagerightboxed}{#4}{Ancho de la linea no definido}
	\checkoutsideenvimage
	~
	\vspace{-\baselineskip}
	\par
	\begin{wrapfigure}{r}{#3\textwidth}
		\setcaptionmargincm{0}
		\ifthenelse{\equal{\figurecaptiontop}{true}}{}{
			\vspace{\marginfloatimages pt}
		}
		\begingroup
			\setlength{\fboxsep}{0 pt}
			\setlength{\fboxrule}{#4 pt}
			\centering
			\fbox{\includegraphics[width=\linewidth]{#2}}
		\endgroup
		\ifx\hfuzz#5\hfuzz
			\vspace{\captionlessmarginimage cm}
		\else
			\caption{#5 #1}
		\fi
	\end{wrapfigure}
	\setcaptionmargincm{\captionlrmargin}
}

% Insertar una imagen a la derecha, ajustada en un número de líneas, escalada, ancho fijo
% 	#1	Label (opcional)
%	#2	Dirección de la imagen
%	#3	Ancho de la imagen (en textwidth)
%	#4	Altura en líneas de la imagen
%	#5	Leyenda de la imagen (opcional)
\newcommand{\insertimagerightline}[5][]{
	\insertimagerightlineboxed[#1]{#2}{#3}{0}{#4}{#5}
}

% Insertar una imagen recuadrada a la derecha, ajustada en un número de líneas, escalada, ancho fijo
% 	#1	Label (opcional)
%	#2	Dirección de la imagen
%	#3	Ancho de la imagen (en textwidth)
%	#4	Ancho de la línea (en pt)
%	#5	Altura en líneas de la imagen
%	#6	Leyenda de la imagen (opcional)
\newcommand{\insertimagerightlineboxed}[6][]{
	\emptyvarerr{\insertimagerightlineboxed}{#2}{Direccion de la imagen no definida}
	\emptyvarerr{\insertimagerightlineboxed}{#3}{Ancho de la imagen no defindo}
	\emptyvarerr{\insertimagerightlineboxed}{#4}{Ancho de la linea no definido}
	\emptyvarerr{\insertimagerightlineboxed}{#5}{Altura en lineas de la imagen flotante derecha no definida}
	\checkoutsideenvimage
	~
	\vspace{-\baselineskip}
	\par
	\begin{wrapfigure}[#5]{r}{#3\textwidth}
		\setcaptionmargincm{0}
		\ifthenelse{\equal{\figurecaptiontop}{true}}{}{
			\vspace{\marginfloatimages pt}
		}
		\begingroup
			\setlength{\fboxsep}{0 pt}
			\setlength{\fboxrule}{#4 pt}
			\centering
			\fbox{\includegraphics[width=\linewidth]{#2}}
		\endgroup
		\ifx\hfuzz#6\hfuzz
			\vspace{\captionlessmarginimage cm}
		\else
			\caption{#6 #1}
		\fi
	\end{wrapfigure}
	\setcaptionmargincm{\captionlrmargin}
}

% Insertar una imagen a la izquierda, propiedades variables
% 	#1	Label (opcional)
%	#2	Dirección de la imagen
%	#3	Ancho del objeto
%	#4	Propiedades de la imagen
%	#5	Leyenda de la imagen (opcional)
\newcommand{\insertimageleftp}[5][]{
	\xspace~\\
	\vspace{-2\baselineskip}
	\par
	\insertimageleftboxedp[#1]{#2}{#3}{#4}{0}{#5}
}

% Insertar una imagen a la izquierda, propiedades variables
% 	#1	Label (opcional)
%	#2	Dirección de la imagen
%	#3	Ancho del objeto
%	#4	Propiedades de la imagen
%	#5	Ancho de la línea (en pt)
%	#6	Leyenda de la imagen (opcional)
\newcommand{\insertimageleftboxedp}[6][]{
	\emptyvarerr{\insertimageleftboxedp}{#2}{Direccion de la imagen no definida}
	\emptyvarerr{\insertimageleftboxedp}{#3}{Ancho del objeto no definido}
	\emptyvarerr{\insertimageleftboxedp}{#4}{Propiedades de la imagen no defindos}
	\emptyvarerr{\insertimageleftboxedp}{#5}{Ancho de la linea no definido}
	\checkoutsideenvimage
	~
	\vspace{-\baselineskip}
	\par
	\begin{wrapfigure}{l}{#3}
		\setcaptionmargincm{0}
		\ifthenelse{\equal{\figurecaptiontop}{true}}{}{
			\vspace{\marginfloatimages pt}
		}
		\begingroup
			\setlength{\fboxsep}{0 pt}
			\setlength{\fboxrule}{#5 pt}
			\centering
			\fbox{\includegraphics[#4]{#2}}
		\endgroup
		\ifx\hfuzz#6\hfuzz
			\vspace{\captionlessmarginimage cm}
		\else
			\caption{#6 #1}
		\fi
	\end{wrapfigure}
	\setcaptionmargincm{\captionlrmargin}
}

% Insertar una imagen a la izquierda, ajustada en un número de líneas, propiedades variables
% 	#1	Label (opcional)
%	#2	Dirección de la imagen
%	#3	Ancho del objeto
%	#4	Propiedades de la imagen
%	#5	Altura en líneas de la imagen
%	#6	Leyenda de la imagen (opcional)
\newcommand{\insertimageleftlinep}[6][]{
	\insertimageleftlineboxedp[#1]{#2}{#3}{#4}{0}{#5}{#6}
}

% Insertar una imagen recuadrada a la izquierda, ajustada en un número de líneas, propiedades variables
% 	#1	Label (opcional)
%	#2	Dirección de la imagen
%	#3	Ancho del objeto
%	#4	Propiedades de la imagen
%	#5	Ancho de la línea (en pt)
%	#6	Altura en líneas de la imagen
%	#7	Leyenda de la imagen (opcional)
\newcommand{\insertimageleftlineboxedp}[7][]{
	\emptyvarerr{\insertimageleftlineboxedp}{#2}{Direccion de la imagen no definida}
	\emptyvarerr{\insertimageleftlineboxedp}{#3}{Ancho del objeto no definido}
	\emptyvarerr{\insertimageleftlineboxedp}{#4}{Propiedades de la imagen no definidos}
	\emptyvarerr{\insertimageleftlineboxedp}{#5}{Ancho de la linea no definido}
	\emptyvarerr{\insertimageleftlineboxedp}{#6}{Altura en lineas de la imagen flotante izquierda no definida}
	\checkoutsideenvimage
	~
	\vspace{-\baselineskip}
	\par
	\begin{wrapfigure}[#6]{l}{#3}
		\setcaptionmargincm{0}
		\ifthenelse{\equal{\figurecaptiontop}{true}}{}{
			\vspace{\marginfloatimages pt}
		}
		\begingroup
			\setlength{\fboxsep}{0 pt}
			\setlength{\fboxrule}{#5 pt}
			\centering
			\fbox{\includegraphics[#4]{#2}}
		\endgroup
		\ifx\hfuzz#7\hfuzz
			\vspace{\captionlessmarginimage cm}
		\else
			\caption{#7 #1}
		\fi
	\end{wrapfigure}
	\setcaptionmargincm{\captionlrmargin}
}

% Insertar una imagen a la derecha, propiedades variables
% 	#1	Label (opcional)
%	#2	Dirección de la imagen
%	#3	Ancho del objeto (en cm)
%	#4	Propiedades de la imagen
%	#5	Leyenda de la imagen (opcional)
\newcommand{\insertimagerightp}[5][]{
	\xspace~\\
	\vspace{-2\baselineskip}
	\par
	\insertimagerightboxedp[#1]{#2}{#3}{#4}{0}{#5}
}

% Insertar una imagen recuadrada a la derecha, propiedades variables
% 	#1	Label (opcional)
%	#2	Dirección de la imagen
%	#3	Ancho del objeto
%	#4	Propiedades de la imagen
%	#5	Ancho de la línea (en pt)
%	#6	Leyenda de la imagen (opcional)
\newcommand{\insertimagerightboxedp}[6][]{
	\emptyvarerr{\insertimagerightboxedp}{#2}{Direccion de la imagen no definida}
	\emptyvarerr{\insertimagerightboxedp}{#3}{Ancho del objeto no definido}
	\emptyvarerr{\insertimagerightboxedp}{#4}{Propiedades de la imagen no definidos}
	\emptyvarerr{\insertimagerightboxedp}{#5}{Ancho de la linea no definido}
	\checkoutsideenvimage
	~
	\vspace{-\baselineskip}
	\par
	\begin{wrapfigure}{r}{#3}
		\setcaptionmargincm{0}
		\ifthenelse{\equal{\figurecaptiontop}{true}}{}{
			\vspace{\marginfloatimages pt}
		}
		\begingroup
			\setlength{\fboxsep}{0 pt}
			\setlength{\fboxrule}{#5 pt}
			\centering
			\fbox{\includegraphics[#4]{#2}}
		\endgroup
		\ifx\hfuzz#6\hfuzz
			\vspace{\captionlessmarginimage cm}
		\else
			\caption{#6 #1}
		\fi
	\end{wrapfigure}
	\setcaptionmargincm{\captionlrmargin}
}

% Insertar una imagen a la derecha, ajustada en un número de líneas, propiedades variables
% 	#1	Label (opcional)
%	#2	Dirección de la imagen
%	#3	Ancho del objeto (en cm)
%	#4	Propiedades de la imagen
%	#5	Altura en líneas de la imagen
%	#6	Leyenda de la imagen (opcional)
\newcommand{\insertimagerightlinep}[6][]{
	\insertimagerightlineboxedp[#1]{#2}{#3}{#4}{0}{#5}{#6}
}

% Insertar una imagen recuadrada a la derecha, ajustada en un número de líneas, propiedades variables
% 	#1	Label (opcional)
%	#2	Dirección de la imagen
%	#3	Ancho del objeto
%	#4	Propiedades de la imagen
%	#5	Ancho de la línea (en pt)
%	#6	Altura en líneas de la imagen
%	#7	Leyenda de la imagen (opcional)
\newcommand{\insertimagerightlineboxedp}[7][]{
	\emptyvarerr{\insertimagerightlineboxedp}{#2}{Direccion de la imagen no definida}
	\emptyvarerr{\insertimagerightlineboxedp}{#3}{Ancho del objeto no definido}
	\emptyvarerr{\insertimagerightlineboxedp}{#4}{Propiedades de la imagen no definidos}
	\emptyvarerr{\insertimagerightlineboxedp}{#5}{Ancho de la linea no definido}
	\emptyvarerr{\insertimagerightlineboxedp}{#6}{Altura en lineas de la imagen flotante derecha no definida}
	\checkoutsideenvimage
	~
	\vspace{-\baselineskip}
	\par
	\begin{wrapfigure}[#6]{r}{#3}
		\setcaptionmargincm{0}
		\ifthenelse{\equal{\figurecaptiontop}{true}}{}{
			\vspace{\marginfloatimages pt}
		}
		\begingroup
			\setlength{\fboxsep}{0 pt}
			\setlength{\fboxrule}{#5 pt}
			\centering
			\fbox{\includegraphics[#4]{#2}}
		\endgroup
		\ifx\hfuzz#7\hfuzz
			\vspace{\captionlessmarginimage cm}
		\else
			\caption{#7 #1}
		\fi
	\end{wrapfigure}
	\setcaptionmargincm{\captionlrmargin}
}
% Template:     Informe/Reporte LaTeX
% Documento:    Funciones para insertar títulos
% Versión:      5.4.0 (21/05/2018)
% Codificación: UTF-8
%
% Autor: Pablo Pizarro R. @ppizarror
%        Facultad de Ciencias Físicas y Matemáticas
%        Universidad de Chile
%        pablo.pizarro@ing.uchile.cl, ppizarror.com
%
% Manual template: [http://latex.ppizarror.com/Template-Informe/]
% Licencia MIT:    [https://opensource.org/licenses/MIT/]

\newcommand{\sectionanum}[1]{\emptyvarerr{\sectionanum}{#1}{Titulo no definido}\phantomsection\needspace{3\baselineskip}\section*{#1}\addcontentsline{toc}{section}{#1}\ifthenelse{\equal{\anumsecaddtocounter}{true}}{\stepcounter{section}}{}\changeheadertitle{#1}}\newcommand{\sectionanumnoi}[1]{\emptyvarerr{\sectionanumnoi}{#1}{Titulo no definido}\phantomsection\needspace{3\baselineskip}\section*{#1}\ifthenelse{\equal{\anumsecaddtocounter}{true}}{\stepcounter{section}}{}\changeheadertitle{#1}}\newcommand{\sectionanumheadless}[1]{\emptyvarerr{\sectionanumnoheadless}{#1}{Titulo no definido}\section*{#1}\addcontentsline{toc}{section}{#1}\ifthenelse{\equal{\anumsecaddtocounter}{true}}{\stepcounter{section}}{}}\newcommand{\sectionanumnoiheadless}[1]{\emptyvarerr{\sectionanumnoi}{#1}{Titulo no definido}\section*{#1}\ifthenelse{\equal{\anumsecaddtocounter}{true}}{\stepcounter{section}}{}}\newcommand{\subsectionanum}[1]{\emptyvarerr{\subsectionanum}{#1}{Subtitulo no definido}\subsection*{#1}\addcontentsline{toc}{subsection}{#1}\ifthenelse{\equal{\anumsecaddtocounter}{true}}{\stepcounter{subsection}}{}}\newcommand{\subsectionanumnoi}[1]{\emptyvarerr{\subsectionanumnoi}{#1}{Subtitulo no definido}\subsection*{#1}\ifthenelse{\equal{\anumsecaddtocounter}{true}}{\stepcounter{subsection}}{}}\newcommand{\subsubsectionanum}[1]{\emptyvarerr{\subsubsectionanum}{#1}{Sub-subtitulo no definido}\subsubsection*{#1}\addcontentsline{toc}{subsubsection}{#1}\ifthenelse{\equal{\anumsecaddtocounter}{true}}{\stepcounter{subsubsection}}{}}\newcommand{\subsubsectionanumnoi}[1]{\emptyvarerr{\subsubsectionanumnoi}{#1}{Sub-subtitulo no definido}\subsubsection*{#1}\ifthenelse{\equal{\anumsecaddtocounter}{true}}{\stepcounter{subsubsection}}{}}\newcommand{\changeheadertitle}[1]{\emptyvarerr{\changeheadertitle}{#1}{Titulo no definido}\markboth{#1}{}}\newcommand{\insertindextitle}[2]{\emptyvarerr{\insertindextitle}{#1}{Titulo no definido}\ifx\hfuzz#2\hfuzz\addtocontents{toc}{\protect\addvspace{\indextitlemargin pt}}\else\addtocontents{toc}{\protect\addvspace{#2 pt}}\fi\addtocontents{toc}{\noindent\hyperref[swpn]{\textbf{#1}}}}\newcommand{\newchapter}[1]{\emptyvarerr{\newchapter}{#1}{Titulo no definido}\newpage\stepcounter{section}\phantomsection\needspace{3\baselineskip}\vspace* {3cm}\noindent {\huge{\textbf{\nomchapter\ \thesection}}} \\\vspace* {0.5cm} \\\noindent {\Huge{\textbf{#1}}} \\\vspace {0.5cm} \\\addcontentsline{toc}{section}{\protect\numberline{\thesection}#1}\markboth{#1}{}}
% Template:     Informe/Reporte LaTeX
% Documento:    Otros estilos
% Versión:      5.4.0 (21/05/2018)
% Codificación: UTF-8
%
% Autor: Pablo Pizarro R. @ppizarror
%        Facultad de Ciencias Físicas y Matemáticas
%        Universidad de Chile
%        pablo.pizarro@ing.uchile.cl, ppizarror.com
%
% Manual template: [http://latex.ppizarror.com/Template-Informe/]
% Licencia MIT:    [https://opensource.org/licenses/MIT/]

\RequirePackage{enumitem}\makeatletter\def\greek#1{\expandafter\@greek\csname c@#1\endcsname}\def\Greek#1{\expandafter\@Greek\csname c@#1\endcsname}\def\@greek#1{\ifcase#1\or $\alpha$\or $\beta$\or $\gamma$\or $\delta$\or $\epsilon$\or $\zeta$\or $\eta$\or $\theta$\or $\iota$\or $\kappa$\or $\lambda$\or $\mu$\or $\nu$\or $\xi$\or $o$\or $\pi$\or $\rho$\or $\sigma$\or $\tau$\or $\upsilon$\or $\phi$\or $\chi$\or $\psi$\or $\omega$\fi}\def\@Greek#1{\ifcase#1\or $\mathrm{A}$\or $\mathrm{B}$\or $\Gamma$\or $\Delta$\or $\mathrm{E}$\or $\mathrm{Z}$\or $\mathrm{H}$\or $\Theta$\or $\mathrm{I}$\or $\mathrm{K}$\or $\Lambda$\or $\mathrm{M}$\or $\mathrm{N}$\or $\Xi$\or $\mathrm{O}$\or $\Pi$\or $\mathrm{P}$\or $\Sigma$\or $\mathrm{T}$\or $\mathrm{Y}$\or $\Phi$\or $\mathrm{X}$\or $\Psi$\or $\Omega$\fi}\makeatother\AddEnumerateCounter{\greek}{\@greek}{24}\AddEnumerateCounter{\Greek}{\@Greek}{12}\newcolumntype{P}[1]{>{\centering\arraybackslash}p{#1}}
% Template:     Informe/Reporte LaTeX
% Documento:    Definición de entornos
% Versión:      5.4.0 (21/05/2018)
% Codificación: UTF-8
%
% Autor: Pablo Pizarro R. @ppizarror
%        Facultad de Ciencias Físicas y Matemáticas
%        Universidad de Chile
%        pablo.pizarro@ing.uchile.cl, ppizarror.com
%
% Manual template: [http://latex.ppizarror.com/Template-Informe/]
% Licencia MIT:    [https://opensource.org/licenses/MIT/]

\newenvironment{references}{\ifthenelse{\equal{\stylecitereferences}{bibtex}}{}{\throwerror{\references}{Solo se puede usar entorno references con estilo citas \noexpand\stylecitereferences=bibtex}}\begingroup\ifthenelse{\equal{\donumrefsection}{true}}{\section{\namereferences}}{\sectionanum{\namereferences}}\renewcommand{\section}[2]{}\begin{thebibliography}{99}}{\end{thebibliography}\endgroup}\newenvironment{anexo}{\begingroup\clearpage\phantomsection\ifthenelse{\equal{\showappendixsectitle}{true}}{\appendixpage}{}\appendixtitleon\appendicestocpagenum\appendixtitletocon\bookmarksetup{numbered,openlevel=0}\begin{appendices}\bookmarksetupnext{level=part}\ifthenelse{\equal{\showappendixsecindex}{true}}{}{\belowpdfbookmark{\nameappendixsection}{contents}}\setcounter{secnumdepth}{3}\setcounter{tocdepth}{3}\ifthenelse{\equal{\appendixindepobjnum}{true}}{\counterwithin{equation}{section}\counterwithin{figure}{section}\counterwithin{lstlisting}{section}\counterwithin{table}{section}}{}}{\end{appendices}\bookmarksetupnext{level=0}\endgroup}\lstnewenvironment{sourcecodep}[4][]{\emptyvarerr{sourcecodep}{#2}{Estilo no definido}\checkvalidsourcecodestyle{#2}\ifthenelse{\equal{\showlinenumbers}{true}}{\rightlinenumbers}{}\lstset{backgroundcolor=\color{\sourcecodebgcolor}}\ifthenelse{\equal{\codecaptiontop}{true}}{\ifx\hfuzz#4\hfuzz\ifx\hfuzz#3\hfuzz\lstset{style=#2}\else\lstset{style=#2,#3}\fi\else\ifx\hfuzz#3\hfuzz\lstset{caption={#4 #1},captionpos=t,style=#2}\else\lstset{caption={#4 #1},captionpos=t,style=#2,#3}\fi\fi}{\ifx\hfuzz#4\hfuzz\ifx\hfuzz#3\hfuzz\lstset{style=#2}\else\lstset{style=#2,#3}\fi\else\ifx\hfuzz#3\hfuzz\lstset{caption={#4 #1},captionpos=b,style=#2}\else\lstset{caption={#4 #1},captionpos=b,style=#2,#3}\fi\fi}}{\ifthenelse{\equal{\showlinenumbers}{true}}{\leftlinenumbers}{}}\lstnewenvironment{sourcecode}[3][]{\emptyvarerr{\equationresize}{#2}{Estilo no definido}\checkvalidsourcecodestyle{#2}\ifthenelse{\equal{\showlinenumbers}{true}}{\rightlinenumbers}{}\lstset{backgroundcolor=\color{\sourcecodebgcolor}}\ifthenelse{\equal{\codecaptiontop}{true}}{\ifx\hfuzz#3\hfuzz\lstset{style=#2}\else\lstset{style=#2,caption={#3 #1},captionpos=t}\fi}{\ifx\hfuzz#3\hfuzz\lstset{style=#2}\else\lstset{style=#2,caption={#3 #1},captionpos=b}\fi}}{\ifthenelse{\equal{\showlinenumbers}{true}}{\leftlinenumbers}{}}\newenvironment{itemizebf}[1][]{\begin{itemize}[font=\bfseries,#1]}{\end{itemize}}\newenvironment{enumeratebf}[1][]{\begin{enumerate}[font=\bfseries,#1]}{\end{enumerate}}\newenvironment{resumen}{\sectionfont{\color{\titlecolor} \fontsizetitle \styletitle \selectfont}\sectionanumnoiheadless{\nameabstract}}{\ifthenelse{\equal{\addemptypagetwosides}{true}}{\checkoddpage\ifoddpage\newpage\null\thispagestyle{empty}\newpage\addtocounter{page}{-1}\else\fi}{}}\newenvironment{images}[2][]{\def\envimageslabelvar {#1}\def\envimagescaptionvar {#2}\def\envimagesinitialized {true}\vspace{\margintopimages cm}\captionsetup{margin=\captionmarginmultimg cm}\begin{figure}[H] \centering\vspace{-\marginrightmultimage cm}\vspace{-\marginrightmultimage cm}\vspace{-\marginrightmultimage cm}}{\setcaptionmargincm{\captionlrmargin}\ifx\hfuzz\envimagescaptionvar\hfuzz\vspace{\captionlessmarginimage cm}\else\caption{\envimagescaptionvar\envimageslabelvar}\fi\end{figure}\setcaptionmargincm{\captionlrmargin}\vspace{\marginbottomimages cm}\def\envimagesinitialized {false}}\def\envimagesinitialized {false}



% IMPORTACIÓN DE ESTILOS
% Template:     Informe/Reporte LaTeX
% Documento:    Carga las funciones del template
% Versión:      5.4.0 (21/05/2018)
% Codificación: UTF-8
%
% Autor: Pablo Pizarro R. @ppizarror
%        Facultad de Ciencias Físicas y Matemáticas
%        Universidad de Chile
%        pablo.pizarro@ing.uchile.cl, ppizarror.com
%
% Manual template: [http://latex.ppizarror.com/Template-Informe/]
% Licencia MIT:    [https://opensource.org/licenses/MIT/]

% Template:     Informe/Reporte LaTeX
% Documento:    Funciones del núcleo del template
% Versión:      5.4.0 (21/05/2018)
% Codificación: UTF-8
%
% Autor: Pablo Pizarro R. @ppizarror
%        Facultad de Ciencias Físicas y Matemáticas
%        Universidad de Chile
%        pablo.pizarro@ing.uchile.cl, ppizarror.com
%
% Manual template: [http://latex.ppizarror.com/Template-Informe/]
% Licencia MIT:    [https://opensource.org/licenses/MIT/]

\newcommand{\throwerror}[2]{\errmessage{LaTeX Error: \noexpand#1 #2 (linea \the\inputlineno)}\stop}\newcommand{\throwwarning}[1]{\errmessage{LaTeX Warning: #1 (linea \the\inputlineno)}}\newcommand{\throwbadconfigondoc}[3]{\errmessage{#1 \noexpand #2=#2. Valores esperados: #3}\stop}\newcommand{\checkvardefined}[1]{\ifthenelse{\isundefined{#1}}{\errmessage{LaTeX Warning: Variable \noexpand#1 no definida}\stop}{}}\newcommand{\checkextravarexist}[2]{\ifthenelse{\isundefined{#1}}{\errmessage{LaTeX Warning: Variable \noexpand#1 no definida}\ifx\hfuzz#2\hfuzz\errmessage{LaTeX Warning: Defina la variable en el bloque de INFORMACION DEL DOCUMENTO al comienzo del archivo principal del Template}\else\errmessage{LaTeX Warning: #2}\fi}{}}\newcommand{\emptyvarerr}[3]{\ifx\hfuzz#2\hfuzz\errmessage{LaTeX Warning: \noexpand#1 #3 (linea \the\inputlineno)}\fi}\newcommand{\setcaptionmargincm}[1]{\captionsetup{margin=#1cm}}\newcommand{\setpagemargincm}[4]{\newgeometry{left=#1cm, top=#2cm, right=#3cm, bottom=#4cm}}\newcommand{\changemargin}[2]{\emptyvarerr{\changemargin}{#1}{Margen izquierdo no definido}\emptyvarerr{\changemargin}{#2}{Margen derecho no definido}\list{}{\rightmargin#2\leftmargin#1}\item[]}\let\endchangemargin=\endlist\newcommand{\bgtemplatetestimg}{\definecolor{c040302}{RGB}{4, 3, 2}\definecolor{cb26929}{RGB}{178,105, 41}\definecolor{cfa943c}{RGB}{250, 148, 60}\definecolor{cfdc594}{RGB}{253, 197, 148}\begin{tikzpicture}[y=1.0pt, x=1.0pt, yscale=-1, xscale=1, inner sep=0pt, outer sep=0pt, opacity=0.1]\path[fill=cfdc594](46.3187,318.2550)..controls(46.0737,317.5650)and(45.7487,314.0750)..(45.5987,310.5)--(45.3247,304)--(37.8247,303.5)--(30.3247,303)--(29.8247,295.5)-- (29.3247,288)--(21.8247,287.5)--(14.3247,287)--(13.8247,271.5)--(13.3247,256)--(5.8247,255.5)-- (-1.6753,255)--(-1.6753,206.5)--(-1.6753,158)--(5.8247,157.5)--(13.3247,157)--(13.8247,141.5)-- (14.3247,126)--(21.8247,125.5)--(29.3247,125)--(29.8247,109.5)--(30.3247,94)--(37.8247,93.5)-- (45.3247,93)--(45.8247,69.5)--(46.3247,46)--(53.8247,45.5)--(61.3247,45)--(61.8247,29.5)-- (62.3247,14)--(69.8247,13.5)--(77.3247,13)--(77.8247,5.5)--(78.3247,-2)--(94.8247,-2)-- (111.3247,-2)--(111.8247,5.5)--(112.3247,13)--(119.8247,13.5)--(127.3247,14)--(127.8247,21.5)-- (128.3247,29)--(166.8247,29)--(205.3247,29)--(205.8247,21.5)--(206.3247,14)--(213.8247,13.5)-- (221.3247,13)--(221.8247,5.5)--(222.3247,-2)--(230.8247,-2)--(239.3247,-2)--(239.8247,5.5)-- (240.3247,13)--(247.8247,13.5)--(255.3247,14)--(255.8247,21.5)--(256.3247,29)--(263.8247,29.5)-- (271.3247,30)--(271.8247,45.5)--(272.3247,61)--(279.8247,61.5)--(287.3247,62)--(287.8247,85.5)-- (288.3247,109)--(295.8247,109.5)--(303.3247,110)--(303.8247,133.5)--(304.3247,157)--(311.8247,157.5)-- (319.3247,158)--(319.3247,198.5)--(319.3247,239)--(311.8247,239.5)--(304.3247,240)--(303.8247,255.5)-- (303.3247,271)--(295.8247,271.5)--(288.3247,272)--(287.8247,279.5)--(287.3247,287)--(279.8247,287.5)-- (272.3247,288)--(271.8247,295.5)--(271.3247,303)--(263.8247,303.5)--(256.3247,304)--(255.8247,311.5)-- (255.3247,319)--(151.0447,319.2550)..controls(68.1357,319.4570)and (46.6737,319.2530)..(46.3187,318.2550)--cycle(173.8247,142.5)-- (173.8247,127.5)--(166.8247,127.5)--(159.8247,127.5)-- (159.8247,142.5)--(159.8247,157.5)--(166.8247,157.5)-- (173.8247,157.5)--cycle(269.8247,126.5)--(269.8247,111.5)-- (262.8247,111.5)--(255.8247,111.5)--(255.8247,126.5)-- (255.8247,141.5)--(262.8247,141.5)--(269.8247,141.5)--cycle; \path[fill=cfa943c](47.3187,317.2550)..controls(47.0737,316.5650)and (46.7487,313.0750)..(46.5987,309.5)--(46.3247,303)--(38.8247,302.5)--(31.3247,302)--(30.8247,294.5)-- (30.3247,287)--(22.8247,286.5)--(15.3247,286)--(14.8247,270.5)--(14.3247,255)--(6.8247,254.5)-- (-0.6753,254)--(-0.6753,206.5)--(-0.6753,159)--(6.8247,158.5)--(14.3247,158)--(14.8247,142.5)-- (15.3247,127)--(22.8247,126.5)--(30.3247,126)--(30.8247,110.5)--(31.3247,95)--(38.8247,94.5)-- (46.3247,94)--(46.8247,70.5)--(47.3247,47)--(54.8247,46.5)--(62.3247,46)--(62.8247,30.5)-- (63.3247,15)--(70.8247,14.5)--(78.3247,14)--(78.8247,6.5)--(79.3247,-1)--(94.8247,-1)-- (110.3247,-1)--(110.8247,6.5)--(111.3247,14)--(118.8247,14.5)--(126.3247,15)--(126.8247,22.5)-- (127.3247,30)--(166.8247,30)--(206.3247,30)--(206.8247,22.5)--(207.3247,15)--(214.8247,14.5)-- (222.3247,14)--(222.8247,6.5)--(223.3247,-1)--(230.8247,-1)--(238.3247,-1)--(238.8247,6.5)-- (239.3247,14)--(246.8247,14.5)--(254.3247,15)--(254.8247,22.5)--(255.3247,30)--(262.8247,30.5)-- (270.3247,31)--(270.8247,46.5)--(271.3247,62)--(278.8247,62.5)--(286.3247,63)--(286.8247,86.5)-- (287.3247,110)--(294.8247,110.5)--(302.3247,111)--(302.8247,134.5)--(303.3247,158)--(310.8247,158.5)-- (318.3247,159)--(318.3247,198.5)--(318.3247,238)--(310.8247,238.5)--(303.3247,239)--(302.8247,254.5)-- (302.3247,270)--(294.8247,270.5)--(287.3247,271)--(286.8247,278.5)--(286.3247,286)--(278.8247,286.5)-- (271.3247,287)--(270.8247,294.5)--(270.3247,302)--(262.8247,302.5)--(255.3247,303)--(254.8247,310.5)-- (254.3247,318)--(151.0447,318.2550)..controls(68.9357,318.4570)and (47.6737,318.2520)..(47.3187,317.2550)--cycle(253.8247,294.70).. controls(253.8247,290.7440)and(254.3267,287.3980)..(255.0247,286.70).. controls(255.7227,286.0020)and(259.0687,285.5)..(263.0247,285.5)-- (269.8247,285.5)--(269.8247,278.70)..controls(269.8247,274.7440)and (270.3267,271.3980)..(271.0247,270.70)..controls(271.7227,270.0020)and (275.0687,269.5)..(279.0247,269.5)--(285.8247,269.5)-- (285.8247,254.70)..controls(285.8247,244.5220)and(286.1997,239.5250).. (287.0247,238.70)..controls(287.7227,238.0020)and(291.0687,237.5).. (295.0247,237.5)--(301.8247,237.5)--(301.8247,206)-- (301.8247,174.5)--(294.8247,174.5)--(287.8247,174.5)-- (287.8247,189.80)..controls(287.8247,200.3670)and(287.4537,205.4710).. (286.6247,206.30)..controls(285.9267,206.9980)and(282.5807,207.5).. (278.6247,207.5)--(271.8247,207.5)--(271.8247,214.30)..controls (271.8247,218.2560)and(271.3227,221.6020)..(270.6247,222.30)..controls (269.9267,222.9980)and(266.5807,223.5)..(262.6247,223.5)-- (255.8247,223.5)--(255.8247,230.3780)..controls(255.8247,234.9380)and (255.3687,237.6340)..(254.4727,238.3780)..controls(252.4847,240.0280)and (176.6787,239.9540)..(175.0247,238.30)..controls(174.3337,237.6090)and (173.8247,234.30)..(173.8247,230.5)..controls(173.8247,226.70)and (174.3337,223.3910)..(175.0247,222.70)..controls(175.9167,221.8080)and (186.1807,221.5)..(215.0247,221.5)--(253.8247,221.5)-- (253.8247,214.5)--(253.8247,207.5)--(249.0747,207.3720)..controls (246.4627,207.3020)and(235.9997,206.9640)..(225.8247,206.6220)-- (207.3247,206)--(206.8247,198.5)--(206.3247,191)--(190.8247,190.5)--(175.3247,190)--(174.8247,182.5)-- (174.3247,175)--(134.8247,175)--(95.3247,175)--(94.8247,182.5)--(94.3247,190)--(86.8247,190.5)-- (79.3247,191)--(79.3247,222.5)--(79.3247,254)--(86.8247,254.5)--(94.3247,255)--(94.8247,270.5)-- (95.3247,286)--(102.8247,286.5)--(110.3247,287)-- (110.6187,294.2500)--(110.9127,301.5)--(182.3687,301.5)-- (253.8247,301.5)--cycle(174.6247,158.30)..controls (176.4747,156.4500)and(176.2867,128.1290)..(174.4147,126.5750)..controls (173.5527,125.8600)and(170.3467,125.5210)..(166.1647,125.7030)-- (159.3247,126)--(159.0477,141.4580)..controls(158.8957,149.9600)and (158.9937,157.4970)..(159.2667,158.2080)..controls(159.9187,159.9070)and (172.9397,159.9850)..(174.6257,158.30)--cycle(270.6247,142.30).. controls(272.4747,140.4500)and(272.2867,112.1290)..(270.4147,110.5750).. controls(269.5527,109.8600)and(266.3467,109.5210)..(262.1647,109.7030)-- (255.3247,110)--(255.0477,125.4580)..controls(254.8957,133.9600)and (254.9937,141.4970)..(255.2667,142.2080)..controls(255.9187,143.9070)and (268.9397,143.9850)..(270.6257,142.30)--cycle(78.8247,86.5)-- (79.3247,79)--(86.8247,78.5)--(94.3247,78)--(94.8247,70.5)--(95.3247,63)--(102.8247,62.5)-- (110.3247,62)--(110.8247,54.5)--(111.3247,47)-- (118.5747,46.7060)--(125.8247,46.4120)--(125.8247,38.9560)-- (125.8247,31.5)--(119.0247,31.5)..controls(115.0687,31.5)and (111.7227,30.9980)..(111.0247,30.30)..controls(110.3267,29.6020)and (109.8247,26.2560)..(109.8247,22.30)--(109.8247,15.5)-- (94.8247,15.5)--(79.8247,15.5)--(79.8247,30.30)..controls (79.8247,40.4780)and(79.4497,45.4750)..(78.6247,46.30)..controls (77.9267,46.9980)and(74.5807,47.5)..(70.6247,47.5)-- (63.8247,47.5)--(63.8247,71.0440)--(63.8247,94.5880)-- (71.0747,94.2940)--(78.3247,94)--cycle(253.8247,39)-- (253.8247,31.5)--(247.0247,31.5)..controls(243.0687,31.5)and (239.7227,30.9980)..(239.0247,30.30)..controls(238.3267,29.6020)and (237.8247,26.2560)..(237.8247,22.30)--(237.8247,15.5)-- (230.8247,15.5)--(223.8247,15.5)--(223.8247,22.30)..controls (223.8247,26.2560)and(223.3227,29.6020)..(222.6247,30.30)..controls (221.9267,30.9980)and(218.5807,31.5)..(214.6247,31.5)-- (207.8247,31.5)--(207.8247,39)--(207.8247,46.5)-- (230.8247,46.5)--(253.8247,46.5)--cycle; \path[fill=cb26929](47.3187,317.2550)..controls(47.0737,316.5650)and (46.7487,313.0750)..(46.5987,309.5)--(46.3247,303)--(38.8247,302.5)--(31.3247,302)--(30.8247,294.5)-- (30.3247,287)--(22.8247,286.5)--(15.3247,286)--(14.8247,270.5)--(14.3247,255)--(6.8247,254.5)-- (-0.6753,254)--(-0.6753,206.5)--(-0.6753,159)--(6.8247,158.5)--(14.3247,158)--(14.8247,142.5)-- (15.3247,127)--(22.8247,126.5)--(30.3247,126)--(30.8247,110.5)--(31.3247,95)--(38.8247,94.5)-- (46.3247,94)--(46.8247,70.5)--(47.3247,47)--(54.8247,46.5)--(62.3247,46)--(62.8247,30.5)-- (63.3247,15)--(70.8247,14.5)--(78.3247,14)--(78.8247,6.5)--(79.3247,-1)--(94.8247,-1)-- (110.3247,-1)--(110.8247,6.5)--(111.3247,14)--(118.8247,14.5)--(126.3247,15)--(126.8247,22.5)-- (127.3247,30)--(166.8247,30)--(206.3247,30)--(206.8247,22.5)--(207.3247,15)--(214.8247,14.5)-- (222.3247,14)--(222.8247,6.5)--(223.3247,-1)--(230.8247,-1)--(238.3247,-1)--(238.8247,6.5)-- (239.3247,14)--(246.8247,14.5)--(254.3247,15)--(254.8247,22.5)--(255.3247,30)--(262.8247,30.5)-- (270.3247,31)--(270.8247,46.5)--(271.3247,62)--(278.8247,62.5)--(286.3247,63)--(286.8247,86.5)-- (287.3247,110)--(294.8247,110.5)--(302.3247,111)--(302.8247,134.5)--(303.3247,158)--(310.8247,158.5)-- (318.3247,159)--(318.3247,198.5)--(318.3247,238)--(310.8247,238.5)--(303.3247,239)--(302.8247,254.5)-- (302.3247,270)--(294.8247,270.5)--(287.3247,271)--(286.8247,278.5)--(286.3247,286)--(278.8247,286.5)-- (271.3247,287)--(270.8247,294.5)--(270.3247,302)--(262.8247,302.5)--(255.3247,303)--(254.8247,310.5)-- (254.3247,318)--(151.0447,318.2550)..controls(68.9357,318.4570)and (47.6737,318.2520)..(47.3187,317.2550)--cycle(253.8247,294.70).. controls(253.8247,290.7440)and(254.3267,287.3980)..(255.0247,286.70).. controls(255.7227,286.0020)and(259.0687,285.5)..(263.0247,285.5)-- (269.8247,285.5)--(269.8247,278.70)..controls(269.8247,274.7440)and (270.3267,271.3980)..(271.0247,270.70)..controls(271.7227,270.0020)and (275.0687,269.5)..(279.0247,269.5)--(285.8247,269.5)-- (285.8247,254.70)..controls(285.8247,244.5220)and(286.1997,239.5250).. (287.0247,238.70)..controls(287.7227,238.0020)and(291.0717,237.5).. (295.0367,237.5)--(301.8487,237.5)--(301.5867,198.7500)-- (301.3247,160)--(294.8247,160)--(288.3247,160)-- (288.0547,182.4210)..controls(287.8807,196.9060)and(287.3947,205.3120).. (286.6827,206.1710)..controls(285.9607,207.0410)and(283.2017,207.5).. (278.7027,207.5)--(271.8247,207.5)--(271.8247,214.30)..controls (271.8247,218.2560)and(271.3227,221.6020)..(270.6247,222.30)..controls (269.9267,222.9980)and(266.5807,223.5)..(262.6247,223.5)-- (255.8247,223.5)--(255.8247,230.3780)..controls(255.8247,234.9380)and (255.3687,237.6340)..(254.4727,238.3780)..controls(252.4847,240.0280)and (176.6787,239.9540)..(175.0247,238.30)..controls(174.3337,237.6090)and (173.8247,234.30)..(173.8247,230.5)..controls(173.8247,226.70)and (174.3337,223.3910)..(175.0247,222.70)..controls(175.9167,221.8080)and (186.1807,221.5)..(215.0247,221.5)--(253.8247,221.5)-- (253.8247,214.5)--(253.8247,207.5)--(247.0747,207.4850)..controls (243.3627,207.4770)and(239.7747,207.1200)..(239.1027,206.6930)..controls (238.2507,206.1520)and(237.7967,201.3880)..(237.6027,190.9580)-- (237.3247,176)--(229.8247,175.5)--(222.3247,175)--(222.3247,166.5)--(222.3247,158)--(253.8247,157.5)-- (285.3247,157)--(285.3247,134.5)--(285.3247,112)--(277.8247,111.5)--(270.3247,111)--(269.8247,87.5)-- (269.3247,64)--(261.8247,63.5)--(254.3247,63)-- (254.0497,47.2500)--(253.7737,31.5)--(246.9987,31.5)..controls (243.0637,31.5)and(239.7217,30.9970)..(239.0247,30.30)..controls (238.3267,29.6020)and(237.8247,26.2560)..(237.8247,22.30)-- (237.8247,15.5)--(230.8247,15.5)--(223.8247,15.5)-- (223.8247,22.30)..controls(223.8247,26.2560)and(223.3227,29.6020).. (222.6247,30.30)..controls(221.9277,30.9970)and(218.5897,31.5).. (214.6667,31.5)--(207.9087,31.5)--(207.6167,39.2500)-- (207.3247,47)--(166.8247,47)--(126.3247,47)-- (126.0327,39.2500)--(125.7407,31.5)--(118.9827,31.5)..controls (115.0597,31.5)and(111.7217,30.9970)..(111.0247,30.30)..controls (110.3267,29.6020)and(109.8247,26.2560)..(109.8247,22.30)-- (109.8247,15.5)--(94.8247,15.5)--(79.8247,15.5)-- (79.8247,30.30)..controls(79.8247,40.4780)and(79.4497,45.4750).. (78.6247,46.30)..controls(77.9277,46.9970)and(74.5847,47.5).. (70.6427,47.5)--(63.8607,47.5)--(63.5927,71.2500)-- (63.3247,95)--(55.8247,95.5)--(48.3247,96)--(47.8247,111.5)--(47.3247,127)--(39.8247,127.5)-- (32.3247,128)--(31.8247,143.5)--(31.3247,159)--(23.8247,159.5)--(16.3247,160)--(16.3247,175)-- (16.3247,190)--(23.3247,190.5)--(30.3247,191)--(30.8247,198.5)--(31.3247,206)--(38.8247,206.5)-- (46.3247,207)--(46.8247,214.5)--(47.3247,222)--(62.8247,222.5)--(78.3247,223)--(78.8247,238.5)-- (79.3247,254)--(86.8247,254.5)--(94.3247,255)--(94.8247,270.5)--(95.3247,286)--(102.8247,286.5)-- (110.3247,287)--(110.6187,294.2500)--(110.9127,301.5)-- (182.3687,301.5)--(253.8247,301.5)--cycle(126.2657,158.2080).. controls(125.9927,157.4970)and(125.8947,149.9600)..(126.0467,141.4580)-- (126.3247,126)--(142.5747,125.7250)--(158.8247,125.4500)-- (158.8247,142.4750)--(158.8247,159.5)--(142.7937,159.5)..controls (130.6137,159.5)and(126.6437,159.1900)..(126.2667,158.2080)-- cycle(222.2657,142.2080)..controls(221.9927,141.4970)and (221.8947,133.9600)..(222.0467,125.4580)--(222.3247,110)-- (238.5747,109.7250)--(254.8247,109.4500)--(254.8247,126.4750)-- (254.8247,143.5)--(238.7937,143.5)..controls(226.6137,143.5)and (222.6437,143.1900)..(222.2667,142.2080)--cycle; \path[fill=c040302](46.3247,303)--(79.0747,303.0080)..controls (95.9877,303.0120)and(142.2457,303.1250)..(181.8707,303.2580)-- (253.9167,303.5)--(253.6207,310.2500)--(253.3247,317)-- (151.0447,317.2550)..controls(65.3137,317.4680)and(48.6807,317.2880).. (48.2407,316.1430)..controls(46.9629,307.2110)and(46.3247,303).. (46.3247,303)--cycle(32.2567,300.1840)..controls(31.9597,299.4100) and(31.8537,296.1260)..(32.0207,292.8880)--(32.3247,287)-- (39.3247,287)--(46.3247,287)--(46.3247,294)-- (46.3247,301)--(39.5607,301.2960)..controls(34.4617,301.5190)and (32.6637,301.2460)..(32.2567,300.1840)--cycle(255.8247,294.5460)-- (255.8247,287.5)--(262.8707,287.5)--(269.9167,287.5)-- (269.6207,294.2500)--(269.3247,301)--(262.5747,301.2960)-- (255.8247,301.5920)--cycle(16.2817,284.2480)..controls(15.9977,283.5090) and(15.8917,276.6260)..(16.0457,268.9520)--(16.3247,255)-- (23.3247,255)--(30.3247,255)--(30.3247,270)-- (30.3247,285)--(23.5607,285.2960)..controls(18.6047,285.5130)and (16.6587,285.2330)..(16.2807,284.2480)--cycle(271.8247,278.5460)-- (271.8247,271.5)--(278.8707,271.5)--(285.9167,271.5)-- (285.6207,278.2500)--(285.3247,285)--(278.5747,285.2960)-- (271.8247,285.5920)--cycle(287.8247,254.5460)--(287.8247,239.5)-- (294.8517,239.5)--(301.8787,239.5)--(301.6017,254.2500)-- (301.3247,269)--(294.5747,269.2960)--(287.8247,269.5920)-- cycle(0.3007,252.2960)..controls(0.0267,251.5830)and(-0.0793,230.5250).. (0.0637,205.5)--(0.3247,160)--(6.9587,159.7070)-- (13.5927,159.4140)--(14.2807,190.7070)..controls(14.6597,207.9180)and (14.8247,228.9750)..(14.6467,237.5)--(14.3247,253)-- (7.5607,253.2960)..controls(2.7217,253.5080)and(0.6557,253.2230).. (0.2997,252.2960)--cycle(175.8247,230.5)--(175.8247,223.5)-- (214.8247,223.5)--(253.8247,223.5)--(253.8247,230.5)-- (253.8247,237.5)--(214.8247,237.5)--(175.8247,237.5)-- cycle(303.8247,198.5)--(303.8247,159.4090)--(310.5747,159.7050)-- (317.3247,160.0010)--(317.3247,198.5010)--(317.3247,237.0010)-- (310.5747,237.2970)--(303.8247,237.5930)--cycle(255.8247,214.70).. controls(255.8247,210.7440)and(255.3227,207.3980)..(254.6247,206.70).. controls(253.9267,206.0020)and(250.5807,205.5)..(246.6247,205.5)-- (239.8247,205.5)--(239.8247,190.70)..controls(239.8247,180.5220)and (239.4497,175.5250)..(238.6247,174.70)..controls(237.9267,174.0020)and (234.5807,173.5)..(230.6247,173.5)--(223.8247,173.5)-- (223.8247,166.5)--(223.8247,159.5)--(254.8247,159.5)-- (285.8247,159.5)--(285.8247,182.5)--(285.8247,205.5)-- (279.0247,205.5)..controls(275.0687,205.5)and(271.7227,206.0020).. (271.0247,206.70)..controls(270.3267,207.3980)and(269.8247,210.7440).. (269.8247,214.70)--(269.8247,221.5)--(262.8247,221.5)-- (255.8247,221.5)--cycle(16.0477,142.7500)--(16.3247,128)-- (23.0747,127.7040)--(29.8247,127.4090)--(29.8247,142.4540)-- (29.8247,157.5)--(22.7977,157.5)--(15.7717,157.5)-- cycle(127.8247,142.5)--(127.8247,127.5)--(142.8247,127.5)-- (157.8247,127.5)--(157.8247,142.5)--(157.8247,157.5)-- (142.8247,157.5)--(127.8247,157.5)--cycle(287.8247,134.4540)-- (287.8247,111.4080)--(294.5747,111.7040)--(301.3247,112)-- (301.5937,134.7500)--(301.8627,157.5)--(294.8437,157.5)-- (287.8247,157.5)--cycle(223.8247,126.5)--(223.8247,111.5)-- (238.8247,111.5)--(253.8247,111.5)--(253.8247,126.5)-- (253.8247,141.5)--(238.8247,141.5)--(223.8247,141.5)-- cycle(32.0477,110.7500)--(32.3247,96)--(39.0747,95.7040)-- (45.8247,95.4090)--(45.8247,110.4540)--(45.8247,125.5)-- (38.7977,125.5)--(31.7717,125.5)--cycle(271.8247,86.4540)-- (271.8247,63.4090)--(278.5747,63.7050)--(285.3247,64.0010)-- (285.5937,86.7510)--(285.8627,109.5010)--(278.8437,109.5010)-- (271.8247,109.5010)--cycle(48.0557,70.7500)--(48.3247,48)-- (55.0747,47.7040)--(61.8247,47.4090)--(61.8247,70.4540)-- (61.8247,93.5)--(54.8057,93.5)--(47.7867,93.5)-- cycle(255.8247,46.4540)--(255.8247,31.4090)--(262.5747,31.7050)-- (269.3247,32.0010)--(269.6017,46.7510)--(269.8787,61.5010)-- (262.8517,61.5010)--(255.8247,61.5010)--cycle(64.0477,30.7500)-- (64.3247,16)--(71.0747,15.7040)--(77.8247,15.4090)--(77.8247,30.4540)--(77.8247,45.5)--(70.7977,45.5)-- (63.7707,45.5)--cycle(127.8247,38.5)--(127.8247,31.5)--(166.8247,31.5)--(205.8247,31.5)--(205.8247,38.5)-- (205.8247,45.5)--(166.8247,45.5)--(127.8247,45.5)-- cycle(111.8247,22.4540)--(111.8247,15.4080)--(118.5747,15.7040)-- (125.3247,16)--(125.6207,22.7500)--(125.9167,29.5)-- (118.8707,29.5)--(111.8247,29.5)--cycle(208.0287,22.7500)-- (208.3247,16)--(215.0747,15.7040)--(221.8247,15.4090)-- (221.8247,22.4540)--(221.8247,29.5)--(214.7787,29.5)-- (207.7327,29.5)--cycle(239.8247,22.4540)--(239.8247,15.4080)-- (246.5747,15.7040)--(253.3247,16)--(253.6207,22.7500)-- (253.9167,29.5)--(246.8707,29.5)--(239.8247,29.5)--cycle(80.0287,6.7500)--(80.3247,0)--(94.8247,0)-- (109.3247,0)--(109.6207,6.7500)--(109.9167,13.5)--(94.8247,13.5)--(79.7337,13.5)--cycle(224.0287,6.7500)-- (224.3247,0)--(230.8247,0)--(237.3247,0)--(237.6207,6.7500)--(237.9167,13.5)--(230.8247,13.5)-- (223.7337,13.5)--cycle;\end{tikzpicture}}\def\bgtemplatetestcode {d0g3}\def\beginplandscape {\newgeometry{landscape,left=\pagemarginleft cm,top=\pagemargintop cm,right=\pagemarginright cm,bottom=\pagemarginbottom cm}\headwidth=\textwidth}\def\endplandscape {\restoregeometry\setpagemargincm{\pagemarginleft}{\pagemargintop}{\pagemarginright}{\pagemarginbottom}\headwidth=\textwidth}\newcommand{\checkonlyonenvimage}{\ifthenelse{\equal{\envimagesinitialized}{true}}{}{\throwwarning{Funciones \noexpand\addimage o \noexpand\addimageboxed no pueden usarse fuera del entorno \noexpand\images}\stop}}\newcommand{\checkoutsideenvimage}{\ifthenelse{\equal{\envimagesinitialized}{true}}{\throwwarning{Esta funcion solo puede usarse fuera del entorno \noexpand\images}\stop}{}}\newcommand{\alertdeprecatedcmdimage}[1]{\throwwarning{Funcion \noexpand#1 deprecada, se recomienda utilizar el entorno \noexpand\images en su lugar}}\newcommand{\coreimporttikz}{\ifthenelse{\equal{\importtikz}{false}}{\usepackage{tikz}\usetikzlibrary{babel}}{}}
% Template:     Informe/Reporte LaTeX
% Documento:    Funciones matemáticas
% Versión:      5.1.5 (15/05/2018)
% Codificación: UTF-8
%
% Autor: Pablo Pizarro R. @ppizarror
%        Facultad de Ciencias Físicas y Matemáticas
%        Universidad de Chile
%        pablo.pizarro@ing.uchile.cl, ppizarror.com
%
% Manual template: [http://latex.ppizarror.com/Template-Informe/]
% Licencia MIT:    [https://opensource.org/licenses/MIT/]

% Insertar sub-índice, a_b
% 	#1	Elemento inferior (a)
%	#2	Elemento superior (b)
\newcommand{\lpow}[2]{
	\ensuremath{{#1}_{#2}}
}

% Insertar elevado, a^b
% 	#1	Elemento inferior (a)
%	#2	Elemento superior (b)
\newcommand{\pow}[2]{
	\ensuremath{{#1}^{#2}}
}

% Inserta inverso función seno, sin^-1
%	#1	Elemento
\newcommand{\aasin}[1][]{
	\ifx\hfuzz#1\hfuzz
		\ensuremath{\sin^{-1}#1}
	\else
		\ensuremath{{\sin}^{-1}}
	\fi
}

% Inserta inverso función coseno, cos^-1
%	#1	Elemento
\newcommand{\aacos}[1][]{
	\ifx\hfuzz#1\hfuzz
		\ensuremath{\cos^{-1}#1}
	\else
		\ensuremath{\cos^{-1}}
	\fi
}

% Inserta inverso función tangente, tan^-1
%	#1	Elemento
\newcommand{\aatan}[1][]{
	\ifx\hfuzz#1\hfuzz
		\ensuremath{\tan^{-1}#1}
	\else
		\ensuremath{\tan^{-1}}
	\fi
}

% Inserta inverso función cosecante, csc^-1
%	#1	Elemento
\newcommand{\aacsc}[1][]{
	\ifx\hfuzz#1\hfuzz
		\ensuremath{\csc^{-1}#1}
	\else
		\ensuremath{\csc^{-1}}
	\fi
}

% Inserta inverso función secante, sec^-1
%	#1	Elemento
\newcommand{\aasec}[1][]{
	\ifx\hfuzz#1\hfuzz
		\ensuremath{\sec^{-1}#1}
	\else
		\ensuremath{\sec^{-1}}
	\fi
}

% Inserta inverso función cotangente, cot^-1
%	#1	Elemento
\newcommand{\aacot}[1][]{
	\ifx\hfuzz#1\hfuzz
		\ensuremath{\cot^{-1}#1}
	\else
		\ensuremath{\cot^{-1}}
	\fi
}

% Fracción de derivadas parciales af/ax
% 	#1	Función a derivar (f)
%	#2	Variable a derivar (x)
\newcommand{\fracpartial}[2]{
	\ensuremath{\pdv{#1}{#2}}
}

% Fracción de derivadas parciales dobles a^2f/ax^2
% 	#1	Función a derivar (f)
%	#2	Variable a derivar (x)
\newcommand{\fracdpartial}[2]{
	\ensuremath{\pdv[2]{#1}{#2}}
}

% Fracción de derivadas parciales en n, a^nf/ax^n
% 	#1	Función a derivar (f)
%	#2	Variable a derivar (x)
%	#3	Orden (n)
\newcommand{\fracnpartial}[3]{
	\ensuremath{\pdv[#3]{#1}{#2}}
}

% Fracción de derivadas df/dx
% 	#1	Función a derivar (f)
%	#2	Variable a derivar (x)
\newcommand{\fracderivat}[2]{
	\ensuremath{\dv{#1}{#2}}
}

% Fracción de derivadas dobles d^2/dx^2
% 	#1	Función a derivar (f)
%	#2	Variable a derivar (x)
\newcommand{\fracdderivat}[2]{
	\ensuremath{\dv[2]{#1}{#2}}
}

% Fracción de derivadas en n d^nf/dx^n
% 	#1	Función a derivar (f)
%	#2	Variable a derivar (x)
%	#3	Orden de la derivada (n)
\newcommand{\fracnderivat}[3]{
	\ensuremath{\dv[#3]{#1}{#2}}
}

% Llave superior de equivalencia
% 	#1	Elemento a igualar
%	#2	Igualdad
\newcommand{\topequal}[2]{
	\ensuremath{\overbrace{#1}^{\mathclap{#2}}}
}

% Llave inferior de equivalencia
% 	#1	Elemento a igualar
%	#2	Igualdad
\newcommand{\underequal}[2]{
	\ensuremath{\underbrace{#1}_{\mathclap{#2}}}
}

% Rectángulo superior de equivalencia
% 	#1	Elemento a igualar
%	#2	Igualdad
\newcommand{\topsequal}[2]{
	\ensuremath{\overbracket{#1}^{\mathclap{#2}}}
}

% Rectángulo inferior de equivalencia
% 	#1	Elemento a igualar
%	#2	Igualdad
\newcommand{\undersequal}[2]{
	\ensuremath{\underbracket{#1}_{\mathclap{#2}}}
}
% Template:     Informe/Reporte LaTeX
% Documento:    Funciones para insertar ecuaciones
% Versión:      5.1.5 (15/05/2018)
% Codificación: UTF-8
%
% Autor: Pablo Pizarro R. @ppizarror
%        Facultad de Ciencias Físicas y Matemáticas
%        Universidad de Chile
%        pablo.pizarro@ing.uchile.cl, ppizarror.com
%
% Manual template: [http://latex.ppizarror.com/Template-Informe/]
% Licencia MIT:    [https://opensource.org/licenses/MIT/]

% Redimensiona una ecuación en textwidth
% 	#1	Tamaño del nuevo objeto (En textwidth)
%	#2	Ecuación a redimensionar
\newcommand{\equationresize}[2]{
	\emptyvarerr{\equationresize}{#1}{Dimension no definida}
	\emptyvarerr{\equationresize}{#2}{Ecuacion a redimensionar no definida}
	\resizebox{#1\textwidth}{!}{$#2$}
}

% Insertar una ecuación
% 	#1	Label (opcional)
%	#2	Ecuación
\newcommand{\insertequation}[2][]{
	\emptyvarerr{\insertequation}{#2}{Ecuacion no definida}
	\ifthenelse{\equal{\numberedequation}{true}}{
		\vspace{-0.1cm}
		\begin{equation}
			\text{#1} #2
		\end{equation}
		\vspace{-0.26cm}
	}{
		\ifx\hfuzz#1\hfuzz
		\else
			\throwwarning{Label invalido en ecuacion sin numero}
		\fi
		\insertequationanum{#2}
	}
}

% Insertar una ecuación sin número
%	#1	Ecuación
\newcommand{\insertequationanum}[1]{
	\emptyvarerr{\insertequationanum}{#1}{Ecuacion no definida}
	\vspace{-0.1cm}
	\begin{equation*}
		\ensuremath{#1}
	\end{equation*}
	\vspace{-0.26cm}
}

% Insertar una ecuación con leyenda
% 	#1	Label (opcional)
%	#2	Ecuación
%	#3	Leyenda
\newcommand{\insertequationcaptioned}[3][]{
	\emptyvarerr{\insertequationcaptioned}{#2}{Ecuacion no definida}
	\ifx\hfuzz#3\hfuzz
		\insertequation[#1]{#2}
	\else
		\ifthenelse{\equal{\numberedequation}{true}}{
			\vspace{-0.10cm}
			\begin{equation}
				\text{#1} #2
			\end{equation}
			\vspace{-0.65cm}
			\begin{changemargin}{\captionlrmargin cm}{\captionlrmargin cm}
				\centering \textcolor{\captiontextcolor}{#3}
				\vspace{0.05cm}
			\end{changemargin}
			\vspace{0.05cm}
		}{
			\ifx\hfuzz#1\hfuzz
			\else
				\throwwarning{Label invalido en ecuacion sin numero}
			\fi
			\insertequationcaptionedanum{#2}{#3}
		}
	\fi
}

% Insertar una ecuación con leyenda sin número
%	#1	Ecuación
%	#2	Leyenda
\newcommand{\insertequationcaptionedanum}[2]{
	\emptyvarerr{\insertequationcaptionedanum}{#1}{Ecuacion no definida}
	\ifx\hfuzz#2\hfuzz
		\insertequationanum{#1}
	\else
		\vspace{-0.10cm}
		\begin{equation*}
			\ensuremath{#1}
		\end{equation*}
		\vspace{-0.65cm}
		\begin{changemargin}{\captionlrmargin cm}{\captionlrmargin cm}
			\centering \textcolor{\captiontextcolor}{#2}
			\vspace{0.05cm}
		\end{changemargin}
		\vspace{0.05cm}
	\fi
}

% Insertar una ecuación con el ambiente gather
%	#1	Ecuación
\newcommand{\insertgather}[1]{
	\emptyvarerr{\insertgather}{#1}{Ecuacion no definida}
	\ifthenelse{\equal{\numberedequation}{true}}{
		\vspace{-0.4cm}
		\begin{gather}
			\ensuremath{#1}
		\end{gather}
		\vspace{-0.4cm}
	}{
		\insertgatheranum{#1}
	}
}

% Insertar una ecuación con el ambiente gather sin número
%	#1	Ecuación
\newcommand{\insertgatheranum}[1]{
	\emptyvarerr{\insertgatheranum}{#1}{Ecuacion no definida}
	\vspace{-0.4cm}
	\begin{gather*}
		\ensuremath{#1}
	\end{gather*}
	\vspace{-0.4cm}
}

% Insertar una ecuación (gather) con leyenda
%	#1	Ecuación
%	#2	Leyenda
\newcommand{\insertgathercaptioned}[2]{
	\emptyvarerr{\insertgathercaptioned}{#1}{Ecuacion no definida}
	\ifx\hfuzz#2\hfuzz
		\insertgather{#1}
	\else
		\ifthenelse{\equal{\numberedequation}{true}}{
			\vspace{-0.45cm}
			\begin{gather}
				\ensuremath{#1}
			\end{gather}
			\vspace{-0.77cm}
			\begin{changemargin}{\captionlrmargin cm}{\captionlrmargin cm}
				\centering \textcolor{\captiontextcolor}{#2}
				\vspace{0.05cm}
			\end{changemargin}
			\vspace{0cm}
		}{
			\insertgathercaptionedanum{#1}{#2}
		}
	\fi
}

% Insertar una ecuación (gather) con leyenda sin número
%	#1	Ecuación
%	#2	Leyenda
\newcommand{\insertgathercaptionedanum}[2]{
	\emptyvarerr{\insertgathercaptionedanum}{#1}{Ecuacion no definida}
	\ifx\hfuzz#2\hfuzz
		\insertgatheranum{#1}
	\else
		\vspace{-0.45cm}
		\begin{gather*}
			\ensuremath{#1}
		\end{gather*}
		\vspace{-0.77cm}
		\begin{changemargin}{\captionlrmargin cm}{\captionlrmargin cm}
			\centering \textcolor{\captiontextcolor}{#2}
			\vspace{0.05cm}
		\end{changemargin}
		\vspace{0cm}
	\fi
}

% Insertar una ecuación con el ambiente gathered
% 	#1	Label (opcional)
%	#2	Ecuación
\newcommand{\insertgathered}[2][]{
	\emptyvarerr{\insertgathered}{#2}{Ecuacion no definida}
	\ifthenelse{\equal{\numberedequation}{true}}{
		\vspace{-0.1cm}
		\begin{equation}
			\begin{gathered}
				\text{#1} \ensuremath{#2}
			\end{gathered}
		\end{equation}
		\vspace{-0.05cm}
	}{
		\ifx\hfuzz#1\hfuzz
		\else
			\throwwarning{Label invalido en ecuacion (gathered) sin numero}
		\fi
		\insertgatheredanum{#2}
	}
}

% Insertar una ecuación con el ambiente gathered sin número
%	#1	Ecuación
\newcommand{\insertgatheredanum}[1]{
	\emptyvarerr{\insertgatheredanum}{#1}{Ecuacion no definida}
	\vspace{-0.4cm}
	\begin{gather*}
		\ensuremath{#1}
	\end{gather*}
	\vspace{-0.4cm}
}

% Insertar una ecuación (gathered) con leyenda
% 	#1	Label (opcional)
%	#2	Ecuación
%	#3	Leyenda
\newcommand{\insertgatheredcaptioned}[3][]{
	\emptyvarerr{\insertgatheredcaptioned}{#2}{Ecuacion no definida}
	\ifx\hfuzz#3\hfuzz
		\insertgathered[#1]{#2}
	\else
		\ifthenelse{\equal{\numberedequation}{true}}{
			\vspace{0cm}
			\begin{equation}
				\begin{gathered}
					\text{#1} \ensuremath{#2}
				\end{gathered}
			\end{equation}
			\vspace{-0.65cm}
			\begin{changemargin}{\captionlrmargin cm}{\captionlrmargin cm}
				\centering \textcolor{\captiontextcolor}{#3}
				\vspace{0.05cm}
			\end{changemargin}
			\vspace{0cm}
		}{
			\ifx\hfuzz#1\hfuzz
			\else
				\throwwarning{Label invalido en ecuacion (gathered) sin numero}
			\fi
			\insertgatheredcaptionedanum{#2}{#3}
		}
		\fi
}

% Insertar una ecuación (gathered) con leyenda sin número
%	#1	Ecuación
%	#2	Leyenda
\newcommand{\insertgatheredcaptionedanum}[2]{
	\emptyvarerr{\insertgatheredcaptionedanum}{#1}{Ecuacion no definida}
	\ifx\hfuzz#2\hfuzz
		\insertgatheredanum{#1}
	\else
		\vspace{-0.45cm}
		\begin{gather*}
			\ensuremath{#1}
		\end{gather*}
		\vspace{-0.7cm}
		\begin{changemargin}{\captionlrmargin cm}{\captionlrmargin cm}
			\centering \textcolor{\captiontextcolor}{#2}
			\vspace{0.05cm}
		\end{changemargin}
		\vspace{0cm}
	\fi
}

% Insertar una ecuación con el ambiente align
%	#1	Ecuación
\newcommand{\insertalign}[1]{
	\emptyvarerr{\insertalign}{#1}{Ecuacion no definida}
	\ifthenelse{\equal{\numberedequation}{true}}{
		\vspace{-0.45cm}
		\begin{align}
			\ensuremath{#1}
		\end{align}
		\vspace{-0.4cm}
	}{
		\insertalignanum{#1}
	}
}

% Insertar una ecuación con el ambiente align sin número
%	#1	Ecuación
\newcommand{\insertalignanum}[1]{
	\emptyvarerr{\insertalignanum}{#1}{Ecuacion no definida}
	\vspace{-0.45cm}
	\begin{align*}
		\ensuremath{#1}
	\end{align*}
	\vspace{-0.4cm}
}

% Insertar una ecuación (align) con leyenda
%	#1	Ecuación
%	#2	Leyenda
\newcommand{\insertaligncaptioned}[2]{
	\emptyvarerr{\insertaligncaptioned}{#1}{Ecuacion no definida}
	\ifx\hfuzz#2\hfuzz
		\insertalign{#1}
	\else
		\ifthenelse{\equal{\numberedequation}{true}}{
			\vspace{-0.45cm}
			\begin{align}
				\ensuremath{#1}
			\end{align}
			\vspace{-0.77cm}
			\begin{changemargin}{\captionlrmargin cm}{\captionlrmargin cm}
				\centering \textcolor{\captiontextcolor}{#2}
				\vspace{0.05cm}
			\end{changemargin}
			\vspace{0cm}
		}{
			\insertaligncaptionedanum{#1}{#2}
		}
	\fi
}

% Insertar una ecuación (align) con leyenda sin número
%	#1	Ecuación
%	#2	Leyenda
\newcommand{\insertaligncaptionedanum}[2]{
	\emptyvarerr{\insertaligncaptioned}{#1}{Ecuacion no definida}
	\ifx\hfuzz#2\hfuzz
		\insertalignanum{#1}
	\else
		\vspace{-0.45cm}
		\begin{align*}
			\ensuremath{#1}
		\end{align*}
		\vspace{-0.77cm}
		\begin{changemargin}{\captionlrmargin cm}{\captionlrmargin cm}
			\centering \textcolor{\captiontextcolor}{#2}
			\vspace{0.05cm}
		\end{changemargin}
		\vspace{0cm}
	\fi
}

% Insertar una ecuación con el ambiente aligned
% 	#1	Label (opcional)
%	#2	Ecuación
\newcommand{\insertaligned}[2][]{
	\emptyvarerr{\insertaligned}{#2}{Ecuacion no definida}
	\ifthenelse{\equal{\numberedequation}{true}}{
		\vspace{-0.1cm}
		\begin{equation}
			\begin{aligned}
				\text{#1} \ensuremath{#2}
			\end{aligned}
		\end{equation}
		\vspace{-0.05cm}
	}{
		\ifx\hfuzz#1\hfuzz
		\else
			\throwwarning{Label invalido en ecuacion (aligned) sin numero}
		\fi
		\insertalignedanum{#2}
	}
}

% Insertar una ecuación con el ambiente aligned sin número
%	#1	Ecuación
\newcommand{\insertalignedanum}[1]{
	\emptyvarerr{\insertalignedanum}{#1}{Ecuacion no definida}
	\vspace{-0.45cm}
	\begin{align*}
		\ensuremath{#1}
	\end{align*}
	\vspace{-0.4cm}
}

% Insertar una ecuación (aligned) con leyenda
% 	#1	Label (opcional)
%	#2	Ecuación
%	#3	Leyenda
\newcommand{\insertalignedcaptioned}[3][]{
	\emptyvarerr{\insertalignedcaptioned}{#2}{Ecuacion no definida}
	\ifx\hfuzz#3\hfuzz
		\insertaligned[#1]{#2}
	\else
		\ifthenelse{\equal{\numberedequation}{true}}{
			\vspace{0cm}
			\begin{equation}
				\begin{aligned}
					\text{#1} \ensuremath{#2}
				\end{aligned}
			\end{equation}
			\vspace{-0.65cm}
			\begin{changemargin}{\captionlrmargin cm}{\captionlrmargin cm}
				\centering \textcolor{\captiontextcolor}{#3}
				\vspace{0.05cm}
			\end{changemargin}
			\vspace{0cm}
		}{
			\ifx\hfuzz#1\hfuzz
			\else
				\throwwarning{Label invalido en ecuacion (aligned) sin numero}
			\fi
			\insertalignedcaptionedanum{#2}{#3}
		}
	\fi
}

% Insertar una ecuación (aligned) con leyenda sin número
%	#1	Ecuación
%	#2	Leyenda
\newcommand{\insertalignedcaptionedanum}[2]{
	\emptyvarerr{\insertalignedcaptioned}{#1}{Ecuacion no definida}
	\ifx\hfuzz#2\hfuzz
		\insertalignedanum{#1}
	\else
		\vspace{0cm}
		\begin{equation}
			\begin{aligned}
				\ensuremath{#1}
			\end{aligned}
		\end{equation}
		\vspace{-0.65cm}
		\begin{changemargin}{\captionlrmargin cm}{\captionlrmargin cm}
			\centering \textcolor{\captiontextcolor}{#2}
			\vspace{0.05cm}
		\end{changemargin}
		\vspace{0cm}
	\fi
}
% Template:     Informe/Reporte LaTeX
% Documento:    Funciones para insertar imágenes
% Versión:      5.1.5 (15/05/2018)
% Codificación: UTF-8
%
% Autor: Pablo Pizarro R. @ppizarror
%        Facultad de Ciencias Físicas y Matemáticas
%        Universidad de Chile
%        pablo.pizarro@ing.uchile.cl, ppizarror.com
%
% Manual template: [http://latex.ppizarror.com/Template-Informe/]
% Licencia MIT:    [https://opensource.org/licenses/MIT/]

% Añade una imagen a un env 'image' con borde
%	#1	Dirección de la imagen
%	#2	Parámetros de la imagen
%	#3	Leyenda de la imagen (opcional)
\newcommand{\addimage}[3]{
	\addimageboxed{#1}{#2}{0}{#3}
}

% Añade una imagen a un env 'image' con borde
%	#1	Dirección de la imagen
%	#2	Parámetros de la imagen
%	#3	Ancho de la línea (en pt)
%	#4	Leyenda de la imagen (opcional)
\newcommand{\addimageboxed}[4]{
	\checkonlyonenvimage
	\begingroup
		\setlength{\fboxsep}{0 pt}
		\setlength{\fboxrule}{#3 pt}
		\hspace{\marginrightmultimage cm}
		\subfloat[#4]{
			\fbox{\includegraphics[#2]{#1}}
		}
	\endgroup
}

% Insertar una imagen
% 	#1	Label (opcional)
%	#2	Dirección de la imagen
%	#3	Parámetros de la imagen
%	#4	Leyenda de la imagen (opcional)
\newcommand{\insertimage}[4][]{
	\insertimageboxed[#1]{#2}{#3}{0}{#4}
}

% Insertar una imagen con recuadro
% 	#1	Label (opcional)
%	#2	Dirección de la imagen
%	#3	Parámetros de la imagen
%	#4	Ancho de la línea (en pt)
%	#5	Leyenda de la imagen (opcional)
\newcommand{\insertimageboxed}[5][]{
	\emptyvarerr{\insertimageboxed}{#2}{Direccion de la imagen no definida}
	\emptyvarerr{\insertimageboxed}{#3}{Parametros de la imagen no definidos}
	\emptyvarerr{\insertimageboxed}{#4}{Ancho de la linea no definido}
	\checkoutsideenvimage
	\vspace{\margintopimages cm}
	\begin{figure}[H]
		\begingroup
			\setlength{\fboxsep}{0 pt}
			\setlength{\fboxrule}{#4 pt}
			\centering
			\fbox{\includegraphics[#3]{#2}}
		\endgroup
		\ifx\hfuzz#5\hfuzz
			\vspace{\captionlessmarginimage cm}
		\else
			\hspace{0cm}\caption{#5 #1}
		\fi
	\end{figure}
	\vspace{\marginbottomimages cm}
}

% Insertar una imagen doble
% [DEPRECADA]
% 	#1	Label (opcional)
%	#2	Dirección de la imagen 1
%	#3	Parámetros de la imagen 1
%	#4	Leyenda de la imagen 1
%	#5	Dirección de la imagen 2
%	#6	Parámetros de la imagen 2
%	#7	Leyenda de la imagen 2
%	#8	Leyenda de la imagen doble (opcional)
\newcommand{\insertdoubleimage}[8][]{
	\emptyvarerr{\insertdoubleimage}{#2}{Direccion de la imagen 1 no definida}
	\emptyvarerr{\insertdoubleimage}{#3}{Parametros de la imagen 1 no definidos}
	\emptyvarerr{\insertdoubleimage}{#5}{Direccion de la imagen 2 no definida}
	\emptyvarerr{\insertdoubleimage}{#6}{Parametros de la imagen 2 no definidos}
	\checkoutsideenvimage
	\alertdeprecatedcmdimage
	\vspace{\margintopimages cm}
	\captionsetup{margin=\captionmarginmultimg cm}
	\begin{figure}[H] \centering
		\subfloat[#4]{
			\includegraphics[#3]{#2}
		}
		\hspace{\marginrightmultimage cm}
		\subfloat[#7]{
			\includegraphics[#6]{#5}
		}
		\setcaptionmargincm{\captionlrmargin}
		\ifx\hfuzz#8\hfuzz
			\vspace{\captionlessmarginimage cm}
		\else
			\caption{#8 #1}
		\fi
	\end{figure}
	\setcaptionmargincm{\captionlrmargin}
	\vspace{\marginbottomimages cm}
}

% Insertar una imagen doble, igual propiedades
% [DEPRECADA]
% 	#1	Label (opcional)
%	#2	Dirección de la imagen 1
%	#3	Leyenda de la imagen 1
%	#4	Dirección de la imagen 2
%	#5	Leyenda de la imagen 2
%	#6	Propiedades de las imágenes
%	#7 	Leyenda de la imagen doble (opcional)
\newcommand{\insertdoubleeqimage}[7][]{
	\insertdoubleimage[#1]{#2}{#6}{#3}{#4}{#6}{#5}{#7}
}

% Insertar una imagen triple
% [DEPRECADA]
% 	#1	Label (opcional)
%	#2	Dirección de la imagen 1
%	#3	Parámetros de la imagen 1
%	#4	Dirección de la imagen 2
%	#5	Parámetros de la imagen 2
%	#6	Dirección de la imagen 3
%	#7	Parámetros de la imagen 3
%	#8	Leyenda de la imagen triple (opcional)
\newcommand{\inserttripleimage}[8][]{
	\emptyvarerr{\inserttripleimage}{#2}{Direccion de la imagen 1 no definida}
	\emptyvarerr{\inserttripleimage}{#3}{Parametros de la imagen 1 no definidos}
	\emptyvarerr{\inserttripleimage}{#4}{Direccion de la imagen 2 no definida}
	\emptyvarerr{\inserttripleimage}{#5}{Parametros de la imagen 2 no definidos}
	\emptyvarerr{\inserttripleimage}{#6}{Direccion de la imagen 3 no definida}
	\emptyvarerr{\inserttripleimage}{#7}{Parametros de la imagen 3 no definidos}
	\checkoutsideenvimage
	\alertdeprecatedcmdimage
	\vspace{\margintopimages cm}
	\captionsetup{margin=\captionmarginmultimg cm}
	\begin{figure}[H] \centering
		\subfloat[]{
			\includegraphics[#3]{#2}
		}
		\hspace{\marginrightmultimage cm}
		\subfloat[]{
			\includegraphics[#5]{#4}
		}
		\hspace{\marginrightmultimage cm}
		\subfloat[]{
			\includegraphics[#7]{#6}
		}
		\setcaptionmargincm{\captionlrmargin}
		\ifx\hfuzz#8\hfuzz
			\vspace{\captionlessmarginimage cm}
		\else
			\caption{#8 #1}
		\fi
	\end{figure}
	\setcaptionmargincm{\captionlrmargin}
	\vspace{\marginbottomimages cm}
}

% Insertar una imagen triple, igual propiedades
% [DEPRECADA]
% 	#1	Label (opcional)
%	#2	Dirección de la imagen 1
%	#3	Dirección de la imagen 2
%	#4	Dirección de la imagen 3
%	#5	Propiedades de las imágenes
%	#6	Leyenda de la imagen triple (opcional)
\newcommand{\inserttripleeqimage}[6][]{
	\inserttripleimage[#1]{#2}{#5}{#3}{#5}{#4}{#5}{#6}
}

% Insertar una imagen cuádruple, igual propiedades
% [DEPRECADA]
% 	#1	Label (opcional)
%	#2	Dirección de la imagen 1
%	#3	Dirección de la imagen 2
%	#4	Dirección de la imagen 3
%	#5	Dirección de la imagen 4
%	#6	Propiedades de las imágenes
%	#7	Leyenda de la imagen cuádruple (opcional)
\newcommand{\insertquadimage}[7][]{
	\emptyvarerr{\insertquadimage}{#2}{Direccion de la imagen 1 no definida}
	\emptyvarerr{\insertquadimage}{#3}{Direccion de la imagen 2 no definida}
	\emptyvarerr{\insertquadimage}{#4}{Direccion de la imagen 3 no definida}
	\emptyvarerr{\insertquadimage}{#5}{Direccion de la imagen 4 no definida}
	\emptyvarerr{\insertquadimage}{#6}{Propiedades de las imagenes no definidos}
	\checkoutsideenvimage
	\alertdeprecatedcmdimage
	\vspace{\margintopimages cm}
	\captionsetup{margin=\captionmarginmultimg cm cm}
	\begin{figure}[H] \centering
		\subfloat[]{
			\includegraphics[#6]{#2}
		}
		\hspace{\marginrightmultimage cm}
		\subfloat[]{
			\includegraphics[#6]{#3}
		}
		\hspace{\marginrightmultimage cm}
		\subfloat[]{
			\includegraphics[#6]{#4}
		}
		\hspace{\marginrightmultimage cm}
		\subfloat[]{
			\includegraphics[#6]{#5}
		}
		\setcaptionmargincm{\captionlrmargin}
		\ifx\hfuzz#7\hfuzz
			\vspace{\captionlessmarginimage cm}
		\else
			\caption{#7 #1}
		\fi
	\end{figure}
	\setcaptionmargincm{\captionlrmargin}
	\vspace{\marginbottomimages cm}
}

% Insertar una imagen quíntuple, igual propiedades
% [DEPRECADA]
% 	#1	Label (opcional)
%	#2	Dirección de la imagen 1
%	#3	Dirección de la imagen 2
%	#4	Dirección de la imagen 3
%	#5	Dirección de la imagen 4
%	#6	Dirección de la imagen 5
%	#7	Propiedades de las imágenes
%	#8	Leyenda de la imagen quíntuple (opcional)
\newcommand{\insertpentaimage}[8][]{
	\emptyvarerr{\insertpentaimage}{#2}{Direccion de la imagen 1 no definida}
	\emptyvarerr{\insertpentaimage}{#3}{Direccion de la imagen 2 no definida}
	\emptyvarerr{\insertpentaimage}{#4}{Direccion de la imagen 3 no definida}
	\emptyvarerr{\insertpentaimage}{#5}{Direccion de la imagen 4 no definida}
	\emptyvarerr{\insertpentaimage}{#6}{Direccion de la imagen 5 no definida}
	\emptyvarerr{\insertpentaimage}{#7}{Propiedades de las imagenes no definidas}
	\checkoutsideenvimage
	\alertdeprecatedcmdimage
	\vspace{\margintopimages cm}
	\captionsetup{margin=\captionmarginmultimg cm cm}
	\begin{figure}[H] \centering
		\subfloat[]{
			\includegraphics[#7]{#2}
		}
		\hspace{\marginrightmultimage cm}
		\subfloat[]{
			\includegraphics[#7]{#3}
		}
		\hspace{\marginrightmultimage cm}
		\subfloat[]{
			\includegraphics[#7]{#4}
		}
		\hspace{\marginrightmultimage cm}
		\subfloat[]{
			\includegraphics[#7]{#5}
		}
		\hspace{\marginrightmultimage cm}
		\subfloat[]{
			\includegraphics[#7]{#6}
		}
		\setcaptionmargincm{\captionlrmargin}
		\ifx\hfuzz#8\hfuzz
			\vspace{\captionlessmarginimage cm}
		\else
			\caption{#8 #1}
		\fi
	\end{figure}
	\setcaptionmargincm{\captionlrmargin}
	\vspace{\marginbottomimages cm}
}

% Insertar una imagen con 6 imágenes, igual propiedades
% [DEPRECADA]
% 	#1	Label (opcional)
%	#2	Dirección de la imagen 1
%	#3	Dirección de la imagen 2
%	#4	Dirección de la imagen 3
%	#5	Dirección de la imagen 4
%	#6	Dirección de la imagen 5
%	#7	Dirección de la imagen 6
%	#8	Propiedades de las imágenes
%	#9	Leyenda de la imagen global (opcional)
\newcommand{\inserthexaimage}[9][]{
	\emptyvarerr{\inserthexaimage}{#2}{Direccion de la imagen 1 no definida}
	\emptyvarerr{\inserthexaimage}{#3}{Direccion de la imagen 2 no definida}
	\emptyvarerr{\inserthexaimage}{#4}{Direccion de la imagen 3 no definida}
	\emptyvarerr{\inserthexaimage}{#5}{Direccion de la imagen 4 no definida}
	\emptyvarerr{\inserthexaimage}{#6}{Direccion de la imagen 5 no definida}
	\emptyvarerr{\inserthexaimage}{#7}{Direccion de la imagen 6 no definida}
	\emptyvarerr{\inserthexaimage}{#8}{Propiedades de las imagenes no definidas}
	\checkoutsideenvimage
	\alertdeprecatedcmdimage
	\vspace{\margintopimages cm}
	\captionsetup{margin=\captionmarginmultimg cm}
	\begin{figure}[H] \centering
		\subfloat[]{
			\includegraphics[#8]{#2}
		}
		\hspace{\marginrightmultimage cm}
		\subfloat[]{
			\includegraphics[#8]{#3}
		}
		\hspace{\marginrightmultimage cm}
		\subfloat[]{
			\includegraphics[#8]{#4}
		}
		\hspace{\marginrightmultimage cm}
		\subfloat[]{
			\includegraphics[#8]{#5}
		}
		\hspace{\marginrightmultimage cm}
		\subfloat[]{
			\includegraphics[#8]{#6}
		}
		\hspace{\marginrightmultimage cm}
		\subfloat[]{
			\includegraphics[#8]{#7}
		}
		\setcaptionmargincm{\captionlrmargin}
		\ifx\hfuzz#9\hfuzz
			\vspace{\captionlessmarginimage cm}
		\else
			\caption{#9 #1}
		\fi
	\end{figure}
	\setcaptionmargincm{\captionlrmargin}
	\vspace{\marginbottomimages cm}
}

% Insertar una imagen a la izquierda, escalada, ancho fijo
% 	#1	Label (opcional)
%	#2	Dirección de la imagen
%	#3	Ancho de la imagen (en textwidth)
%	#4	Leyenda de la imagen (opcional)
\newcommand{\insertimageleft}[4][]{
	\insertimageleftboxed[#1]{#2}{#3}{0}{#4}
}

% Insertar una imagen a la izquierda, escalada, ancho fijo
% 	#1	Label (opcional)
%	#2	Dirección de la imagen
%	#3	Ancho de la imagen (en textwidth)
%	#4	Ancho de la línea (en pt)
%	#5	Leyenda de la imagen (opcional)
\newcommand{\insertimageleftboxed}[5][]{
	\emptyvarerr{\insertimageleftboxed}{#2}{Direccion de la imagen no definida}
	\emptyvarerr{\insertimageleftboxed}{#3}{Ancho de la imagen no definido}
	\emptyvarerr{\insertimageleftboxed}{#4}{Ancho de la linea no definido}
	\checkoutsideenvimage
	~
	\vspace{-\baselineskip}
	\par
	\begin{wrapfigure}{l}{#3\textwidth}
		\setcaptionmargincm{0}
		\ifthenelse{\equal{\figurecaptiontop}{true}}{}{
			\vspace{\marginfloatimages pt}
		}
		\begingroup
			\setlength{\fboxsep}{0 pt}
			\setlength{\fboxrule}{#4 pt}
			\centering
			\fbox{\includegraphics[width=\linewidth]{#2}}
		\endgroup
		\ifx\hfuzz#5\hfuzz
			\vspace{\captionlessmarginimage cm}
		\else
			\caption{#5 #1}
		\fi
	\end{wrapfigure}
	\setcaptionmargincm{\captionlrmargin}
}

% Insertar una imagen a la izquierda, ajustada en un número de líneas, escalada, ancho fijo
% 	#1	Label (opcional)
%	#2	Dirección de la imagen
%	#3	Ancho de la imagen (en textwidth)
%	#4	Altura en líneas de la imagen
%	#5	Leyenda de la imagen (opcional)
\newcommand{\insertimageleftline}[5][]{
	\insertimageleftlineboxed[#1]{#2}{#3}{0}{#4}{#5}
}

% Insertar una imagen recuadrada a la izquierda, ajustada en un número de líneas, escalada, ancho fijo
% 	#1	Label (opcional)
%	#2	Dirección de la imagen
%	#3	Ancho de la imagen (en textwidth)
%	#4	Ancho de la línea (en pt)
%	#5	Altura en líneas de la imagen
%	#6	Leyenda de la imagen (opcional)
\newcommand{\insertimageleftlineboxed}[6][]{
	\emptyvarerr{\insertimageleftlineboxed}{#2}{Direccion de la imagen no definida}
	\emptyvarerr{\insertimageleftlineboxed}{#3}{Ancho de la imagen no definido}
	\emptyvarerr{\insertimageleftlineboxed}{#4}{Ancho de la linea no definido}
	\emptyvarerr{\insertimageleftlineboxed}{#5}{Altura en lineas de la imagen flotante izquierda no definida}
	\checkoutsideenvimage
	~
	\vspace{-\baselineskip}
	\par
	\begin{wrapfigure}[#5]{l}{#3\textwidth}
		\setcaptionmargincm{0}
		\ifthenelse{\equal{\figurecaptiontop}{true}}{}{
			\vspace{\marginfloatimages pt}}
		\begingroup
			\setlength{\fboxsep}{0 pt}
			\setlength{\fboxrule}{#4 pt}
			\centering
			\fbox{\includegraphics[width=\linewidth]{#2}}
		\endgroup
		\ifx\hfuzz#6\hfuzz
			\vspace{\captionlessmarginimage cm}
		\else
			\caption{#6 #1}
		\fi
	\end{wrapfigure}
	\setcaptionmargincm{\captionlrmargin}
}

% Insertar una imagen a la derecha, escalada, ancho fijo
% 	#1	Label (opcional)
%	#2	Dirección de la imagen
%	#3	Ancho de la imagen (en textwidth)
%	#4	Leyenda de la imagen (opcional)
\newcommand{\insertimageright}[4][]{
	\insertimagerightboxed[#1]{#2}{#3}{0}{#4}
}

% Insertar una imagen recuadrada a la derecha, escalada, ancho fijo
% 	#1	Label (opcional)
%	#2	Dirección de la imagen
%	#3	Ancho de la imagen (en textwidth)
%	#4	Ancho de la línea (en pt)
%	#5	Leyenda de la imagen (opcional)
\newcommand{\insertimagerightboxed}[5][]{
	\emptyvarerr{\insertimagerightboxed}{#2}{Direccion de la imagen no definida}
	\emptyvarerr{\insertimagerightboxed}{#3}{Ancho de la imagen no defindo}
	\emptyvarerr{\insertimagerightboxed}{#4}{Ancho de la linea no definido}
	\checkoutsideenvimage
	~
	\vspace{-\baselineskip}
	\par
	\begin{wrapfigure}{r}{#3\textwidth}
		\setcaptionmargincm{0}
		\ifthenelse{\equal{\figurecaptiontop}{true}}{}{
			\vspace{\marginfloatimages pt}
		}
		\begingroup
			\setlength{\fboxsep}{0 pt}
			\setlength{\fboxrule}{#4 pt}
			\centering
			\fbox{\includegraphics[width=\linewidth]{#2}}
		\endgroup
		\ifx\hfuzz#5\hfuzz
			\vspace{\captionlessmarginimage cm}
		\else
			\caption{#5 #1}
		\fi
	\end{wrapfigure}
	\setcaptionmargincm{\captionlrmargin}
}

% Insertar una imagen a la derecha, ajustada en un número de líneas, escalada, ancho fijo
% 	#1	Label (opcional)
%	#2	Dirección de la imagen
%	#3	Ancho de la imagen (en textwidth)
%	#4	Altura en líneas de la imagen
%	#5	Leyenda de la imagen (opcional)
\newcommand{\insertimagerightline}[5][]{
	\insertimagerightlineboxed[#1]{#2}{#3}{0}{#4}{#5}
}

% Insertar una imagen recuadrada a la derecha, ajustada en un número de líneas, escalada, ancho fijo
% 	#1	Label (opcional)
%	#2	Dirección de la imagen
%	#3	Ancho de la imagen (en textwidth)
%	#4	Ancho de la línea (en pt)
%	#5	Altura en líneas de la imagen
%	#6	Leyenda de la imagen (opcional)
\newcommand{\insertimagerightlineboxed}[6][]{
	\emptyvarerr{\insertimagerightlineboxed}{#2}{Direccion de la imagen no definida}
	\emptyvarerr{\insertimagerightlineboxed}{#3}{Ancho de la imagen no defindo}
	\emptyvarerr{\insertimagerightlineboxed}{#4}{Ancho de la linea no definido}
	\emptyvarerr{\insertimagerightlineboxed}{#5}{Altura en lineas de la imagen flotante derecha no definida}
	\checkoutsideenvimage
	~
	\vspace{-\baselineskip}
	\par
	\begin{wrapfigure}[#5]{r}{#3\textwidth}
		\setcaptionmargincm{0}
		\ifthenelse{\equal{\figurecaptiontop}{true}}{}{
			\vspace{\marginfloatimages pt}
		}
		\begingroup
			\setlength{\fboxsep}{0 pt}
			\setlength{\fboxrule}{#4 pt}
			\centering
			\fbox{\includegraphics[width=\linewidth]{#2}}
		\endgroup
		\ifx\hfuzz#6\hfuzz
			\vspace{\captionlessmarginimage cm}
		\else
			\caption{#6 #1}
		\fi
	\end{wrapfigure}
	\setcaptionmargincm{\captionlrmargin}
}

% Insertar una imagen a la izquierda, propiedades variables
% 	#1	Label (opcional)
%	#2	Dirección de la imagen
%	#3	Ancho del objeto
%	#4	Propiedades de la imagen
%	#5	Leyenda de la imagen (opcional)
\newcommand{\insertimageleftp}[5][]{
	\xspace~\\
	\vspace{-2\baselineskip}
	\par
	\insertimageleftboxedp[#1]{#2}{#3}{#4}{0}{#5}
}

% Insertar una imagen a la izquierda, propiedades variables
% 	#1	Label (opcional)
%	#2	Dirección de la imagen
%	#3	Ancho del objeto
%	#4	Propiedades de la imagen
%	#5	Ancho de la línea (en pt)
%	#6	Leyenda de la imagen (opcional)
\newcommand{\insertimageleftboxedp}[6][]{
	\emptyvarerr{\insertimageleftboxedp}{#2}{Direccion de la imagen no definida}
	\emptyvarerr{\insertimageleftboxedp}{#3}{Ancho del objeto no definido}
	\emptyvarerr{\insertimageleftboxedp}{#4}{Propiedades de la imagen no defindos}
	\emptyvarerr{\insertimageleftboxedp}{#5}{Ancho de la linea no definido}
	\checkoutsideenvimage
	~
	\vspace{-\baselineskip}
	\par
	\begin{wrapfigure}{l}{#3}
		\setcaptionmargincm{0}
		\ifthenelse{\equal{\figurecaptiontop}{true}}{}{
			\vspace{\marginfloatimages pt}
		}
		\begingroup
			\setlength{\fboxsep}{0 pt}
			\setlength{\fboxrule}{#5 pt}
			\centering
			\fbox{\includegraphics[#4]{#2}}
		\endgroup
		\ifx\hfuzz#6\hfuzz
			\vspace{\captionlessmarginimage cm}
		\else
			\caption{#6 #1}
		\fi
	\end{wrapfigure}
	\setcaptionmargincm{\captionlrmargin}
}

% Insertar una imagen a la izquierda, ajustada en un número de líneas, propiedades variables
% 	#1	Label (opcional)
%	#2	Dirección de la imagen
%	#3	Ancho del objeto
%	#4	Propiedades de la imagen
%	#5	Altura en líneas de la imagen
%	#6	Leyenda de la imagen (opcional)
\newcommand{\insertimageleftlinep}[6][]{
	\insertimageleftlineboxedp[#1]{#2}{#3}{#4}{0}{#5}{#6}
}

% Insertar una imagen recuadrada a la izquierda, ajustada en un número de líneas, propiedades variables
% 	#1	Label (opcional)
%	#2	Dirección de la imagen
%	#3	Ancho del objeto
%	#4	Propiedades de la imagen
%	#5	Ancho de la línea (en pt)
%	#6	Altura en líneas de la imagen
%	#7	Leyenda de la imagen (opcional)
\newcommand{\insertimageleftlineboxedp}[7][]{
	\emptyvarerr{\insertimageleftlineboxedp}{#2}{Direccion de la imagen no definida}
	\emptyvarerr{\insertimageleftlineboxedp}{#3}{Ancho del objeto no definido}
	\emptyvarerr{\insertimageleftlineboxedp}{#4}{Propiedades de la imagen no definidos}
	\emptyvarerr{\insertimageleftlineboxedp}{#5}{Ancho de la linea no definido}
	\emptyvarerr{\insertimageleftlineboxedp}{#6}{Altura en lineas de la imagen flotante izquierda no definida}
	\checkoutsideenvimage
	~
	\vspace{-\baselineskip}
	\par
	\begin{wrapfigure}[#6]{l}{#3}
		\setcaptionmargincm{0}
		\ifthenelse{\equal{\figurecaptiontop}{true}}{}{
			\vspace{\marginfloatimages pt}
		}
		\begingroup
			\setlength{\fboxsep}{0 pt}
			\setlength{\fboxrule}{#5 pt}
			\centering
			\fbox{\includegraphics[#4]{#2}}
		\endgroup
		\ifx\hfuzz#7\hfuzz
			\vspace{\captionlessmarginimage cm}
		\else
			\caption{#7 #1}
		\fi
	\end{wrapfigure}
	\setcaptionmargincm{\captionlrmargin}
}

% Insertar una imagen a la derecha, propiedades variables
% 	#1	Label (opcional)
%	#2	Dirección de la imagen
%	#3	Ancho del objeto (en cm)
%	#4	Propiedades de la imagen
%	#5	Leyenda de la imagen (opcional)
\newcommand{\insertimagerightp}[5][]{
	\xspace~\\
	\vspace{-2\baselineskip}
	\par
	\insertimagerightboxedp[#1]{#2}{#3}{#4}{0}{#5}
}

% Insertar una imagen recuadrada a la derecha, propiedades variables
% 	#1	Label (opcional)
%	#2	Dirección de la imagen
%	#3	Ancho del objeto
%	#4	Propiedades de la imagen
%	#5	Ancho de la línea (en pt)
%	#6	Leyenda de la imagen (opcional)
\newcommand{\insertimagerightboxedp}[6][]{
	\emptyvarerr{\insertimagerightboxedp}{#2}{Direccion de la imagen no definida}
	\emptyvarerr{\insertimagerightboxedp}{#3}{Ancho del objeto no definido}
	\emptyvarerr{\insertimagerightboxedp}{#4}{Propiedades de la imagen no definidos}
	\emptyvarerr{\insertimagerightboxedp}{#5}{Ancho de la linea no definido}
	\checkoutsideenvimage
	~
	\vspace{-\baselineskip}
	\par
	\begin{wrapfigure}{r}{#3}
		\setcaptionmargincm{0}
		\ifthenelse{\equal{\figurecaptiontop}{true}}{}{
			\vspace{\marginfloatimages pt}
		}
		\begingroup
			\setlength{\fboxsep}{0 pt}
			\setlength{\fboxrule}{#5 pt}
			\centering
			\fbox{\includegraphics[#4]{#2}}
		\endgroup
		\ifx\hfuzz#6\hfuzz
			\vspace{\captionlessmarginimage cm}
		\else
			\caption{#6 #1}
		\fi
	\end{wrapfigure}
	\setcaptionmargincm{\captionlrmargin}
}

% Insertar una imagen a la derecha, ajustada en un número de líneas, propiedades variables
% 	#1	Label (opcional)
%	#2	Dirección de la imagen
%	#3	Ancho del objeto (en cm)
%	#4	Propiedades de la imagen
%	#5	Altura en líneas de la imagen
%	#6	Leyenda de la imagen (opcional)
\newcommand{\insertimagerightlinep}[6][]{
	\insertimagerightlineboxedp[#1]{#2}{#3}{#4}{0}{#5}{#6}
}

% Insertar una imagen recuadrada a la derecha, ajustada en un número de líneas, propiedades variables
% 	#1	Label (opcional)
%	#2	Dirección de la imagen
%	#3	Ancho del objeto
%	#4	Propiedades de la imagen
%	#5	Ancho de la línea (en pt)
%	#6	Altura en líneas de la imagen
%	#7	Leyenda de la imagen (opcional)
\newcommand{\insertimagerightlineboxedp}[7][]{
	\emptyvarerr{\insertimagerightlineboxedp}{#2}{Direccion de la imagen no definida}
	\emptyvarerr{\insertimagerightlineboxedp}{#3}{Ancho del objeto no definido}
	\emptyvarerr{\insertimagerightlineboxedp}{#4}{Propiedades de la imagen no definidos}
	\emptyvarerr{\insertimagerightlineboxedp}{#5}{Ancho de la linea no definido}
	\emptyvarerr{\insertimagerightlineboxedp}{#6}{Altura en lineas de la imagen flotante derecha no definida}
	\checkoutsideenvimage
	~
	\vspace{-\baselineskip}
	\par
	\begin{wrapfigure}[#6]{r}{#3}
		\setcaptionmargincm{0}
		\ifthenelse{\equal{\figurecaptiontop}{true}}{}{
			\vspace{\marginfloatimages pt}
		}
		\begingroup
			\setlength{\fboxsep}{0 pt}
			\setlength{\fboxrule}{#5 pt}
			\centering
			\fbox{\includegraphics[#4]{#2}}
		\endgroup
		\ifx\hfuzz#7\hfuzz
			\vspace{\captionlessmarginimage cm}
		\else
			\caption{#7 #1}
		\fi
	\end{wrapfigure}
	\setcaptionmargincm{\captionlrmargin}
}
% Template:     Informe/Reporte LaTeX
% Documento:    Funciones para insertar títulos
% Versión:      5.4.0 (21/05/2018)
% Codificación: UTF-8
%
% Autor: Pablo Pizarro R. @ppizarror
%        Facultad de Ciencias Físicas y Matemáticas
%        Universidad de Chile
%        pablo.pizarro@ing.uchile.cl, ppizarror.com
%
% Manual template: [http://latex.ppizarror.com/Template-Informe/]
% Licencia MIT:    [https://opensource.org/licenses/MIT/]

\newcommand{\sectionanum}[1]{\emptyvarerr{\sectionanum}{#1}{Titulo no definido}\phantomsection\needspace{3\baselineskip}\section*{#1}\addcontentsline{toc}{section}{#1}\ifthenelse{\equal{\anumsecaddtocounter}{true}}{\stepcounter{section}}{}\changeheadertitle{#1}}\newcommand{\sectionanumnoi}[1]{\emptyvarerr{\sectionanumnoi}{#1}{Titulo no definido}\phantomsection\needspace{3\baselineskip}\section*{#1}\ifthenelse{\equal{\anumsecaddtocounter}{true}}{\stepcounter{section}}{}\changeheadertitle{#1}}\newcommand{\sectionanumheadless}[1]{\emptyvarerr{\sectionanumnoheadless}{#1}{Titulo no definido}\section*{#1}\addcontentsline{toc}{section}{#1}\ifthenelse{\equal{\anumsecaddtocounter}{true}}{\stepcounter{section}}{}}\newcommand{\sectionanumnoiheadless}[1]{\emptyvarerr{\sectionanumnoi}{#1}{Titulo no definido}\section*{#1}\ifthenelse{\equal{\anumsecaddtocounter}{true}}{\stepcounter{section}}{}}\newcommand{\subsectionanum}[1]{\emptyvarerr{\subsectionanum}{#1}{Subtitulo no definido}\subsection*{#1}\addcontentsline{toc}{subsection}{#1}\ifthenelse{\equal{\anumsecaddtocounter}{true}}{\stepcounter{subsection}}{}}\newcommand{\subsectionanumnoi}[1]{\emptyvarerr{\subsectionanumnoi}{#1}{Subtitulo no definido}\subsection*{#1}\ifthenelse{\equal{\anumsecaddtocounter}{true}}{\stepcounter{subsection}}{}}\newcommand{\subsubsectionanum}[1]{\emptyvarerr{\subsubsectionanum}{#1}{Sub-subtitulo no definido}\subsubsection*{#1}\addcontentsline{toc}{subsubsection}{#1}\ifthenelse{\equal{\anumsecaddtocounter}{true}}{\stepcounter{subsubsection}}{}}\newcommand{\subsubsectionanumnoi}[1]{\emptyvarerr{\subsubsectionanumnoi}{#1}{Sub-subtitulo no definido}\subsubsection*{#1}\ifthenelse{\equal{\anumsecaddtocounter}{true}}{\stepcounter{subsubsection}}{}}\newcommand{\changeheadertitle}[1]{\emptyvarerr{\changeheadertitle}{#1}{Titulo no definido}\markboth{#1}{}}\newcommand{\insertindextitle}[2]{\emptyvarerr{\insertindextitle}{#1}{Titulo no definido}\ifx\hfuzz#2\hfuzz\addtocontents{toc}{\protect\addvspace{\indextitlemargin pt}}\else\addtocontents{toc}{\protect\addvspace{#2 pt}}\fi\addtocontents{toc}{\noindent\hyperref[swpn]{\textbf{#1}}}}\newcommand{\newchapter}[1]{\emptyvarerr{\newchapter}{#1}{Titulo no definido}\newpage\stepcounter{section}\phantomsection\needspace{3\baselineskip}\vspace* {3cm}\noindent {\huge{\textbf{\nomchapter\ \thesection}}} \\\vspace* {0.5cm} \\\noindent {\Huge{\textbf{#1}}} \\\vspace {0.5cm} \\\addcontentsline{toc}{section}{\protect\numberline{\thesection}#1}\markboth{#1}{}}
% Template:     Informe/Reporte LaTeX
% Documento:    Otros estilos
% Versión:      5.4.0 (21/05/2018)
% Codificación: UTF-8
%
% Autor: Pablo Pizarro R. @ppizarror
%        Facultad de Ciencias Físicas y Matemáticas
%        Universidad de Chile
%        pablo.pizarro@ing.uchile.cl, ppizarror.com
%
% Manual template: [http://latex.ppizarror.com/Template-Informe/]
% Licencia MIT:    [https://opensource.org/licenses/MIT/]

\RequirePackage{enumitem}\makeatletter\def\greek#1{\expandafter\@greek\csname c@#1\endcsname}\def\Greek#1{\expandafter\@Greek\csname c@#1\endcsname}\def\@greek#1{\ifcase#1\or $\alpha$\or $\beta$\or $\gamma$\or $\delta$\or $\epsilon$\or $\zeta$\or $\eta$\or $\theta$\or $\iota$\or $\kappa$\or $\lambda$\or $\mu$\or $\nu$\or $\xi$\or $o$\or $\pi$\or $\rho$\or $\sigma$\or $\tau$\or $\upsilon$\or $\phi$\or $\chi$\or $\psi$\or $\omega$\fi}\def\@Greek#1{\ifcase#1\or $\mathrm{A}$\or $\mathrm{B}$\or $\Gamma$\or $\Delta$\or $\mathrm{E}$\or $\mathrm{Z}$\or $\mathrm{H}$\or $\Theta$\or $\mathrm{I}$\or $\mathrm{K}$\or $\Lambda$\or $\mathrm{M}$\or $\mathrm{N}$\or $\Xi$\or $\mathrm{O}$\or $\Pi$\or $\mathrm{P}$\or $\Sigma$\or $\mathrm{T}$\or $\mathrm{Y}$\or $\Phi$\or $\mathrm{X}$\or $\Psi$\or $\Omega$\fi}\makeatother\AddEnumerateCounter{\greek}{\@greek}{24}\AddEnumerateCounter{\Greek}{\@Greek}{12}\newcolumntype{P}[1]{>{\centering\arraybackslash}p{#1}}
% Template:     Informe/Reporte LaTeX
% Documento:    Definición de entornos
% Versión:      5.4.0 (21/05/2018)
% Codificación: UTF-8
%
% Autor: Pablo Pizarro R. @ppizarror
%        Facultad de Ciencias Físicas y Matemáticas
%        Universidad de Chile
%        pablo.pizarro@ing.uchile.cl, ppizarror.com
%
% Manual template: [http://latex.ppizarror.com/Template-Informe/]
% Licencia MIT:    [https://opensource.org/licenses/MIT/]

\newenvironment{references}{\ifthenelse{\equal{\stylecitereferences}{bibtex}}{}{\throwerror{\references}{Solo se puede usar entorno references con estilo citas \noexpand\stylecitereferences=bibtex}}\begingroup\ifthenelse{\equal{\donumrefsection}{true}}{\section{\namereferences}}{\sectionanum{\namereferences}}\renewcommand{\section}[2]{}\begin{thebibliography}{99}}{\end{thebibliography}\endgroup}\newenvironment{anexo}{\begingroup\clearpage\phantomsection\ifthenelse{\equal{\showappendixsectitle}{true}}{\appendixpage}{}\appendixtitleon\appendicestocpagenum\appendixtitletocon\bookmarksetup{numbered,openlevel=0}\begin{appendices}\bookmarksetupnext{level=part}\ifthenelse{\equal{\showappendixsecindex}{true}}{}{\belowpdfbookmark{\nameappendixsection}{contents}}\setcounter{secnumdepth}{3}\setcounter{tocdepth}{3}\ifthenelse{\equal{\appendixindepobjnum}{true}}{\counterwithin{equation}{section}\counterwithin{figure}{section}\counterwithin{lstlisting}{section}\counterwithin{table}{section}}{}}{\end{appendices}\bookmarksetupnext{level=0}\endgroup}\lstnewenvironment{sourcecodep}[4][]{\emptyvarerr{sourcecodep}{#2}{Estilo no definido}\checkvalidsourcecodestyle{#2}\ifthenelse{\equal{\showlinenumbers}{true}}{\rightlinenumbers}{}\lstset{backgroundcolor=\color{\sourcecodebgcolor}}\ifthenelse{\equal{\codecaptiontop}{true}}{\ifx\hfuzz#4\hfuzz\ifx\hfuzz#3\hfuzz\lstset{style=#2}\else\lstset{style=#2,#3}\fi\else\ifx\hfuzz#3\hfuzz\lstset{caption={#4 #1},captionpos=t,style=#2}\else\lstset{caption={#4 #1},captionpos=t,style=#2,#3}\fi\fi}{\ifx\hfuzz#4\hfuzz\ifx\hfuzz#3\hfuzz\lstset{style=#2}\else\lstset{style=#2,#3}\fi\else\ifx\hfuzz#3\hfuzz\lstset{caption={#4 #1},captionpos=b,style=#2}\else\lstset{caption={#4 #1},captionpos=b,style=#2,#3}\fi\fi}}{\ifthenelse{\equal{\showlinenumbers}{true}}{\leftlinenumbers}{}}\lstnewenvironment{sourcecode}[3][]{\emptyvarerr{\equationresize}{#2}{Estilo no definido}\checkvalidsourcecodestyle{#2}\ifthenelse{\equal{\showlinenumbers}{true}}{\rightlinenumbers}{}\lstset{backgroundcolor=\color{\sourcecodebgcolor}}\ifthenelse{\equal{\codecaptiontop}{true}}{\ifx\hfuzz#3\hfuzz\lstset{style=#2}\else\lstset{style=#2,caption={#3 #1},captionpos=t}\fi}{\ifx\hfuzz#3\hfuzz\lstset{style=#2}\else\lstset{style=#2,caption={#3 #1},captionpos=b}\fi}}{\ifthenelse{\equal{\showlinenumbers}{true}}{\leftlinenumbers}{}}\newenvironment{itemizebf}[1][]{\begin{itemize}[font=\bfseries,#1]}{\end{itemize}}\newenvironment{enumeratebf}[1][]{\begin{enumerate}[font=\bfseries,#1]}{\end{enumerate}}\newenvironment{resumen}{\sectionfont{\color{\titlecolor} \fontsizetitle \styletitle \selectfont}\sectionanumnoiheadless{\nameabstract}}{\ifthenelse{\equal{\addemptypagetwosides}{true}}{\checkoddpage\ifoddpage\newpage\null\thispagestyle{empty}\newpage\addtocounter{page}{-1}\else\fi}{}}\newenvironment{images}[2][]{\def\envimageslabelvar {#1}\def\envimagescaptionvar {#2}\def\envimagesinitialized {true}\vspace{\margintopimages cm}\captionsetup{margin=\captionmarginmultimg cm}\begin{figure}[H] \centering\vspace{-\marginrightmultimage cm}\vspace{-\marginrightmultimage cm}\vspace{-\marginrightmultimage cm}}{\setcaptionmargincm{\captionlrmargin}\ifx\hfuzz\envimagescaptionvar\hfuzz\vspace{\captionlessmarginimage cm}\else\caption{\envimagescaptionvar\envimageslabelvar}\fi\end{figure}\setcaptionmargincm{\captionlrmargin}\vspace{\marginbottomimages cm}\def\envimagesinitialized {false}}\def\envimagesinitialized {false}



% CONFIGURACIÓN INICIAL DEL DOCUMENTO
% Template:     Informe/Reporte LaTeX
% Documento:    Configuración inicial del template
% Versión:      5.4.0 (21/05/2018)
% Codificación: UTF-8
%
% Autor: Pablo Pizarro R. @ppizarror
%        Facultad de Ciencias Físicas y Matemáticas
%        Universidad de Chile
%        pablo.pizarro@ing.uchile.cl, ppizarror.com
%
% Manual template: [http://latex.ppizarror.com/Template-Informe/]
% Licencia MIT:    [https://opensource.org/licenses/MIT/]

\checkvardefined{\titulodelinforme}\checkvardefined{\temaatratar}\checkvardefined{\autordeldocumento}\checkvardefined{\nombredelcurso}\checkvardefined{\codigodelcurso}\checkvardefined{\nombreuniversidad}\checkvardefined{\nombrefacultad}\checkvardefined{\departamentouniversidad}\checkvardefined{\localizacionuniversidad}\makeatletter\g@addto@macro\titulodelinforme\xspace\g@addto@macro\temaatratar\xspace\g@addto@macro\autordeldocumento\xspace\g@addto@macro\nombredelcurso\xspace\g@addto@macro\codigodelcurso\xspace\g@addto@macro\nombreuniversidad\xspace\g@addto@macro\nombrefacultad\xspace\g@addto@macro\departamentouniversidad\xspace\g@addto@macro\localizacionuniversidad\xspace\makeatother\ifthenelse{\isundefined{\tablaintegrantes}}{\errmessage{LaTeX Warning: Se borro la variable \noexpand\tablaintegrantes, creando una vacia}\def\tablaintegrantes {}}{}\ifthenelse{\equal{\cfgpdfsecnumbookmarks}{true}}{\bookmarksetup{numbered}}{}\ifthenelse{\equal{\cfgshowbookmarkmenu}{true}}{\def\cdfpagemodepdf {UseOutlines}}{\def\cdfpagemodepdf {UseNone}}\hypersetup{bookmarksopen={\cfgpdfbookmarkopen},bookmarksopenlevel={\cfgbookmarksopenlevel},bookmarkstype={toc},pdfauthor={\autordeldocumento},pdfcenterwindow={\cfgpdfcenterwindow},pdfcopyright={\cfgpdfcopyright},pdfcreator={LaTeX},pdfdisplaydoctitle={\cfgpdfdisplaydoctitle},pdffitwindow={\cfgpdffitwindow},pdfinfo={Curso.Codigo={\codigodelcurso},Curso.Nombre={\nombredelcurso},Documento.Autor={\autordeldocumento},Documento.Tema={\temaatratar},Documento.Titulo={\titulodelinforme},Template.Autor.Alias={ppizarror},Template.Autor.Email={pablo.pizarro@ing.uchile.cl},Template.Autor.Nombre={Pablo Pizarro R.},Template.Autor.Web={http://ppizarror.com/},Template.Codificacion={UTF-8},Template.Fecha={21/05/2018},Template.Latex.Compilador={pdflatex},Template.Licencia.Tipo={MIT},Template.Licencia.Web={https://opensource.org/licenses/MIT/},Template.Nombre={Template-Informe},Template.Tipo={Normal},Template.Version.Dev={5.4.0-N},Template.Version.Hash={261892B0B7A8D82869BFD359FC94DCE3},Template.Version.Release={5.4.0},Template.Web.Dev={https://github.com/Template-Latex/Template-Informe/},Template.Web.Manual={http://latex.ppizarror.com/Template-Informe/},Universidad.Departamento={\departamentouniversidad},Universidad.Nombre={\nombreuniversidad},Universidad.Ubicacion={\localizacionuniversidad}},pdfkeywords={\nombreuniversidad, \codigodelcurso \nombredelcurso, \localizacionuniversidad},pdflang={\documentlang},pdfmenubar={\cfgpdfmenubar},pdfpagelayout={\cfgpdfpagemode},pdfpagemode={\cdfpagemodepdf},pdfproducer={Template-Informe v5.4.0 | (Pablo Pizarro R.) ppizarror.com},pdfremotestartview={Fit},pdfstartpage={1},pdfstartview={\cfgpdfpageview},pdfsubject={\temaatratar},pdftitle={\titulodelinforme},pdftoolbar={\cfgpdftoolbar},pdfview={\cfgpdfpageview}}\graphicspath{{./\defaultimagefolder}}\newcounter{templatepagecounter}\renewcommand{\baselinestretch}{\defaultinterline}\setlength{\headheight}{64 pt}\setcounter{MaxMatrixCols}{20}\setlength{\footnotemargin}{4 mm}\setlength{\columnsep}{\columnsepwidth em}\ifthenelse{\equal{\showlinenumbers}{true}}{\setlength{\linenumbersep}{0.22cm}\renewcommand\linenumberfont{\normalfont\tiny\color{\linenumbercolor}}}{}\floatplacement{figure}{\imagedefaultplacement}\floatplacement{table}{\tabledefaultplacement}\floatplacement{tikz}{\tikzdefaultplacement}\color{\maintextcolor}\arrayrulecolor{\tablelinecolor}\sethlcolor{\highlightcolor}\ifthenelse{\equal{\showborderonlinks}{true}}{\hypersetup{citebordercolor=\numcitecolor,linkbordercolor=\linkcolor,urlbordercolor=\urlcolor}}{\hypersetup{hidelinks,colorlinks=true,citecolor=\numcitecolor,linkcolor=\linkcolor,urlcolor=\urlcolor}}\ifthenelse{\equal{\colorpage}{white}}{}{\pagecolor{\colorpage}}\setcaptionmargincm{\captionlrmargin}\ifthenelse{\equal{\captiontextbold}{true}}{\renewcommand{\captiontextbold}{bf}}{\renewcommand{\captiontextbold}{}}\captionsetup{labelfont={color=\captioncolor, \captiontextbold},textfont={color=\captiontextcolor},singlelinecheck=on}\floatsetup[figure]{captionskip=\captiontbmarginfigure pt}\floatsetup[table]{captionskip=\captiontbmargintable pt}\ifthenelse{\equal{\figurecaptiontop}{true}}{\floatsetup[figure]{position=above}}{}\ifthenelse{\equal{\tablecaptiontop}{true}}{\floatsetup[table]{position=top}}{\floatsetup[table]{position=bottom}}\ifthenelse{\equal{\captionalignment}{justified}}{\captionsetup{justification=justified,format=plain}}{\ifthenelse{\equal{\captionalignment}{centered}}{\captionsetup{justification=centering}}{\ifthenelse{\equal{\captionalignment}{left}}{\captionsetup{justification=raggedright,singlelinecheck=false}}{\ifthenelse{\equal{\captionalignment}{right}}{\captionsetup{justification=raggedleft,singlelinecheck=false}}{\throwbadconfig{Posicion de leyendas desconocida}{\captionalignment}{justified,centered,left,right}}}}}\ifthenelse{\equal{\stylecitereferences}{natbib}}{\bibliographystyle{apa}\setlength{\bibsep}{\natbibrefsep pt}}{\ifthenelse{\equal{\stylecitereferences}{apacite}}{\bibliographystyle{apacite}\setlength{\bibitemsep}{\apaciterefsep pt}}{\ifthenelse{\equal{\stylecitereferences}{bibtex}}{\bibliographystyle{apa}\newlength{\bibitemsep}\setlength{\bibitemsep}{.2\baselineskip plus .05\baselineskip minus .05\baselineskip}\newlength{\bibparskip}\setlength{\bibparskip}{0pt}\let\oldthebibliography\thebibliography\renewcommand\thebibliography[1]{\oldthebibliography{#1}\setlength{\parskip}{\bibitemsep}\setlength{\itemsep}{\bibparskip}}\setlength{\bibitemsep}{\bibtexrefsep pt}}{\throwbadconfig{Estilo citas desconocido}{\stylecitereferences}{bibtex,apacite,natbib}}}}\makeatletter\ifthenelse{\equal{\twocolumnreferences}{true}}{\renewenvironment{thebibliography}[1]{\begin{multicols}{2}[\section*{\refname}]\@mkboth{\MakeUppercase\refname}{\MakeUppercase\refname}\list{\@biblabel{\@arabic\c@enumiv}}{\settowidth\labelwidth{\@biblabel{#1}}\leftmargin\labelwidth\advance\leftmargin\labelsep\@openbib@code\usecounter{enumiv}\let\p@enumiv\@empty\renewcommand\theenumiv{\@arabic\c@enumiv}}\sloppy\clubpenalty 4000\@clubpenalty \clubpenalty\widowpenalty 4000\sfcode`\.\@m}{\def\@noitemerr{\@latex@warning{Ambiente `thebibliography' no definido}}\endlist\end{multicols}}}{}\makeatother\patchcmd{\appendices}{\quad}{\sectionappendixlastchar\quad}{}{}\begingroup\makeatletter\let\newcounter\@gobble\let\setcounter\@gobbletwo\globaldefs\@ne\let\c@loldepth\@ne\newlistof{listings}{lol}{\lstlistlistingname}\newlistentry{lstlisting}{lol}{0}\def\cftlstlistingindent {1.5em}\makeatother\endgroup\makeatletter\def\ifGm@preamble#1{\@firstofone}\appto\restoregeometry{\pdfpagewidth=\paperwidth\pdfpageheight=\paperheight}\apptocmd\newgeometry{\pdfpagewidth=\paperwidth\pdfpageheight=\paperheight}{}{}\makeatother\hfuzz=200pt \vfuzz=200pt\hbadness=\maxdimen \vbadness=\maxdimen\strictpagecheck\titlespacing{\section}{0pt}{20pt}{10pt}\titlespacing{\subsection}{0pt}{15pt}{10pt}\ttfamily \hyphenchar\the\font=`\-\urlstyle{tt}\ifthenelse{\equal{\portraitstyle}{style16}}{\coreimporttikz}{}\ifthenelse{\equal{\portraitstyle}{\bgtemplatetestcode}}{\coreimporttikz}{}\pdfminorversion=\pdfcompileversion\setcounter{tocdepth}{\indexdepth}

% INICIO DE LAS PÁGINAS
\begin{document}

% PORTADA
% Template:     Informe/Reporte LaTeX
% Documento:    Definición de portadas
% Versión:      5.4.0 (21/05/2018)
% Codificación: UTF-8
%
% Autor: Pablo Pizarro R. @ppizarror
%        Facultad de Ciencias Físicas y Matemáticas
%        Universidad de Chile
%        pablo.pizarro@ing.uchile.cl, ppizarror.com
%
% Manual template: [http://latex.ppizarror.com/Template-Informe/]
% Licencia MIT:    [https://opensource.org/licenses/MIT/]

% Template:     Informe/Reporte LaTeX
% Documento:    Configuraciones adicionales de portadas
% Versión:      5.4.0 (21/05/2018)
% Codificación: UTF-8
%
% Autor: Pablo Pizarro R. @ppizarror
%        Facultad de Ciencias Físicas y Matemáticas
%        Universidad de Chile
%        pablo.pizarro@ing.uchile.cl, ppizarror.com
%
% Manual template: [http://latex.ppizarror.com/Template-Informe/]
% Licencia MIT:    [https://opensource.org/licenses/MIT/]

% Configuración portada style15 [A]
\def\headerimageA {departamentos/uchile}            % Imagen en el header
\def\headerimagescaleA {0.4}                        % Escala de la imagen

% Configuración portada style16 [B]
\def\portraitbackgroundimageB {ejemplos/portada-1}  % Imagen de fondo
\def\portraitbackgroundcolorB {ocre}                % Color principal

% Configuración portada style17 [C]
\def\portraitimageC {img/ejemplos/test-image}       % Imagen de la portada
\def\portraitimageboxedC {true}                     % Imagen recuadrada
\def\portraitimageboxedwidthC {0.5}                 % Grosor línea recuadro
\def\portraitimagewidthC {8cm}                      % Ancho de la imagen en cm

% Configuración portada style18 [D]
\def\portraitimageD {img/ejemplos/test-image-wrap}  % Imagen de la portada
\def\portraitimageboxedD {false}                    % Imagen recuadrada
\def\portraitimageboxedwidthD {0.5}                 % Grosor línea recuadro
\def\portraitimagewidthD {4cm}                      % Ancho de la imagen en cm
\newpage\renewcommand{\thepage}{\nameportraitpage}
\ifthenelse{\equal{\portraitstyle}{style1}}{\setpagemargincm{\pagemarginleft}{\firstpagemargintop}{\pagemarginright}{\pagemarginbottom}\pagestyle{fancy}\fancyhf{}\fancyhead[L]{\nombreuniversidad ~ \\\nombrefacultad ~ \\\departamentouniversidad ~ \\\vspace{-0.43cm}}\fancyhead[R]{\includegraphics[scale=\imagendepartamentoescala]{\imagendepartamento}}~ \\\vfill\begin{center}\textcolor{\portraittitlecolor}{{\noindent \Huge{\titulodelinforme} \vspace{0.5cm}} ~ \\{\noindent \large{\temaatratar}}}\end{center}\vfill\noindent\begin{minipage}{1.0\textwidth}\begin{flushright}\tablaintegrantes\end{flushright}\end{minipage}}{\ifthenelse{\equal{\portraitstyle}{style2}}{\setpagemargincm{\pagemarginleft}{\firstpagemargintop}{\pagemarginright}{\pagemarginbottom}\pagestyle{fancy}\fancyhf{}\fancyhead[L]{\nombreuniversidad ~ \\\nombrefacultad ~ \\\departamentouniversidad ~ \\\vspace{-0.43cm}}\fancyhead[R]{\includegraphics[scale=\imagendepartamentoescala]{\imagendepartamento}}~ \\\vfill\begin{center}{\noindent \LARGE{\nombredelcurso} \vspace{0.3cm}} ~ \\\vspace*{1.5cm}\textcolor{\portraittitlecolor}{{\centering \noindent \Huge{\titulodelinforme} \vspace{0.3cm}} ~ \\{\noindent \large{\temaatratar}}}\end{center}\vfill\noindent\begin{minipage}{1.0\textwidth}\begin{flushright}\tablaintegrantes\end{flushright}\end{minipage}}{\ifthenelse{\equal{\portraitstyle}{style3}}{\setpagemargincm{\pagemarginleft}{\firstpagemargintop}{\pagemarginright}{\pagemarginbottom}\pagestyle{fancy}\fancyhf{}\fancyhead[L]{\nombreuniversidad ~ \\\nombrefacultad ~ \\\departamentouniversidad ~ \\\vspace{-0.43cm}}\fancyhead[R]{\includegraphics[scale=\imagendepartamentoescala]{\imagendepartamento}}~ \\\vfill\begin{center}\vspace*{-1.0cm}{\noindent \huge{\nombredelcurso} \vspace{0.3cm}} ~ \\{\noindent \large{Código del curso: \codigodelcurso}} ~ \\\vspace*{1.8cm}\textcolor{\portraittitlecolor}{{\noindent \Huge{\titulodelinforme} \vspace{0.3cm}} ~ \\{\noindent \large{\temaatratar}}}\end{center}\vfill\noindent\begin{minipage}{1.0\textwidth}\begin{flushright}\tablaintegrantes\end{flushright}\end{minipage}}{\ifthenelse{\equal{\portraitstyle}{style4}}{\setpagemargincm{\pagemarginleft}{\pagemargintop}{\pagemarginright}{\pagemarginbottom}\thispagestyle{empty}\vspace*{-1.5cm}\noindent \includegraphics[width=1.75cm]{departamentos/uchile2}\hspace*{-0.15cm}\begin{tabular}{l}\small \scshape{\MakeUppercase{\nombreuniversidad}} ~ \\\small \scshape{\MakeUppercase{\nombrefacultad}} ~ \\\small \scshape{\MakeUppercase{\departamentouniversidad}} ~ \\\vspace*{1.25cm}\mbox{}\end{tabular}\vfill\begin{center}{\fontsize{22pt}{10pt} \selectfont\noindent \textcolor{\portraittitlecolor}{\titulodelinforme} \vspace*{0.35cm}} ~ \\{\noindent \fontsize{10pt}{5pt} \selectfont  \textcolor{\portraittitlecolor}{\codigodelcurso\ - \nombredelcurso}} ~ \\\vspace*{3cm}\end{center}\vfill\noindent\begin{minipage}{1.0\textwidth}\begin{flushright}\tablaintegrantes\end{flushright}\end{minipage}}{\ifthenelse{\equal{\portraitstyle}{style5}}{\setpagemargincm{\pagemarginleft}{\pagemargintop}{\pagemarginright}{\pagemarginbottom}\thispagestyle{empty}\includegraphics[width=1.5cm]{departamentos/uchile3}\hspace{-0.2cm}\begin{tabular}{l}\small \scshape{\MakeUppercase{\nombreuniversidad}} ~ \\\small \scshape{\MakeUppercase{\nombrefacultad}} ~ \\\small \scshape{\MakeUppercase{\departamentouniversidad}} ~ \\\vspace*{1cm}\mbox{}\end{tabular}\vfill\begin{center}\fontsize{8mm}{9mm}\selectfont\textcolor{\portraittitlecolor}{\noindent \titulodelinforme ~ \\\noindent \temaatratar ~ \\}\vspace*{1cm}\footnotesize{\codigodelcurso\ - \nombredelcurso} ~ \\\vspace*{1.4cm}\end{center}\vfill\begin{center}\noindent \normalsize{\tablaintegrantes}\end{center}}{\ifthenelse{\equal{\portraitstyle}{style6}}{\setpagemargincm{\pagemarginleft}{\pagemargintop}{\pagemarginright}{\pagemarginbottom}\thispagestyle{empty}\begin{wrapfigure}{l}{0.3\textwidth}\vspace{-0.69cm}\noindent \hspace{-1.10cm} \includegraphics[scale=1.35]{departamentos/fcfm2}\end{wrapfigure}\hspace*{0.3cm}\noindent \textsc{\color{red} \hspace{-2.2cm} \departamentouniversidad} ~ \\\hspace*{0.3cm}\noindent \textsc{\color{dgray} \hspace{-1.6cm} \nombrefacultad} ~ \\\hspace*{0.3cm}\noindent \textsc{\color{dgray} \hspace{-1.6cm} \nombreuniversidad} ~ \\\hspace*{0.3cm}\noindent \textsc{\color{dgray} \hspace{-1.6cm} \codigodelcurso \nombredelcurso} ~ \\\vfill\begin{center}\vspace*{0.5cm}{\color{dgray} \Large \textbf{\MakeUppercase{\temaatratar}}} ~ \\\noindent \rule{\linewidth}{0.3mm} ~ \\\Huge \textup \bfseries \textsc{\textcolor{\portraittitlecolor}{\titulodelinforme}} ~ \\\noindent \rule{\linewidth}{0.3mm} ~ \\\end{center}\begin{minipage}{.5\textwidth}~\end{minipage}\vfill\begin{minipage}{1.0\textwidth}\begin{flushright}\noindent \tablaintegrantes\end{flushright}\end{minipage}}{\ifthenelse{\equal{\portraitstyle}{style7}}{\setpagemargincm{\pagemarginleft}{\pagemargintop}{\pagemarginright}{\pagemarginbottom}\thispagestyle{empty}\begin{center}\vspace*{-1.5cm}\includegraphics[scale=\imagendepartamentoescala]{\imagendepartamento}\hspace*{-0.15cm}\begin{tabular}{l}\vspace*{0.26cm}\mbox{} ~ \\\small \textsc{\MakeUppercase{\nombreuniversidad}} ~ \\\small \textsc{\MakeUppercase{\nombrefacultad}} ~ \\\small \textsc{\MakeUppercase{\departamentouniversidad}} ~ \\\vspace*{1.25cm}\mbox{}\end{tabular}\end{center}\vfill\begin{center}\noindent \rule{\textwidth}{0.4mm} \\ \vspace{0.3cm}{\huge \textcolor{\portraittitlecolor}{\titulodelinforme} \vspace{0.2cm} ~ \\}\noindent \rule{\textwidth}{0.4mm} ~ \\ \vspace{0.40cm}{\large \textcolor{\portraittitlecolor}{\temaatratar} ~ \\}\end{center}\vfill\noindent\begin{minipage}{1.0\textwidth}\begin{flushright}\scshape{\tablaintegrantes}\end{flushright}\end{minipage}}{\ifthenelse{\equal{\portraitstyle}{style8}}{\setpagemargincm{\pagemarginleft}{\pagemargintop}{\pagemarginright}{\pagemarginbottom}\thispagestyle{empty}\begin{center}\vspace*{-1.0cm}\begin{tabular}{c}\includegraphics[scale=\imagendepartamentoescala]{\imagendepartamento} \vspace{0.5cm} ~ \\\small \scshape{\MakeUppercase{\nombreuniversidad}} ~ \\\small \scshape{\MakeUppercase{\nombrefacultad}} ~ \\\small \scshape{\MakeUppercase{\departamentouniversidad}}\end{tabular}\end{center}\vfill\begin{center}\noindent \rule{\textwidth}{0.4mm} \\ \vspace{0.3cm}{\huge \textcolor{\portraittitlecolor}{\titulodelinforme} \vspace{0.2cm} ~ \\}\noindent \rule{\textwidth}{0.4mm} ~ \\ \vspace{0.40cm}{\large \textcolor{\portraittitlecolor}{\temaatratar} ~ \\}\end{center}\vfill\noindent\begin{minipage}{1.0\textwidth}\begin{flushright}\scshape{\tablaintegrantes}\end{flushright}\end{minipage}}{\ifthenelse{\equal{\portraitstyle}{style9}}{\setpagemargincm{\pagemarginleft}{\pagemargintop}{\pagemarginright}{\pagemarginbottom}\thispagestyle{empty}\noindent \includegraphics[scale=\imagendepartamentoescala]{\imagendepartamento}\vfill\begin{center}\noindent \rule{\textwidth}{0.4mm} \\ \vspace{0.3cm}{\huge \textcolor{\portraittitlecolor}{\titulodelinforme} \vspace{0.2cm} \\}\noindent \rule{\textwidth}{0.4mm} \\ \vspace{0.35cm}{\large \textcolor{\portraittitlecolor}{\temaatratar} \\}\end{center}\vfill\begin{center}\begin{tabular}{c}\small \scshape{\MakeUppercase{\nombreuniversidad}} ~ \\\small \scshape{\MakeUppercase{\nombrefacultad}} ~ \\\small \scshape{\MakeUppercase{\departamentouniversidad}}\end{tabular}\end{center}\vfill\begin{center}\indent \scshape{\tablaintegrantes}\end{center}}{\ifthenelse{\equal{\portraitstyle}{style10}}{\setpagemargincm{\pagemarginleft}{\pagemargintop}{\pagemarginright}{\pagemarginbottom}\thispagestyle{empty}~ \\\vfill\begin{center}\ifthenelse{\equal{\nombreuniversidad}{\xspace}}{\noindent {\large \textsc{\departamentouniversidad}}}{\noindent {\large \textsc{\nombreuniversidad, \departamentouniversidad}}}\vspace{1.0cm}\end{center}\vfill\begin{center}\noindent {\large \scshape{\nombredelcurso}} \vspace{0.5cm} ~ \\\noindent {\large \scshape{\codigodelcurso}} \vspace{0.5cm} ~ \\\noindent \rule{\textwidth}{0.4mm} \\ \vspace{0.3cm}{\huge \bfseries \textcolor{\portraittitlecolor}{\titulodelinforme} \vspace{0.2cm} \\}\noindent \rule{\textwidth}{0.4mm} \\ \vspace{2.5cm}\end{center}\vfill\begin{center}\indent \tablaintegrantes\end{center}\vfill~ \\}{\ifthenelse{\equal{\portraitstyle}{style11}}{\setpagemargincm{\pagemarginleft}{\pagemargintop}{\pagemarginright}{\pagemarginbottom}\thispagestyle{empty}\begin{center}\vspace*{-1.0cm}\scshape{\nombreuniversidad} ~ \\\scshape{\nombrefacultad} ~ \\\scshape{\departamentouniversidad}\end{center}\vfill\begin{center}{\setstretch{1.2} \fontsize{21pt}{22pt} \selectfont \textcolor{\portraittitlecolor}{\scshape{\titulodelinforme}} \vspace{0.5cm}} ~ \\{\fontsize{13pt}{10pt} \selectfont \textcolor{\portraittitlecolor}{\scshape{\temaatratar}}}\end{center}\vfill\begin{center}\indent \tablaintegrantes\end{center}}{\ifthenelse{\equal{\portraitstyle}{style12}}{\setpagemargincm{\pagemarginleft}{\pagemargintop}{\pagemarginright}{\pagemarginbottom}\thispagestyle{empty}\begin{center}\vspace*{-1.0cm}\includegraphics[scale=\imagendepartamentoescala]{\imagendepartamento}\end{center}\vfill\begin{center}{\bf \Huge \scshape{\textcolor{\portraittitlecolor}{\titulodelinforme}} \vspace{0.3cm}} \\{\bf \Large \textcolor{\portraittitlecolor}{\temaatratar}}\end{center}\vfill\begin{flushright}\noindent \tablaintegrantes\end{flushright}\vspace{0.5cm}\noindent \rule{\textwidth}{0.4mm}\begin{center}\ifthenelse{\equal{\nombreuniversidad}{\xspace}}{\scshape{\nombrefacultad} \\}{\scshape{\nombreuniversidad, \nombrefacultad} \\}\scshape{\departamentouniversidad}\end{center}}{\ifthenelse{\equal{\portraitstyle}{style13}}{\setpagemargincm{\pagemarginleft}{\pagemargintop}{\pagemarginright}{\pagemarginbottom}\thispagestyle{empty}\noindent\vspace*{-1.5cm}\begin{flushleft}\begin{minipage}{0.65\textwidth}\ifthenelse{\equal{\nombreuniversidad}{\xspace}}{{\fontsize{3.5mm}{0.5mm} \selectfont \noindent  \textsf{\nombrefacultad}} ~ \\}{{\fontsize{3.5mm}{0.5mm} \selectfont \noindent  \textsf{\nombreuniversidad, \nombrefacultad}} ~ \\}\noindent {\fontsize{3.0mm}{0.5mm} \selectfont \textsf{\departamentouniversidad} \vspace{-0.2cm}} ~ \\\noindent \textcolor{lgray}{\rule{\textwidth}{0.3mm}}\end{minipage}\end{flushleft}\vspace*{-2.15cm}\begin{flushright}\begin{minipage}{0.3\textwidth}\noindent \includegraphics[width=1.0\textwidth]{\imagendepartamento}\end{minipage}\end{flushright}\vfill\begin{center}\begin{minipage}{0.9\textwidth}\begin{framed}\LARGE\vspace{1cm}\centering \textcolor{\portraittitlecolor}{\textbf{\titulodelinforme}}\vspace{1cm}\end{framed}\end{minipage}\end{center}\vfill\begin{flushright}\noindent \textsf{\tablaintegrantes}\end{flushright}}{\ifthenelse{\equal{\portraitstyle}{style14}}{\setpagemargincm{\pagemarginleft}{\pagemargintop}{\pagemarginright}{\pagemarginbottom}\thispagestyle{empty}\noindent\begin{flushleft}\vspace*{-1.0cm}\noindent \includegraphics[scale=\imagendepartamentoescala]{\imagendepartamento} \\\end{flushleft}\vfill{\bf \huge \noindent \textcolor{\portraittitlecolor}{\textsf{\MakeUppercase{\titulodelinforme}} \vspace*{0.05cm}}} \\{\bf \large \noindent \textcolor{\portraittitlecolor}{\textsf{\MakeUppercase{\temaatratar}}}} \\\vfill\begin{flushright}\noindent \textsf{\tablaintegrantes}\end{flushright}}{\ifthenelse{\equal{\portraitstyle}{style15}}{\setpagemargincm{\pagemarginleft}{\pagemargintop}{\pagemarginright}{\pagemarginbottom}\thispagestyle{empty}\checkextravarexist{\headerimageA}{Defina la imagen extra de la portada en el archivo lib/page/portrait-config.tex (VERSION NORMAL) o bien en el bloque PORTADA (VERSION COMPACTA)}\checkextravarexist{\headerimagescaleA}{Defina la escala de la imagen extra de la portada en el archivo lib/page/portrait-config.tex (VERSION NORMAL) o bien en el bloque PORTADA (VERSION COMPACTA)}\vspace*{-1.5cm}\noindent \begin{minipage}{0.8\textwidth}\noindent \begin{minipage}{0.22\textwidth}\includegraphics[scale=1.0]{departamentos/fcfm2} \\\end{minipage}\begin{minipage}{0.6\textwidth}\begin{flushleft}\textsc{\begin{tabular}{l}{\small \nombreuniversidad} ~ \\{\small \nombrefacultad} ~ \\{\small \departamentouniversidad}\end{tabular}}\end{flushleft}\end{minipage}\end{minipage}\noindent \begin{minipage}{0.2\textwidth}\begin{flushright}\ifthenelse{\isundefined{\headerimageA}}{}{\ifthenelse{\isundefined{\headerimagescaleA}}{}{\noindent \includegraphics[scale=\headerimagescaleA]{\headerimageA} \\}}\end{flushright}\end{minipage}\vfill\begin{center}{\fontsize{25pt}{15pt} \selectfont \textcolor{\portraittitlecolor}{\textbf{\titulodelinforme}} \vspace{0.7cm}} \\{\Large \textcolor{\portraittitlecolor}{\temaatratar}}\end{center}\vfill\begin{center}\noindent \tablaintegrantes\end{center}}{\ifthenelse{\equal{\portraitstyle}{style16}}{\setpagemargincm{\pagemarginleft}{\pagemargintop}{\pagemarginright}{\pagemarginbottom}\checkextravarexist{\portraitbackgroundimageB}{Defina el fondo de la portada en el archivo lib/page/portrait-config.tex (VERSION NORMAL) o bien en el bloque PORTADA (VERSION COMPACTA)}\checkextravarexist{\portraitbackgroundcolorB}{Defina el color del bloque del titulo de la portada en el archivo lib/page/portrait-config.tex (VERSION NORMAL) o bien en el bloque PORTADA (VERSION COMPACTA)}\begingroup\thispagestyle{empty}\begin{tikzpicture}[remember picture,overlay]\node[inner sep=0pt] (background) at (current page.center) {\includegraphics[width=\paperwidth]{\portraitbackgroundimageB}};\draw (current page.center) node [fill=\portraitbackgroundcolorB!30!white,fill opacity=0.6,text opacity=1,inner sep=1cm]{\Huge\centering\bfseries\sffamily\parbox[c][][t]{\paperwidth}{\centering \textcolor{\portraittitlecolor}{\titulodelinforme} \\ [10pt]{\Large \temaatratar} \\ [25pt]{\huge \autordeldocumento}}};\end{tikzpicture}\vfill\endgroup}{\ifthenelse{\equal{\portraitstyle}{style17}}{\setpagemargincm{\pagemarginleft}{\firstpagemargintop}{\pagemarginright}{\pagemarginbottom}\pagestyle{fancy}\checkextravarexist{\portraitimageC}{[portrait-style17] Defina la imagen de la portada en el archivo lib/page/portrait-config.tex (VERSION NORMAL) o bien en el bloque PORTADA (VERSION COMPACTA)}\checkextravarexist{\portraitimageboxedC}{Defina si la imagen de la portada se encierra en un recuadro en el archivo lib/page/portrait-config.tex (VERSION NORMAL) o bien en el bloque PORTADA (VERSION COMPACTA)}\checkextravarexist{\portraitimageboxedwidthC}{Defina el grosor del recuadro de la imagen de la portada en el archivo lib/page/portrait-config.tex (VERSION NORMAL) o bien en el bloque PORTADA (VERSION COMPACTA)}\checkextravarexist{\portraitimagewidthC}{Defina los parametros de la imagen de la portada en el archivo lib/page/portrait-config.tex (VERSION NORMAL) o bien en el bloque PORTADA (VERSION COMPACTA)}\fancyhf{}\fancyhead[L]{\nombreuniversidad ~ \\\nombrefacultad ~ \\\departamentouniversidad ~ \\\vspace{-0.43cm}}\fancyhead[R]{\includegraphics[scale=\imagendepartamentoescala]{\imagendepartamento}}~ \\\vfill\begin{center}\textcolor{\portraittitlecolor}{{\noindent \Huge{\titulodelinforme} \vspace{0.5cm}} ~ \\{\noindent \large{\temaatratar}}}\end{center}~ \\\ifthenelse{\equal{\portraitimageboxedC}{true}}{\insertimageboxed{\portraitimageC}{width=\portraitimagewidthC}{\portraitimageboxedwidthC}{}}{\insertimage{\portraitimageC}{width=\portraitimagewidthC}{}}~ \\\vfill\noindent\begin{minipage}{1.0\textwidth}\begin{flushright}\tablaintegrantes\end{flushright}\end{minipage}}{\ifthenelse{\equal{\portraitstyle}{style18}}{\setpagemargincm{\pagemarginleft}{\firstpagemargintop}{\pagemarginright}{\pagemarginbottom}\pagestyle{fancy}\checkextravarexist{\portraitimageD}{[portrait-style17] Defina la imagen de la portada en el archivo lib/page/portrait-config.tex (VERSION NORMAL) o bien en el bloque PORTADA (VERSION COMPACTA)}\checkextravarexist{\portraitimageboxedD}{Defina si la imagen de la portada se encierra en un recuadro en el archivo lib/page/portrait-config.tex (VERSION NORMAL) o bien en el bloque PORTADA (VERSION COMPACTA)}\checkextravarexist{\portraitimageboxedwidthD}{Defina el grosor del recuadro de la imagen de la portada en el archivo lib/page/portrait-config.tex (VERSION NORMAL) o bien en el bloque PORTADA (VERSION COMPACTA)}\checkextravarexist{\portraitimagewidthD}{Defina los parametros de la imagen de la portada en el archivo lib/page/portrait-config.tex (VERSION NORMAL) o bien en el bloque PORTADA (VERSION COMPACTA)}\fancyhf{}\fancyhead[L]{\nombreuniversidad ~ \\\nombrefacultad ~ \\\departamentouniversidad ~ \\\vspace{-0.43cm}}\fancyhead[R]{\includegraphics[scale=\imagendepartamentoescala]{\imagendepartamento}}~ \\\ifthenelse{\equal{\portraitimageboxedD}{true}}{\insertimageboxed{\portraitimageD}{width=\portraitimagewidthD}{\portraitimageboxedwidthD}{}}{\insertimage{\portraitimageD}{width=\portraitimagewidthD}{}}\vfill\begin{center}\textcolor{\portraittitlecolor}{{\noindent \Huge{\titulodelinforme} \vspace{0.5cm}} ~ \\{\noindent \large{\temaatratar}}}\end{center}\vfill\noindent\begin{minipage}{1.0\textwidth}\begin{flushright}\tablaintegrantes\end{flushright}\end{minipage}}{\ifthenelse{\equal{\portraitstyle}{\bgtemplatetestcode}}{\setpagemargincm{\pagemarginleft}{\pagemargintop}{\pagemarginright}{\pagemarginbottom}\pagestyle{empty}\pagecolor{lbrown}\begin{center}\vspace*{-1.0cm}\scshape{\nombreuniversidad} ~ \\\scshape{\nombrefacultad} ~ \\\scshape{\departamentouniversidad}\end{center}~ \\\begin{center}\bgtemplatetestimg\end{center}\begin{center}\vspace*{-6cm}{\setstretch{1.2} \fontsize{25pt}{22pt} \selectfont \textcolor{\portraittitlecolor}{\scshape{\titulodelinforme}} \vspace{0.5cm}} \\{\fontsize{15pt}{10pt} \selectfont \textcolor{\portraittitlecolor}{\scshape{\temaatratar}}}\end{center}\vfill\begin{flushright}\noindent \tablaintegrantes\end{flushright}\newpage\pagecolor{white}}{\throwbadconfigondoc{Estilo de portada incorrecto}{\portraitstyle}{style1 .. style18}}}}}}}}}}}}}}}}}}}}\ifthenelse{\equal{\addemptypagetwosides}{true}}{\newpage\null\thispagestyle{empty}\renewcommand{\thepage}{}\newpage}{}


% CONFIGURACIÓN DE PÁGINA Y ENCABEZADOS
% Template:     Informe/Reporte LaTeX
% Documento:    Configuración de página
% Versión:      5.4.0 (21/05/2018)
% Codificación: UTF-8
%
% Autor: Pablo Pizarro R. @ppizarror
%        Facultad de Ciencias Físicas y Matemáticas
%        Universidad de Chile
%        pablo.pizarro@ing.uchile.cl, ppizarror.com
%
% Manual template: [http://latex.ppizarror.com/Template-Informe/]
% Licencia MIT:    [https://opensource.org/licenses/MIT/]

\newpage\ifthenelse{\equal{\romanpageuppercase}{true}}{\pagenumbering{Roman}}{\pagenumbering{roman}}\setcounter{page}{1}\setcounter{footnote}{1}\setpagemargincm{\pagemarginleft}{\pagemargintop}{\pagemarginright}{\pagemarginbottom}\def\arraystretch {\tablepadding}\def\pagewidth {0.995\linewidth}\ifthenelse{\equal{\pointdecimal}{true}}{\decimalpoint}{}\renewcommand{\appendixname}{\nomltappendixsection}\renewcommand{\appendixpagename}{\nameappendixsection}\renewcommand{\appendixtocname}{\nameappendixsection}\renewcommand{\contentsname}{\nomltcont}\renewcommand{\figurename}{\nomltwfigure}\renewcommand{\listfigurename}{\nomltfigure}\renewcommand{\listtablename}{\nomlttable}\renewcommand{\lstlistingname}{\nomltwsrc}\renewcommand{\lstlistlistingname}{\nomltsrc}\renewcommand{\refname}{\namereferences}\renewcommand{\tablename}{\nomltwtable}\sectionfont{\color{\titlecolor} \fontsizetitle \styletitle \selectfont}\subsectionfont{\color{\subtitlecolor} \fontsizesubtitle \stylesubtitle \selectfont}\subsubsectionfont{\color{\subsubtitlecolor} \fontsizesubsubtitle \stylesubsubtitle \selectfont}\ifthenelse{\equal{\hfstyle}{style1}}{\pagestyle{fancy} \fancyhf{}\fancyhead[L]{\nouppercase{\rightmark}}\fancyhead[R]{\small \rm \thepage}\fancyfoot[L]{\small \rm \textit{\titulodelinforme}}\fancyfoot[R]{\small \rm \textit{\codigodelcurso \nombredelcurso}}\renewcommand{\headrulewidth}{0.5pt}\renewcommand{\footrulewidth}{0.5pt}\renewcommand{\sectionmark}[1]{\markboth{#1}{}}}{\ifthenelse{\equal{\hfstyle}{style2}}{\pagestyle{fancy} \fancyhf{}\fancyhead[L]{\nouppercase{\rightmark}}\fancyhead[R]{\small \rm \thepage}\fancyfoot[L]{\small \rm \textit{\titulodelinforme}}\fancyfoot[R]{\small \rm \textit{\codigodelcurso \nombredelcurso}}\renewcommand{\headrulewidth}{0.5pt}\renewcommand{\footrulewidth}{0pt}\renewcommand{\sectionmark}[1]{\markboth{#1}{}}}{\ifthenelse{\equal{\hfstyle}{style3}}{\pagestyle{fancy} \fancyhf{}\fancyhead[L]{\small \rm \textit{\codigodelcurso \nombredelcurso}\vspace{0.04cm}}\fancyhead[R]{\includegraphics[width=1.2cm]{departamentos/fcfm2}\vspace{-0.10cm}}\fancyfoot[C]{\thepage}\renewcommand{\headrulewidth}{0.5pt}\renewcommand{\footrulewidth}{0pt}}{\ifthenelse{\equal{\hfstyle}{style4}}{\pagestyle{fancy} \fancyhf{}\fancyhead[L]{\nouppercase{\rightmark}}\fancyhead[R]{}\fancyfoot[C]{\small \rm \thepage}\renewcommand{\headrulewidth}{0.5pt}\renewcommand{\footrulewidth}{0pt}\renewcommand{\sectionmark}[1]{\markboth{#1}{}}}{\ifthenelse{\equal{\hfstyle}{style5}}{\pagestyle{fancy} \fancyhf{}\fancyhead[L]{\codigodelcurso \nombredelcurso}\fancyhead[R]{\nouppercase{\rightmark}}\fancyfoot[L]{\departamentouniversidad, \nombreuniversidad}\fancyfoot[R]{\small \rm \thepage}\renewcommand{\headrulewidth}{0pt}\renewcommand{\footrulewidth}{0pt}\renewcommand{\sectionmark}[1]{\markboth{#1}{}}}{\ifthenelse{\equal{\hfstyle}{style6}}{\pagestyle{fancy} \fancyhf{}\fancyfoot[L]{\departamentouniversidad}\fancyfoot[C]{\thepage}\fancyfoot[R]{\nombreuniversidad}\renewcommand{\headrulewidth}{0pt}\renewcommand{\footrulewidth}{0pt}\setlength{\headheight}{49pt}}{\ifthenelse{\equal{\hfstyle}{style7}}{\pagestyle{fancy} \fancyhf{}\fancyfoot[C]{\thepage}\renewcommand{\headrulewidth}{0pt}\renewcommand{\footrulewidth}{0pt}\setlength{\headheight}{49pt}}{\ifthenelse{\equal{\hfstyle}{style8}}{\pagestyle{fancy} \fancyhf{}\fancyfoot[R]{\thepage}\renewcommand{\headrulewidth}{0pt}\renewcommand{\footrulewidth}{0pt}\setlength{\headheight}{49pt}}{\ifthenelse{\equal{\hfstyle}{style9}}{\pagestyle{fancy} \fancyhf{}\fancyhead[L]{\nouppercase{\rightmark}}\fancyhead[R]{}\fancyfoot[L]{\small \rm \textit{\titulodelinforme}}\fancyfoot[R]{\small \rm \thepage}\renewcommand{\headrulewidth}{0.5pt}\renewcommand{\footrulewidth}{0.5pt}\renewcommand{\sectionmark}[1]{\markboth{#1}{}}}{\ifthenelse{\equal{\hfstyle}{style10}}{\pagestyle{fancy} \fancyhf{}\fancyhead[L]{\nouppercase{\rightmark}}\fancyhead[R]{\small \rm \textit{\titulodelinforme}}\fancyfoot[L]{}\fancyfoot[R]{\small \rm \thepage}\renewcommand{\headrulewidth}{0.5pt}\renewcommand{\footrulewidth}{0.5pt}\renewcommand{\sectionmark}[1]{\markboth{#1}{}}}{\throwbadconfigondoc{Estilo de header-footer incorrecto}{\hfstyle}{style1 .. style10}}}}}}}}}}}\ifthenelse{\equal{\showlinenumbers}{true}}{\linenumbers}{}


% RESUMEN O ABSTRACT
\begin{resumen}
	\newp Por definir.
\end{resumen}

% TABLA DE CONTENIDOS - ÍNDICE
% Template:     Informe/Reporte LaTeX
% Documento:    Índice
% Versión:      5.4.0 (21/05/2018)
% Codificación: UTF-8
%
% Autor: Pablo Pizarro R. @ppizarror
%        Facultad de Ciencias Físicas y Matemáticas
%        Universidad de Chile
%        pablo.pizarro@ing.uchile.cl, ppizarror.com
%
% Manual template: [http://latex.ppizarror.com/Template-Informe/]
% Licencia MIT:    [https://opensource.org/licenses/MIT/]

\ifthenelse{\equal{\showindex}{true}}{\newpage\sectionfont{\color{\indextitlecolor} \fontsizetitlei \styletitlei \selectfont}\ifthenelse{\equal{\addindextobookmarks}{true}}{\belowpdfbookmark{\nomltcont}{contents}}{}\tocloftpagestyle{fancy}\ifthenelse{\equal{\showdotontitles}{true}}{\def\cftsecaftersnum {.}\def\cftsubsecaftersnum {.}\def\cftsubsubsecaftersnum {.}}{}\def\cftfigaftersnum {\charafterobjectindex\enspace}\def\cftsubfigaftersnum {\charafterobjectindex\enspace}\def\cfttabaftersnum {\charafterobjectindex\enspace}\def\cftlstlistingaftersnum {\charafterobjectindex\enspace}\renewcommand{\cftdot}{\charnumpageindex}\ifthenelse{\equal{\showlinenumbers}{true}}{\nolinenumbers}{}\ifthenelse{\equal{\equalmarginnumobject}{true}}{\ifthenelse{\equal{\showsectioncaption}{none}}{\def\cftdefautllistingnumwidth {2.3em}}{\ifthenelse{\equal{\showsectioncaption}{sec}}{\def\cftdefautllistingnumwidth {3.0em}}{\ifthenelse{\equal{\showsectioncaption}{ssec}}{\def\cftdefautllistingnumwidth {3.8em}}{\ifthenelse{\equal{\showsectioncaption}{sssec}}{\def\cftdefautllistingnumwidth {4.3em}}{\throwbadconfig{Valor configuracion incorrecto}{\showsectioncaption}{none,sec,ssec,sssec}}}}}\def\cftfignumwidth {\cftdefautllistingnumwidth}\def\cftsubfignumwidth {\cftdefautllistingnumwidth}\def\cfttabnumwidth {\cftdefautllistingnumwidth}\def\cftlstlistingnumwidth {\cftdefautllistingnumwidth}}{}\ifthenelse{\equal{\showindexofcontents}{true}}{\tableofcontents}{}\iftotalfigures\ifthenelse{\equal{\showindexoffigures}{true}}{\listoffigures}{}\fi\iftotaltables\ifthenelse{\equal{\showindexoftables}{true}}{\listoftables}{}\fi\iftotallstlistings\ifthenelse{\equal{\showindexofcode}{true}}{\lstlistoflistings}{}\fi\ifthenelse{\equal{\addemptypagetwosides}{true}}{\vfill\checkoddpage\ifoddpage\else\newpage\null\thispagestyle{empty}\newpage\addtocounter{page}{-1}\fi}{}}{}
 % Índice, se puede borrar

% CONFIGURACIONES FINALES
% Template:     Informe/Reporte LaTeX
% Documento:    Configuraciones finales
% Versión:      5.4.0 (21/05/2018)
% Codificación: UTF-8
%
% Autor: Pablo Pizarro R. @ppizarror
%        Facultad de Ciencias Físicas y Matemáticas
%        Universidad de Chile
%        pablo.pizarro@ing.uchile.cl, ppizarror.com
%
% Manual template: [http://latex.ppizarror.com/Template-Informe/]
% Licencia MIT:    [https://opensource.org/licenses/MIT/]

\markboth{}{}\newpage\ifthenelse{\equal{\hfstyle}{style1}}{\fancyhead[L]{\nouppercase{\leftmark}}}{}\ifthenelse{\equal{\hfstyle}{style2}}{\fancyhead[L]{\nouppercase{\leftmark}}}{}\ifthenelse{\equal{\hfstyle}{style4}}{\fancyhead[L]{\nouppercase{\leftmark}}}{}\ifthenelse{\equal{\hfstyle}{style5}}{\fancyhead[R]{\nouppercase{\leftmark}}}{}\ifthenelse{\equal{\hfstyle}{style9}}{\fancyhead[L]{\nouppercase{\leftmark}}}{}\ifthenelse{\equal{\hfstyle}{style10}}{\fancyhead[L]{\nouppercase{\leftmark}}}{}\sectionfont{\color{\titlecolor} \fontsizetitle \styletitle \selectfont}\subsectionfont{\color{\subtitlecolor} \fontsizesubtitle \stylesubtitle \selectfont}\subsubsectionfont{\color{\subsubtitlecolor} \fontsizesubsubtitle \stylesubsubtitle \selectfont}\ifthenelse{\equal{\showsectioncaption}{none}}{}{\ifthenelse{\equal{\showsectioncaption}{sec}}{\counterwithin{equation}{section}\counterwithin{figure}{section}\counterwithin{lstlisting}{section}\counterwithin{table}{section}}{\ifthenelse{\equal{\showsectioncaption}{ssec}}{\counterwithin{equation}{subsection}\counterwithin{figure}{subsection}\counterwithin{lstlisting}{subsection}\counterwithin{table}{subsection}}{\ifthenelse{\equal{\showsectioncaption}{sssec}}{\counterwithin{equation}{subsubsection}\counterwithin{figure}{subsubsection}\counterwithin{lstlisting}{subsubsection}\counterwithin{table}{subsubsection}}{\throwbadconfig{Valor configuracion incorrecto}{\showsectioncaption}{none,sec,ssec,sssec}}}}}\renewcommand{\thepage}{\arabic{page}}\setcounter{page}{1}\setcounter{section}{0}\setcounter{footnote}{0}\ifthenelse{\equal{\showlinenumbers}{true}}{\linenumbers}{}


% ======================= INICIO DEL DOCUMENTO =======================

% Template:     Informe/Reporte LaTeX
% Documento:    Archivo de ejemplo
% Versión:      5.1.5 (15/05/2018)
% Codificación: UTF-8
%
% Autor: Francisco Muñoz P.
%        Facultad de Ciencias Físicas y Matemáticas
%        Universidad de Chile
%        fjmunoz95@gmail.com, frmunoz@dcc.uchile.cl
%
% Derechos del template corresponden a Pablo Pizarro. pablo.pizarro@ing.uchile.cl
% Manual template: [http://latex.ppizarror.com/Template-Informe/]
% Licencia MIT:    [https://opensource.org/licenses/MIT/]

% define some usefull commands
\newcommand\FF{\mathcal{F}}
\newcommand\prd{Phys. Rev. D}
\newcommand\MF{\text{MF}}
\newcommand\IL{\int_{-\infty}^{\infty}}
\newcommand\WM{\text{WMF}}
\newcommand\wmf[2]{(#1|#2_{\MF})^{\WM}_{t_0}}
% NUEVA SECCIÓN
\section{Introduction}{\label{sec:intro}}
\newp por definir.

\section{Fourier Transform}{\label{sec:ft}}
\newp The Fourier Transform it's a Mathematical tool that allow us to decompone singals by a given dictionary or in other words, to apply a change of basis to a given signal. Thinking on a continuous space for a signal $s(t)$, it's Fourier Transform is given by the following integral, where $i =  \sqrt{-1}$ denotes the imaginary unit:
\insertequation[\label{ft:ft}]{\tilde{s}(f) = \int_{-\infty}^{\infty} s(t)e^{-2i\pi f t} dt = \FF[s(t)]}

\newp And the inverse transform is:

\insertequation[\label{ft:ift}]{s(t) = \int_{-\infty}^{\infty} \tilde{s}(f)e^{2i\pi f t } df = \FF^{-1}[\tilde{s}(f)]}

\subsection{Properties of the Fourier Transform}{\label{sec:ft::subsec:propft}}
\newp Some usefull properties of the Fourier Transform are:
	\begin{description}
	
	\item \textbf{Fourier Transform is a linear operator}. This means that, given a constant $a$ and 		functions $s(t)$, $h(t)$, we can write:
	\insertequation[\label{propFT:linear1}]{\FF[s(t) + h(t)] =  \FF[s(t)] + \FF[h(t)]} 
	\insertequation[\label{propFT:linear2}]{\FF[as(t)] = a\FF[s(t)] }

	\item \textbf{A time-shift impart a phase in the fourier Transform.} For a given function $s(t)$, it's satisfied:
	\insertequation[\label{propFT:time-shift}]{\FF[s(t - t_0)] = \FF[s(t)]e^{-2i\pi f t_0}}
	
	\newp also can be obtained a conjugate of the transform by:
	
	\insertequation[\label{propFT:time_shift2}]{\FF[s(t_0 - t)] = \FF[s(t)]^{*}e^{-2\pi f t_0}}

	\item \textbf{The Power Spectrum equal to square of absolute value of the Fourier Transform.} This computation allow us to remove the phase and the complex part which is changed to real by the absolute function.
	
	\insertequation[\label{psd}]{\mathcal{P}_s = |\FF[s(t)]|^{2}}
	
	\item \textbf{Convolution Theorem.} The definition of convolution for two functions $s(t)$ and $h(t)$, is given by:
	
	\insertequation[\label{convolution1}]{[s(t) * h(t)] = [h(t) * s(t)] = \int_{-\infty}^{\infty} s(t-\tau) h(\tau) d\tau}
	
	Calculating the Fourier Transform of this convolution leads to:
	
	\insertequation[\label{convolution2}]{\FF[s * h] = \FF[s] \cdotp \FF[h]}
	
	Also, Fourier Transform of a point-wise multiplication of two functions leads to a convolution of Fourier Transforms:
	
	\insertequation[\label{convolution3}]{\FF[s(t) \cdotp h(t)](f) = \FF[s](f) * \FF[h](f)}
	
	In practice using Fourier to calculate a convolution is faster than just calculate the convolution, that's why this is a very important concept.
	
	\item \textbf{Parseval's Formula}, it says that for $s(t)$ and $h(t)$ square-integrable functions with fourier transforms given by (\ref{ft:ift}) and (\ref{ft:ft}), it's satisfied:
	
	\insertequation[\label{parseval}]{\int_{-\infty}^{\infty} s(t)h^{*}(t)dt = \int_{-\infty}^{\infty}\tilde{s}(f)\tilde{h}^{*}(f) df}
	
	 \item \textbf{Plancherel Theorem}, it's a direct consecuence of the Parseval Formula but in the sence of energy:
	 
	 \insertequation[\label{plancherel}]{\int_{-\infty}^{\infty} \mid x(t) \mid^{2} (t)dt = \int_{-\infty}^{\infty}\mid \tilde{s}(f)\mid^{2} df} 	
	
	\item \textbf{Correlation.} The cross-correlation of two real signals $s(t)$ and $h(t)$ can be defined in the sense of a convolution by:
	
	\insertequation[\label{corr_as_conv}]{R_{hs}(t) = h \star s = h^{*}(-t) \ast s(t) = s(t) \ast h^{*}(-t)}
	
	\newp We are more interested in real signals when $s^{*}(-t) = s(-t)$ but the calculations will be in the general case. Using (\ref{convolution1}) we get:
	
	\insertequation[\label{corr1}]{R_{hs}(t) = \int_{-\infty}^{\infty} s(\tau - t) h^{*}(-t) dt}
	
	\newp Applying a simple change of variables($t \longrightarrow -t$) we get:
	
	\insertequation[\label{corr2}]{R_{hs}(t) = \int_{-\infty}^{\infty} s(\tau + t)h^{*}(t) dt}
	
	\newp Using the Time shift property (\ref{propFT:time-shift}) with the Parseval's Formula (\ref{parseval}):
	
	\insertequation[\label{corr3}]{R_{hs}(t) = \int_{-\infty}^{\infty} \tilde{s}(f)\tilde{h}^{*}(f)e^{2i\pi f \tau} df}
	
	\newp And this can be resumed as Fouriers Transforms:
	
	\insertequation[\label{corr4}]{R_{hs}(t) = \FF^{-1}[\FF[s] \FF[h]^{*}] = \FF^{-1}[\FF[s(t) \star h^{*}(-t)]]}
	
	\newp A particular behavior is when calculating the auto-correlation of a signal, which correspond to the Inverse Fourier Transform of the Power Spectral Density:
	
	 \insertequation[\label{autocorr}]{R_{hh} = \FF^{-1}[\mid \tilde{h}(f)\mid^{2}] = \FF^{-1}[\mathcal{P}_h(f)]}
	 
\end{description}

\subsection{Some useful Fourier Transforms}{\label{subsec_ftusefull}}
\begin{description}
	\item \textbf{Fourier Transform of a Sinusoid is a Delta Function.} \texttt{need to insert a figure in this section}
	\insertequation[\label{sindelta}]{\FF[Cos(2\pi f_0t)] = \frac{1}{2}[\delta(f - f_0) + \delta(f + f_0)]}
	
	\item \textbf{Fourier Transform of a Dicar Comb is a Dirac Comb.} The Dirac Comb is an infinite sequence of Dirac delta function placed at even intervals of size $T$:	

	\insertequation[\label{diraccomb}]{III_T(t) = \sum_{n=-\infty}^{\infty} \delta(t-nT)}
	
	thus, the fourier Transform correspond to a Dirac Comb in the frequency domain:
	
	\insertequation[\label{diracft}]{\FF[III_T(t)] = \frac{1}{T} III_{1/T}(f)}

\end{description}

\subsection{Window Function}{\label{subsec_window}}
\newp The Continuous Fourier Transform use the particularity that continuous functions are well defined for all times ($-\infty < t < \infty$). This is just an idealization that doesn't work on real signals because this only involve a finite span of time with some finite rate of sampling. The Fourier Transform of a discrete measured data will no longer be just the transform of the continuous underlying function, but rather the transform of the point-wise product of the signal($s(t)$) and the observing window($W(t)$), which will be a rectangular window of dirac comb.

\insertequation[\label{window1}]{s_{obs}(t) = s(t)W(t)}

\newp Using the convolution theorem (ec. \ref{convolution3}) the transform of eq. \ref{window1} will be:

\insertequation[\label{window2}]{\FF[s_{obs}(t)] = \FF[s(t)] * \FF[W(t)]}

\subsubsection{Fixing periodicity}{\label{subsubsec_fixperiod}}

\newp Fourier Transform works only for periodic signals, this means, for signals where its first point has exactly the same value as the last point (continuity in edges). But in real world, a measured signal has always some grade of noise which lead to a discontinuity in the edges. In order to fix this discontinuity we use a window function.

\newp For these cases, the window function will not be a rectangular, or at least, usualy we don't want to use this kind of window. The most useful windows are kind-of curve like \textit{Hann window}, \textit{Blackmann window} or \textit{Tukey window}. We are going to use mostly the \textit{Tukey window} which is: \texttt{should use a example plot here for all windows.}

\insertequation[\label{tukey}]{w(n) = \begin{cases}
\frac{1}{2} \left[1 + cos\left(\pi \left(\frac{2n}{\alpha (N-1} - 1  \right) \right) \right] & 0 \leq n < \frac{\alpha(N-1)}{2} \\
1 & \frac{\alpha(N-1)}{2} \leq n \leq (N-1)(1-\frac{\alpha}{2})\\
\frac{1}{2} \left[1 + cos\left(\pi \left(\frac{2n}{\alpha (N-1} - \frac{2}{\alpha} +  1  \right) \right) \right] & (N-1)(1 - \frac{\alpha}{2} < n \leq (N-1))
\end{cases}}

\newp Here, $\alpha$ define the form of the window, when $\alpha=0$ we get a rectangular window and for $\alpha=1$ we get a \textit{Hann window}.
\subsection{Nyquist-Shannon Sampling Theorem}{\label{subsec_NSST}}
\newp sampling is a process of converting a signal (function-like of continuous time and/or space) into a numeric sequence, thus the sampling theorem states that, if a function $f(x)$ contains no frequencies higher than B hertz, it is completely determined by givin its ordinates at a series of points spaced $1/(2B)$ seconds apart. This means that a suficient sample-rate for uniform sampling is therefore anything larget than $2B$ samples per second, or from other perspective, for a given sample rate $f_s$, \textbf{perfect reconstruction is guaranteed possible} for a bandlimit $B < f_s/2$.

\newp In the most general case of non-uniform sampling, this theorem states that a band-limited signal can be perfectly reconstructed from its samples if the \textbf{average sampling rate} satisfies the Nyquist condition. this means that even if uniformly spaced samples can be easier reconstruced (easier algorithms), it is not a necessary condition for perfect reconstruction.

\subsection{Nyquist frequency}{\label{subsec_nyqfreq}}
\newp The nyquist frequency is defined as half of the sampling rate  of a discrete signal when we have evenly-sampled data

\insertequation[\label{nyq1}]{\nu_N = 0.5 f_s}

\newp And correspond to the highest frequency limit of one interval over which a periodogram is uniquely defined. If we go far away from this nyquist limit, the periodogram (and the simple Fourier Transform) will have repeated information. 

\newp For unevenly-sampled data, the symmetry that allow to define this frequency as eq. \ref{nyq1} is broken and using an average sampling rate as in the Sampling Theorem doesn't work. In order to find a best approximation to the real nyquist frequency (called \textit{pseudo-nyquist frequency}) there are several methods in literature. The one implemented here correspond to the method delevoped by Koen \cite{koen-nyqFreq} who give a calculation formula to be solved for the nyquist frequency for arbitrary time spacing of measurements, this is the smallest positive root of

\insertequation[\label{koen1}]{SS(\nu) = \sum_{l=1}^{N-1}\sum_{k=l+1}^{N} [sin 2 \pi \nu (t_k - t_l)]^2 = 0}

\newp This calculation will take $O(N_{\nu}N^2)$ where $N_{\nu}$ correspond to the number of frequencies to try in the formula. Since is a very expensive calculation, generaly it is more useful to know what is the maximum frequency that we are interested in.

\subsection{Discrete Fourier Transform}{\label{subsec_dft}}
\newp We can pass from cotinuous space to discrete space by applying a Rectangular Window with a dirac comb (eq. \ref{window2}) This is because the delta serve to collapse the Fourier Integral into a Fourier Sumatory. Considering the most general case, where a continuous signal $s(t)$ is observed at an \textbf{irregular sampling} of size $N_t$, leads to dirac comb:

\insertequation[\label{irr-dirac-comb}]{III_{\{t_j\}}(t) = \sum_{j = 1}^{N_t} \delta(t - t_j)}

\newp Using this, the direct transform will be:

\insertequation[\label{to-discrete1}]{
\begin{aligned}[b]
\FF[s_{obs}(t)] = \tilde{s}_{k} & = \FF[s(t){III}_{\{t_j\}}(t)]  \\
\tilde{s}_{k} & = \int_{-\infty}^{\infty}s(t)\sum_{j = 2}^{N_t} \delta(t - t_j) e^{-2\pi i f t} dt \\
\tilde{s}_{k} & = \sum_{j = 1}^{N_t}\int_{-\infty}^{\infty}s(t)\delta(t - t_j) e^{-2\pi i f t} dt  \\
\tilde{s}_{k} & = \sum_{j = 1}^{N_t} s_je^{-2\pi i t_j k \Delta f}
\end{aligned}
}

\newp Following the same idea we can get the inverse transform, considering that the \textbf{frequencies will always be in a regular grid} so $f_k = k \Delta f$:

\insertequation[\label{to_discrete2}]{s_j = \Delta f \sum_{k = 1}^{N_f} \tilde{s}_k e^{2 \pi i t_j k \Delta f}}

\newp the total number of frequencies ($N_f$) and how we define the $\Delta f$ will be explained more clearly in the next sections.

\subsection{Frequency Sampling}{\label{subsec_freqsampling}}
\newp Since we are considering the most general case of Fourier Transform for non-uniform sampling, it's neccessary to define clearly the frequency grid, more precisely we need to define the frequency limit on the low end and the high end, in addition to define the grid spacing.
\begin{description}

\item \textbf{low-end frequency limit.} For a set of observations in a time interval of lenght $T$, a signal with frequency $1/T$ will complete exactly one oscillation in the observing window. Chosing a low-end frequency limit could be this frequency or just set the minimum frequency to zero for simplicity. Also is posible to give a custom low-end limit based on what frequencies we want to explore. If we have complex signals we will ned to define a negative low-end frequency limit.

\item \textbf{high-end frequency limit.} In order to not miss information that could be relevant, it's neccessary to look for the maximum frequency that could give us information, this is given by the nyquist frequency defined in section \ref{subsec_nyqfreq}. Also is posible to give a custom high-end limit based on what frequencies we want to explore.

\item \textbf{grid spacing.} Choose the correct grid spacing it's also an important point, if we use a too fine grid will leads to unnecessarily long computational times, while, use a grid too coarse leads to missing narrow peaks that fall between grid points. A razonable idea is to choose grid spacing smaller than the spected width of the periodogram peaks, these peaks has a widht of $\sim 1/T$ where $T$ is the duration of the window function. To ensure that our grid sufficiently samples each peak, we can over-sample by some factor ($n_0$) using a grid spacing of

\insertequation[\label{grid-spacing}]{\Delta f = \frac{1}{n_0 T}}

\newp And with this we can know the total number of frequencies to evaluate

\insertequation[\label{n-frequencies}]{N_f = n_0 T (f_{max} - f_{min})}

\newp choose the value of $n_0$ is kind-of arbitrary and we can find uses in literature in the range of $n_0 = 5$ to $n_0 = 10$ \cite{vanderplas-lombScargle}.

\end{description}

\section{Fast Fourier Transform}
\newp The normal Discrete Fourier Transform showed in section \ref{subsec_dft} takes times $O(N^2)$ to compute wheter is the direct or the inverse transform, this is a very expensive computation and can be improved to $O(N log(N))$ usign the algorithm of the Fast Fourier Transform(FFT). \texttt{NEED TO READ MORE ABOUT THE ALGORITHM}

\subsection{Non-uniform Fast Fourier Transform}
\newp The same idea used in the optimization of the Discrete Fourier Tranform for evenly-sampled data called FFT can be used for unevnely-sampled data and we call this Non-unifor Fast Fourier Transform, using the same considerations. \texttt{NEED TO EXPLORE MORE THE ALGORITHM}

\section{Fourier Transform as Signal Decomposition}
\newp An alternative approach to realize the Fourier Transfrom is to do a Signal Decomposition by regression, here we just take our data and find what coeficients give the best fit to the original data given a Dictionary which can be interpreted as a change of basis, if we use for the Dictionary the fourier Matrix, then the coeficients of the signal decomposition will be the values of the Fourier Transform.

\subsection{What is a Signal Decomposition?}
\newp given a signal $s \in \mathbb{R}^{N}$ we want to find a representative set of coeficients $\beta \in\mathbb{R}^{M}$ such that:

\insertequation[\label{signaldecom1}]{\hat{s} = \Phi \hat{\beta}}

\newp Where $\Phi \in \mathbb{R}^{N \times M}$ is known as the dictionary matrix. Every column of this dictionary matrix correspond to an \textit{atom}, i.e. the constituents of signal $s$. A dictionary is called \textit{complete} when $M = N$. $\Phi$ can be either pre-defined or learned from data. We will select the atoms previously, for example, in a Fourier Dictionary we choose the atoms by knowing what frequencies we want.
\newp Solve eq. \ref{signaldecom1} for a complete dictionary that is invertible means find $\hat{\beta}^{*}$ such that:

\insertequation[\label{signaldecom2}]{\hat{\beta}^{*} = \Phi^{-1} \hat{s}}

\newp When the matrix is not invertible, the solution to eq. \ref{signaldecom1} is obtained by solving the  Least Square problem:

\insertequation[\label{leastsquare}]{\underset{\hat{\beta}}{\text{min}} \frac{1}{2} \| \hat{s} - \Phi\hat{\beta} \|^{2}_{2}}

\newp Where $\| \cdotp \|_{2}^{2}$ means the square of the L2-norm, the solution to this problem is:

\insertequation[\label{pseudoinv1}]{\hat{\beta}^{*} = (\Phi^{T}\Phi)^{-1}\Phi^{T} \hat{s}}

\newp Which requieres $(\Phi^{T}\Phi)$ to be invertible, means, having linear independent columns. Te matrix $(\Phi^{T}\Phi)^{-1}\Phi^{T}$ correspond to the left Moore-Penrose pseudo-inverse of $\Phi$.

\newp Then complete dictionary is not often the better dictionary, here we are more interested in the overcomplete dictionary, this is a dictionary with more atoms than data samples ($M > N$). For overcomplete dictionaries eq. (\ref{signaldecom1}) is under-determined and many solutions may exists. When $\Phi$ is overcomplete, the product $(\Phi^{T}\Phi)$ is always a singular matrix and then (\ref{pseudoinv1}) cannot be compute. In order to get an optimal its neccessary to include a regularization coeficient which impose conditions to help choose one of the many solutions of the underdetermined system. There are two kind of regularization usualy used, one is smooth-ness and the other is spartisy, the first refers to the least-rough or least-complex solution, this condition usualy leads to dense solutions with almost all points non-zeros, and the second one refeers to solution with less non-zero points and thus a more sparse solution. here we will show many method that try to find a good $\beta$.

\subsubsection{Ridge Regression}
\newp this is an example of a regression that use as regularization smooth-ness, the ridge regression is defined as:

\insertequation[\label{ridge1}]{\underset{\hat{\beta}}{\text{min}}\alpha \| \beta \|^{2}_{2} + \| s - \Phi \beta \|_{2}^{2}}

\newp Where $\alpha > 0$ is the trade-off between reconstruction error and smooth-ness of the solution. Te solution to \ref{ridge1} is:

\insertequation[\label{ridge2}]{\beta^{*} = (\Phi^{T}\Phi + \alpha I)^{-1}\Phi^{T}s}

\textbf{\texttt{UNCOMPLETE, NEED TO EXPLAIN MORE METHODS}}

\subsection{Fourier Dictionary}

\newp The discrete Fourier Transform from section \ref{subsec_dft} can be interpreted as a change of basis of the signals which means change the time representation of the signal $\vec{s}$ to a frequency representation $vec{\beta}$ using a sine/cose basis. This will means that we can write:

\insertequation[\label{signaldecomp1}]{\vec{s} = \Phi \cdotp \vec{\beta}}

\newp Where $\Phi$ is the Dictionary/Basis to use in the change of representation. This Dictionary is the Fourier Matrix where every colummn (atom $\phi$) correspond to a sine/cosine wave of a given frequency $k\Delta f$:

\insertequation[\label{signaldecomp2}]{\phi_k = e^{2 \pi i t_j k \Delta f} \forall t_j}

 \newp and in order to represent Direct and Inverse Fourier Transform Correctly it's neccessary to use an orthogonal Basis which means an square matrix Dictionary with a defined Inverse, if this matrix is defined correctly (ofter in the evenly-sampled case) his adjoint will be his inverse in a way that allow us to define the Adjoint Transform and use it as the Inverse Transform.
 
\insertequation[\label{signaldecom3}]{\vec{\beta} = \Phi^{T} \cdotp \vec{s}}
 
\newp Sadly for unevenly-sampled data we usually want to compute a frequency grid higher than the time grid producing a oversampling Dictionary, and even if we choose the same number of frequencies than times, there is no guarantee that the resulting Fourier Matrix will be unitary and invertible in a way that his adjoint transform be his Inverse Transform. Because of this, we need to look for a way to compute the Pseudo-inverse of the dictionary ($\Phi^{+}$) instead of the Inverse, usualy we can compute this by Single Value Decompostion, where the left decomposition (known as ):

\insertequation[\label{pseudoinverse}]{\Phi^{+}}

\textbf{\texttt{UNCOMPLETE}}  

\section{Signal Detection}{\label{sec:signalDetect}}
\newp Once we know how to interprete correctly some datas in frequency domain, using Non Unifor Fast Fourier Transform or Signal Decomposition, we are ready to work on the signal detection, i.e., on the search of a known signal in the observed data. This idea comes from the idealization of our data as an underlying signal plus additive \textbf{Wide-Sense Stationary} (WSS) Gaussian noise, then it is posible, through use of linear filters, to find what is this underlaying signal by comparing the observed data to known waveforms. Linears filters are defined as convolution of input data $x(t)$ with an impulse response filter wich is just the time reverse of the template waveform $h(t)$, for this we define the product $(x|h)$:

\insertequation[\label{signalDetect:linearFilter}]{(x|h)_{t_0} = [x(t) * h(-t)](t_0) = \int_{-\infty}^{\infty} \tilde{x}(f)\tilde{h}^{*}(f) e^{2\pi i f t_0} df}

\newp Here we have used eq. (\ref{corr_as_conv}) to write this a correlation. It's also important to notice that in practice we should do a $causal$ time reverse for the impulse response $h(-t) \longrightarrow h(T-t)$ where $T$ is the total duration of the template waveform, but if we always use templates of same duration as the data window, then this factor is not important due to the periodicity in Fourier Transform leadin to $h(T-t)=h(-t)$.

\newp And the optimal Filter is just find the filter that optimize a given quantity. The theory that we are going to use correspond to the \textbf{Matched Filter} (MF) which maximize the \textbf{Signal-to-Noise Ratio} (SNR) of the observed data, and in order to do this we need to consider an input signal to the filter composed of \textbf{only White Gaussian Noise} which is a WSS Gaussian noise but with constant spectral density.

\newp Obviously the raw data detected by any deviced will have mostly colored Gaussian Noise, this means, a WSS Gaussian noise with non-constant spectral density, then it is necessary to do a previous filter which is called \textit{Whitening Filter} (WF) to clear the noise.

\subsection{Whitening Filter}{\label{sec:signalDetect::ssec:WF}}

\newp For an observed data $x(t)$ of additive WSS Gaussian colored noise $n(t)$, which could have or not an underlying signal $s(t)$, we define the whitening Filter with impulse response $h_w(-t)$ as the optimal filter when its output due to the input noise $n(t)$ is white, with variance $\sigma^{2}$. This can be write as:

\insertequation[\label{whiteningFilter:whitenData}]{(x|h_w)_{t_0} = (s|h_w)_{t_0} + (n|h_w)_{t_0} \text{such that} var = \mathbb{E}[(n|h_w)^{2}] = \sigma^{2}}

\newp Working a litle with this variance we will get:

\insertequation[\label{whiteningFilter:noiseVarAsCovarFunct}]{
\begin{aligned}[b]
\mathbb{E}[|(n|h_w)|^{2}]	& = \mathbb{E}[(n|h_w)(n|h_w)^{*}] \\
						& = \mathbb{E}\left[\int_{-\infty}^{\infty} n(t_0 + t_1)h_w^{*}(t_1) dt_1 \left(\int_{-\infty}^{\infty} n(t_0 + t_2) h_w^{*}(t_2) dt_2 \right)^{*}\right] \\
						& = \int_{-\infty}^{\infty} \int_{-\infty}^{\infty} h_w^{*}(t_1) \mathbb{E}[n(t_0 + t_1) n(t_0 + t_2)^{*}] h_w(t_2) dt_1 dt_2 \\
\end{aligned}}

\newp In order to work with the covariance $\mathbb{E}[n(t_0 + t_1) n(t_0 + t_2)^{*}]$ we need to set some considerations, first the noise used will be a 0 mean random noise, in a way that $\mathbb{E}[n(t)] = 0$, second, the noise will be invariant in time (stationary) so $n(t + t_0) = n(t)$, with this que covariance will be reduced to:

\insertequation[\label{whiteningFilter:noiseCovar}]{\mathbb{E}[n(t_0 + t_1) n(t_0 + t_2)^{*}] = \mathbb{E}[n(t_1)n^{*}(t_2)] = k_n(t_1, t_2) = k_n(\tau)}

\newp Where we have used $\tau = t_1 - t_2$, $k_n(\tau)$ correspond to the auto-correlation function of the noise. \textbf{Bochner's theorem} states that the covariance function of a stationary process can be represented as the fourier transform of a positive finite measure\cite{gauss-procss}:

\insertequation[\label{whiteningFilter:noiseCovTransf}]{k_n(\tau) = \int S_n(f)e^{2\pi i f \tau} df}

\newp This transform is defined as the two-sided power spectral density being a real and positive function with no zero values. Using this on eq. (\ref{whiteningFilter:noiseVarAsCovarFunct}) we get:

\insertequation[\label{whiteningFilter:noiseVar}]{
\begin{aligned}[b]
\mathbb{E}[|(n|h_w)|^{2}]	& = \int_{-\infty}^{\infty} \int_{-\infty}^{\infty} h_W^{*}(t_1) k_n(t_1 - t_2) h_w(t_2) dt_1 dt_2 \\
						& =\int_{-\infty}^{\infty} \int_{-\infty}^{\infty} h(t_2) \left( \int_{-\infty}^{\infty} S_n(f)e^{2\pi i f (t_1 - t_2)} df\right) h_w^{*}(t_1)  dt_1 dt_2 \\
						& = \int_{-\infty}^{\infty} \left( \int_{-\infty}^{\infty} h_w^{*}(t_1)e^{2\pi i f t_1} dt_1 \right) S_n(f) \left( \int_{-\infty}^{\infty} h_w(t_2) e^{-2\pi i f t_2} dt_2 \right) df \\
						& = \int_{-\infty}^{\infty}\tilde{h}_w^{*}(f) e^{-2\pi i f T} S_n(f) \tilde{h_w}(f) e^{2\pi i f T} df \\
						& = \int_{-\infty}^{\infty}\tilde{h}_w^{*}(f) S_n(f) h_w(f) df \\
						& = \int_{-\infty}^{\infty}S_n(f) |\tilde{h}_w(f)|^{2} df \\
\end{aligned}}

\newp With this, we define the Transfer function of the Whitening Filter impulse response as:

\insertequation[\label{whiteningFilter:optimalWhiteningFilter}]{S_n(f)|\tilde{h}_w(f)|^{2} = S_w(f) \:\: \textbf{such that} \:\: \sigma^{2} = \int_{-\infty}^{\infty} S_w(f)}

\newp this directly leads to:

\insertequation[\label{whiteningFilter:optimalWhiteningImpulseResponse}]{|\tilde{h}_w(f)|^{2} = S_w(f)/S_n(f)}

\newp In a window of a finite duration with a finite number of measures $N$ and a White Variance of $\sigma_w^{2}$, then $S_w(f) = \sigma_w^{2}/N = N_0$ and then:

\insertequation[\label{whiteningFilter:optimalWhiteningImpulseResponseValue}]{|\tilde{h}_w(f)|^{2} = \frac{N_0}{S_n(f)}}

\newp This definition of whitening will leads to a cleared signal of White noise with variance 1 and mean 0. After this, if we want to do another linear filter, we will need to normalize the impulse response by this whitening optimal filter too in order to get match the template waveform correctly.

\subsection{Matched Filter}{\label{sec:signalDetect::ssec:MF}}
\newp Matched filter correspond to find the optimal linear filter for maximizing the Signal To Noise Ratio (SNR) in the presence of additive white noise of a given variance $\sigma_w$ and 0 mean. For this we considers some data $x(t)$ with finite-energy which can be pure noise $x(t) = n(t)$ or a signal in addition to noise $x(t) = s(t) + n(t)$, this is taken as the input to a filter with impulse response $h_{\MF}(-t)$ and use the linear filter product defined in eq. (\ref{signalDetect:linearFilter})

\newp If we do a MF using two different templates waveform, with different energy, it make hard to say what lineal filter works better because our impulse response functions are unormalized. This is clearly seen from the noise, on the input the noise is white but on the output the noise is convolved with the template and thus is not white anymore, this introduce a change on variance:

\insertequation[\label{matchedFilter:variance}]{\sigma_{\MF} = \mathbb{E}[|(x|h_{\MF})|^{2}] = N_0 \IL |\tilde{h}_{\MF}(f)|^{2} df}

\newp Where we have used the variance of eq. (\ref{whiteningFilter:noiseVar}) with the constant white noise PSD $S_w(f) = N_0$. Then we just need to divide the matched filter by square root of this variance in order to get a normed SNR of normal distribution $\mathcal{N}(\mu_{MF} / \sigma_{MF}, 1)$ where $\mu_{MF}$ is:

\insertequation[\label{matchedFilter:mean}]{\mu_{\MF} = \mathbb{E}[(x|h_{\MF})] = \mathbb{E}[(n|h_{\MF})] + \mathbb{E}[(s|h_{\MF})] = 0 + (s|h_{\MF})_{t_0}}

\newp Here we have that $\mathbb{E}[(s|h_{\MF})] = (s|h_{\MF})_{t_0}$ because the linear filter product itself is an expected value of the output filter. This expected value is for a given $t_0$ where we supose it is produced the maximum SNR response which is writen as:

\insertequation[\label{matchedFilter:SNR}]{SNR_{t_0} = \frac{(x|h_{\MF})_{t_0}}{\sigma_{\MF}}}

\newp And the optimal matched filter is when the expected underlying waveform that comes with noise on the input signals is just equal to the template waveform. from eq. (\ref{matchedFilter:mean}) we can use $x(t)$ instead of $s(t)$ to estimate $\mu_{\MF}$

\subsubsection{scopes and limitations}{\label{sec:signalDetect::ssec:MF::sssec:scopes}}

\newp Here we like to define limits on the behavior of the SNR, first we consider the situation when the input data is only white noise. Using the Cauchy-Schwarz inequality  on eq. (\ref{matchedFilter:SNR}) with $|\tilde{x}(f)|^{2} \thickapprox \lambda S_n(f)$, where $\lambda$ depends on the normalization and method used to estimate the PSD, we get:

\insertequation[\label{matchedFilterScopes:limitOnlyNoise}]{|SNR_{t_0}| = \frac{|(x|h)_{t_0}|}{|\sigma_{\MF}|} = \frac{|(n|h)_{t_0}|}{|\sigma_{\MF}|} \leq \frac{\sqrt{\lambda \sigma_{w}} \sqrt{\int_{-\infty}^{\infty} |\tilde{h}_{\MF}(f)|^{2} df}}{\sqrt{N_0 \int_{-\infty}^{\infty} |\tilde{h}_{\MF}(f)|^{2} df}} \begin{array}{c} \text{Discrete} \\ = \\ case
\end{array} \sqrt{\frac{\lambda N}{\sigma_w}}}

\newp Here we have writen $N_0$ as the values on the discrete case $N_0 = \sigma_w^{2} / N$ of $N$ samples, we are going to continue using this but we will write the equations on continuous case. Eq. (\ref{matchedFilterScopes:limitOnlyNoise}) is a limit for every SNR produced when the input data is only white noise.

\newp Another important limit is to find the maximum value that a SNR can achieve for any type of input data (with or without underlying signal), again, using the Cauchy-Schwarz inequality, we get in this case:

\insertequation[\label{matchedFilterScopes:limitGeneral}]{|SNR_{t_0}| \leq \frac{\sqrt{\int_{-\infty}^{\infty} |\tilde{x}(f)|^{2} df} \sqrt{\int_{-\infty}^{\infty} |\tilde{h}_{\MF}(f)|^{2} df}}{\sqrt{N_0 \int_{-\infty}^{\infty} |\tilde{h}_{\MF}(f)|^{2} df}} \begin{array}{c} \text{Discrete} \\ = \\ case \end{array} \sqrt{\frac{E_x N}{\sigma_w^{2}}}}

\newp Where $E_x = \IL |\tilde{h}_{\MF}(f)|^{2} df = \IL |h_{\MF}(t)|^{2} dt$ is the energy of the input data, here we see that the maximum value possible of the SNR only depend on the input data (energy, number of samples and white noise variance) and not on the template waveform.

\subsection{Whitened Matched Filter}{\label{sec:signalDetect::ssec:WMF}}

\newp We can (and must) use the Whitening Filter (WF) together with the Matched Filter (MF) for our (guessed) input signal with colored WSS noise. Here we just need to apply WF to both the input data and the inpulse response on the MF:
\insertequation[\label{whitenedMatchedFilter:filterConvolution}]{(x|h_{\MF})^{\WM}_{t_0} = [\lbrace x * h_w\rbrace * \lbrace h_{\MF} * h_{w} \rbrace] = [x_w * h_{\WM}]}

\newp Using the definition of lineal filter on eq. (\ref{signalDetect:linearFilter}) with eq. (\ref{convolution2}), this will be:

\insertequation[\label{whitenedMatchedFilter:filterProduct}]{\wmf{x}{h} = \IL \tilde{x}(f) \tilde{h}_{\MF}^{*}(f)|\tilde{h}_w(f)|^{2} e^{2\pi i f t_0} df}

\newp Where $|\tilde{h}_w(f)|^{2}$ is given by eq. (\ref{whiteningFilter:optimalWhiteningImpulseResponseValue}) this leads to:

\insertequation[\label{whitenedMatchedFilter:filterProductValue}]{\wmf{x}{h} = N_0\IL \frac{\tilde{x}(f) \tilde{h}^{*}(f)}{S_n(f)} e^{2\pi i f t_0} df}

\newp This also affect the variance and the expected (value) remains with same definition as shown in eq. (\ref{matchedFilter:mean}) but for the product defined in eq. (\ref{whitenedMatchedFilter:filterProductValue}):

\insertequation[\label{whitenedMatchedFilter:VarianceFilter}]{\sigma_{\WM}^{2} = \mathbb{E}[|\wmf{n}{h}|^{2}] = N_0 \IL |\tilde{h}_{\MF}(f)|^{2}|\tilde{h}_{w}(f)|^{2} df = N_0^{2} \IL \frac{|\tilde{h}_{\MF}(f)|^{2}}{S_n(f)} df}

\insertequation[\label{whitenedMatchedFilter:ExpectedValueFilter}]{\mu_{\WM} = \mathbb{E}[\wmf{x}{h}] = \mathbb{E}[\wmf{x-n}{h}]}

\newp Here (and in the next sections) for simplicity we will write $\wmf{x}{h} = (x|h)$, $\sigma_{\WM} = \sigma$ and $\mu_{\WM} = \mu$.Thus the SNR for the Whitened Matched Filter (writen as $\rho$ for simplicity) will be:

\insertequation[\label{whitenedMatchedFilter:SNR-WMF}]{\rho_{t_0} = \frac{(x|h)}{\sigma}}

\newp This division doesn't depend on $N_0$ so clearly this matched filter doesnt work with White noise (and in next we will ignore this factor), instead, consider any WSS Gaussian noise. For this definition, the limit values of the matched filter defined in eq. (\ref{matchedFilterScopes:limitOnlyNoise}) and (\ref{matchedFilterScopes:limitGeneral}) should be the same but due to numerical computations, could be not exactly the same. This SNR has the same distribution than before $\mathcal{N}(\mu / \sigma, 1)$.

\section{Statistical Theory of Signal Detection}{\label{statTheoSigDet}}
\newp The observed signal carried an additive noise introduced by the detector and is usually modeled as a Wide-Sense Stationary (WSS) Gaussian random process, this means, the problem of extracting the signal from the noise (as shown in section (\ref{sec:signalDetect}) can be interpreted as a statistical one. The presence of signal changes the statistical characteristics of the data observed $x(t)$, affecting his Probability Density Function (PDF). 
\newp All the theory that will be described here are explained more extense in \citep{ligo-gauss-case, mit-course-sigSysAndInfer, ligo-1992}.
\newp We are going to work with statistical hypothesis testing and then define a likelihood ratio with a threshold value in order to decide if there is really a signal or not.

\subsection{Hypothesis test}

\newp The main idea here is to consider two scenarios (binary hypothesis)  one is the \textit{null hypothesis} $H_0$ which consider data absent of signal and \textit{alternative hypothesis} $H_1$ which consider a signal present.

\insertgather{\label{hypoTest:h0} H_0 : x(t) = n(t) \\
\label{hypoTest:h1} H_1: x(t) = n(t) + s(t)}

\newp The \textit{Binary Hypothesis test} or \textit{binary decision rule} consider 4 scenarios which can be representes as a confusion matrix. Two of these fourth posibilities are consider errors:
\begin{itemize}
\item $P(H_0|H_0)$, this is the probability of choose $H_0$ when $H_0$ is true. It's a \textbf{correct decision}.
\item $P(H_1|H_0)$, probability of choose $H_1$ when $H_0$ is true. This is a Type I error and it's called \textbf{false alarm probability}.
\item $P(H_0|H_1)$, probability of choose $H_0$ when $H_1$ is true. Type II error known as \textbf{False dismissal probability}.
\item $P(H_1|H_1)$, probability of choose $H_1$ when $H_1$ is true. It's a \textbf{correct decision}.
\end{itemize}
\newp The type II error is also the \textit{probability of detection} of the signal. For the Test scenario, the type I error is called \textit{significance of the test} and type II error is \textit{power of the test}. 
\newp Next, we need to find a test that is optimal. The main idea here, no mather what approach we use, is to calculate the \textbf{likelihood ratio}.
\subsection{likelihood ratio}
\newp From the binary hypothesis test, for a given measure $x$, we will have a probability $P(H_0|x)$ of $H_0$ being true and probability $P(H_1|x)$ of $H_1$ being true, oviously, if $H_1$ is actually true, then $P(H_1|x) > P(H_0|x)$, same for case of $H_0$ being true, this can be writen in the compact form:

\insertequation[\label{likelihood:desicion}]{P(H_1|x) \begin{array}{c}H_1 \\ > \\ < \\ H_0 \end{array} P(H_0|x)}

\newp Here is implicit that when $P(H_0|x) = P(H_1|x)$ we cannot know what hypothesis is correct. These conditional probabilities are using the posterior probability of the hypotesis, this is different from the \textit{a priori} probabilities of every hypothesis $P(H_0)$ and $P(H_1)$. In order to evaluate the \textit{posteriori} probabilites in (\ref{likelihood:desicion}) we use the Bayes' rule:

\insertequation[\label{likelihood:prioriDesicion}]{\frac{p_1f(x|H_1)}{f(x)} \begin{array}{c}H_1 \\ > \\ < \\ H_0 \end{array} \frac{p_0f(x|H_0)}{f(x)}}

\newp Where $f(x|H_j)$ is just the conditional PDF and $f(x)$ is the PDF of measure the observed data, since is the same for both hypothesis, we will not consider his values. On eq. (\ref{likelihood:prioriDesicion}) we have considered that $f(x|H_j) > 0$ for $j=0,1$. Next we can just rewrite this equation in order to get the likelihood ratio $\Lambda(x)$:

\insertequation[\label{likelihood:ratio}]{\Lambda(x) = \frac{f(x|H_1)}{f(x|H_0)} \begin{array}{c}H_1 \\ > \\ < \\ H_0 \end{array} \frac{p_0}{p_1} = \eta}

\newp Where $\eta$ is a \textbf{threshold} that can be defined in several ways, one can estimate the optimal case usign bayesian inference or Neyman-Pearson (NP) detection between many others. For now we will use the NP theory to define the detection and threshold.

\subsection{Neyman-Pearson Detection}
\newp This approach consider a context in which we want to maximize the probability of detection defined as:

\insertequation[\label{npDetection:probOfDetect}]{P_D = P(H_1|H_1) = \int_{D_1} f(x|H_1) dx}

\newp While keeping the probability of false alarm:

\insertequation[\label{npDetection:probOfFalseAlarm}]{P_{FA} = P(H_1|H_0) = \int_{D_1} f(x|H_0) dx = \alpha}

\newp Below a pre-specified leve $\alpha$. The values in $D_1$ are all values for which the likelihood function is higher than a given threshold $\eta$. Then the threshold is picked to provide the largest possible $P_D$ while ensuring that $P_{FA}$ is not larger than the pre-specified level. The smaller the $\eta$, the largest the desicion region $D_1$ and the value of $P_D$, but $P_{FA}$ grows as well. We can just pick the smaller threshold that satisfied the given bound of $P_{FA}$.

\subsection{NP Detection for Whitened Matched Filter}

\newp Remembering that we consider the noise as a WSS White Gaussian random process, we can modelate this noise as a normal distribution of 0 mean ($\mu = 0$) and a given variance $\sigma$ ($\mathcal{N}(0, \sigma^{2})$), thus,  for both hypothesis from eq. (\ref{hypoTest:h0}) and (\ref{hypoTest:h1}) $n(t)$ distribute as:

\insertequation[\label{npDetWMF:noiseDistr}]{n(t) \thicksim N(0, \sigma^{2}) \longrightarrow f(x) = \frac{1}{\sqrt{2 \pi \sigma^{2}}}e^{-\frac{x^{2}}{2\sigma^{2}}}}

\newp When the matched filter as defined in sec. (\ref{sec:signalDetect::ssec:WMF}) is applied, our hypothesis tests are:

\insertgather{\label{npDetWMF:H0} H_0: \:\: (x|h) = (n|h) \thicksim \mathcal{N}(0, \sigma) \\
\label{npDetWMF:H1} H_1 \:\: (x|h) = (n|h) + (s|h) \thicksim \mathcal{N}(\mu, \sigma)}

\newp then, for the SNR we normalize by square root of the variance, leading to hypothesis tests:

\insertgather{\label{npDetWMF:H0SNR} H_0: \:\: \frac{(x|h)}{\sigma} = \frac{(n|h)}{\sigma} \thicksim \mathcal{N}(0, 1) \\
\label{npDetWMF:H1SNR} H_1 \:\: \frac{(x|h)}{\sigma} = \frac{(n|h)}{\sigma} + \frac{(s|h)}{\sigma} \thicksim \mathcal{N}(\frac{\mu}{\sigma}, 1)}

\newp Here the product $(s|h)$ is optimal when the signal $s(t)$ is equal to the impulse response shifted by the optimal time of match $s(t) = h(t_0 - t)$, using the base definition of linear filter (\ref{signalDetect:linearFilter}) writen as a simple convolution (\ref{convolution1}), with the parseval's equation (\ref{parseval}) and the equations defined in sec. (\ref{sec:signalDetect::ssec:WMF}):

\insertequation[\label{npDetWMF:optimalMean}]{(s|h) = [h(t_0 - t) * h(-t)](t_0) = \IL h(t_0 - (t_0 - t))h(-t) dt = \IL |\tilde{h}(f)|^{2} df = (h|h)_{t_0=0} = \sigma^{2}}

\newp The expected value of the optimal filter is just the variance of the output filter. And the Probability Density Function (PDF) of normal distributions is:

\insertequation[\label{npDetWMF:PDF}]{f(\rho) = \frac{1}{\sqrt{2\pi}\sigma} e^{-\frac{(\rho - \mu)^{2}}{2\sigma^{2}}}}

\newp Using this PDF as a conditional PDF with the hypothesis from eq. (\ref{npDetWMF:H0}) and (\ref{npDetWMF:H0}), where $\mu = (s|h)$ for $H_1$, the likelihood ratio (\ref{likelihood:ratio}) is writen as: 

\insertequation[\label{npDetWMF:likelihoodRatio}]{
\begin{aligned}[b]
\Lambda\lbrace(x|h)\rbrace & = \frac{(\sqrt{2\pi})^{-1} exp\left( -\frac{((x|h) - (s|h))^{2}}{2\sigma^{2}}\right)}{(\sqrt{2\pi})^{-1} exp\left( -\frac{(x|h)^{2}}{2\sigma^{2}}\right)} \\
			   & = exp\left(\frac{2(x|h)(s|h) - (s|h)^{2}}{2 \sigma^{2}} \right) \\
			   & =  exp\left(\frac{(s|h)}{\mathbb{E}[(n|h)^{2}]} \lbrace (x|h) - \frac{1}{2}(s|h) \rbrace \right)
\end{aligned}}

\newp Te factor $(s|h) / \mathbb{E}[(n|h)^{2}]$ on the optimal case is simplified resulting in:

\insertequation[\label{npDetWMF:likelihoodRatioSimplified}]{\Lambda\lbrace(x|h)\rbrace = exp\left( \lbrace (x|h) - \frac{1}{2}(h|h) \rbrace \right)}

\newp This likelihood ratio is the same that the one used in \cite{ligo-gauss-case, ligo-1992} even if we derived the formulas from different perspectives. If we simplify this likelihood ratio by doing a natural logarithm, we get:

\insertequation[\label{npDetWMF:logLikelihoodRatio}]{ln[\Lambda\lbrace(x|h)\rbrace] =  \lbrace (x|h) - \frac{1}{2}(h|h) \rbrace \begin{array}{c}H_1 \\ > \\ < \\ H_0 \end{array} ln\left[ \frac{p_0}{p_1} \right] = ln[\eta]}


\newp And here que can write a critical criteria over Threshold to decide what hypothesis is correct: 


\insertequation[\label{npDetWMF:criteriaOverThreshold}]{(x|h)\begin{array}{c}H_1 \\ > \\ < \\ H_0 \end{array} ln[\eta] + \frac{1}{2}(h|h) = \eta'}

\newp Here we see that we can write the optimal threshold if we know what is the optimal and the \textit{a priori} probability of $H_0$ and $H_1$ (or at least is ratio), but this is not know for the mosts cases, so we need to estimate a threshold for a given constrains. Under the NP criteria, we choose this threshold $\eta'$ to maximize $P_D$ and satisfy the desired false alarm $P_{FA}$.

\newp We know that the solution for eq. (\ref{npDetection:probOfFalseAlarm}) is a \textit{complementary error function}:

\insertequation[\label{npDetWMF:falseAlarmSolution}]{P_{FA} = \int_{\eta'}^{\infty} \frac{1}{\sqrt{2\pi}\sigma} e^{-x^{2}/(2\sigma^{2})} dx = \text{erfc}\left( \frac{\eta'}{\sigma}\right)}
\newp Where $erfc()$ means the complementary error function. Then we can solve for $\eta'$ which leads to:

\insertequation[\label{npDetMF:thresholdFromFalseAlarm}]{\eta' = erfc^{-1}(P_{FA}) \sigma }

\newp With this threshold defined, we calculate the probability of detection (\ref{npDetection:probOfDetect}) which solution is:

\insertequation[\label{npDetWMF:probOfDetectSolution}]{P_{D} = \int_{\eta'}^{\infty} \frac{1}{\sqrt{2\pi}\sigma} e^{-(x - \mu)^{2}/(2\sigma^{2})} dx = \text{erfc}\left( \frac{\eta' - \mu}{\sigma}\right)}

\newp In terms of the False Alarm:

\insertequation[\label{npDetWMF:probOfDetectForGivenFalseAlarm}]{P_D = \text{erfc}\left(\frac{\sigma \text{erfc}^{-1}(O_{FA}) - \mu}{\sigma} \right) = \text{erfc}[\text{erfc}^{-1}(P_{FA}) - \mu / \sigma] = \text{erfc}\left[\text{erfc}^{-1}(P_{FA}) - \frac{(x|h)_{t_0}}{\sqrt{(h|h)_{t_0=0}}} \right]}

\newp This is the final value that we are going to use to determine if our SNR represent a detection or not, the parameters needed for this are the False alarm value $P_{FA}$ and the values calculated for the filter used to estimate the mean and variance. We can also define the counterpart, the probability of False alarm for a given Probability of detection:

\insertequation[\label{npDetWMF:probOfFalseAlarForGivenProbOfDetect}]{P_{FA} = \text{erfc}\left[\text{erfc}^{-1}(P_D) + \frac{(x|h)_{t_0}}{\sqrt{(h|h)_{t_0=0}}}\right]}

\newp What of these two probabilities we want to compute depend on what kind of event we observe.




\newpage
\nocite{*}
%\bibliography{references, revtex-custom}
%\bibliographystyle{plain}
%\printbibliography 
\begin{multicols}{2}[\printbibheading]
\printbibliography[heading=none]
\end{multicols}
 % contenido del reporte

% FIN DEL DOCUMENTO
\end{document}
